\documentclass[a4paper,11pt, reqno]{amsart}

%% Mise en page
	\usepackage{fullpage}

%% Pour AMS art
	%\usepackage[foot]{amsaddr}% Adresses en bas (avec AMSART)
	\calclayout % permet de centrer la page dans amsart

%% Appendice
	\usepackage[titletoc]{appendix}% Notations A, B etc dans l'appendice plutot que 1,2,3

%% langue
	\usepackage[english]{babel} %typos en anglais
	%\usepackage[french]{babel} \frenchbsetup{StandardLists=true} %typos en français
	

%% Encodage et accents
	\usepackage[utf8]{inputenc} %encodage utf8
	\usepackage[T1]{fontenc} %voir les accents
	
%% Couleurs
	\usepackage[dvips]{color}
	\usepackage[dvipsnames,svgnames]{xcolor}


%% Puces et environnements
	\usepackage{enumitem} % Permet de faire break après un itemize ou enumerate
	%\usepackage{enumerate} % permet de sp\'ecifier les option de enumarate, possibilit\'es A a I i or 1 %% Attention : problemes de compatibilités parfois
	\usepackage[normalem]{ulem} % Pour souligner
	\renewcommand{\labelitemi}{\textbullet} %Puces de itemize

	
%% Liens : les r\'ef\'erences, rappels de formules et de sections sont en couleurs
	\usepackage[colorlinks=true,linkcolor=MidnightBlue,citecolor=OliveGreen]{hyperref}
	
%% Maths
	\usepackage{amsmath,amsfonts,amssymb,amscd,stmaryrd,bbm,amsthm,mathrsfs}

%% Pour les insertions de figures
	\usepackage{graphicx} % pour includegraphics
	\usepackage{float} % Pour specifier les options en haut d'une figure [t] ou [h]
	\usepackage{multicol} % Plusieurs colonnes


%% Tikz
	%\usepackage{pgf,tikz,pgfplots}
	%\usetikzlibrary{shapes,snakes,patterns,arrows,automata,positioning,patterns,math}
	%\usepackage{standalone}
	%\usepackage{ifthen,calc}


%% Lettres de cal et bbm
	\newcommand{\cA}{\mathcal{A}}	\renewcommand{\AA}{\mathbbm{A}} 	
	\newcommand{\cB}{\mathcal{B}}	\newcommand{\BB}{\mathbbm{B}}
	\newcommand{\cC}{\mathcal{C}}	\newcommand{\CC}{\mathbbm{C}}
	\newcommand{\cD}{\mathcal{D}}	\newcommand{\DD}{\mathbbm{D}}
	\newcommand{\cE}{\mathcal{E}}	\newcommand{\EE}{\mathbbm{E}}
	\newcommand{\cF}{\mathcal{F}}	\newcommand{\FF}{\mathbbm{F}}
	\newcommand{\cG}{\mathcal{G}}	\newcommand{\GG}{\mathbbm{G}}
	\newcommand{\cH}{\mathcal{H}}	\newcommand{\HH}{\mathbbm{H}}
	\newcommand{\cI}{\mathcal{I}}	\newcommand{\II}{\mathbbm{I}}
	\newcommand{\cJ}{\mathcal{J}}	\newcommand{\JJ}{\mathbbm{J}}
	\newcommand{\cK}{\mathcal{K}}	\newcommand{\KK}{\mathbbm{K}}
	\newcommand{\cL}{\mathcal{L}}	\newcommand{\LL}{\mathbbm{L}}
	\newcommand{\cM}{\mathcal{M}}	\newcommand{\MM}{\mathbbm{M}}
	\newcommand{\cN}{\mathcal{N}}	\newcommand{\NN}{\mathbbm{N}}
	\newcommand{\cO}{\mathcal{O}}	\newcommand{\OO}{\mathbbm{O}}
	\newcommand{\cP}{\mathcal{P}}	\newcommand{\PP}{\mathbbm{P}}
	\newcommand{\cQ}{\mathcal{Q}}	\newcommand{\QQ}{\mathbbm{Q}}
	\newcommand{\cR}{\mathcal{R}}	\newcommand{\RR}{\mathbbm{R}}
	\newcommand{\cS}{\mathcal{S}}	\renewcommand{\SS}{\mathbbm{S}}
	\newcommand{\cT}{\mathcal{T}}	\newcommand{\TT}{\mathbbm{T}}
	\newcommand{\cU}{\mathcal{U}}	\newcommand{\UU}{\mathbbm{U}}
	\newcommand{\cV}{\mathcal{V}}	\newcommand{\VV}{\mathbbm{V}}
	\newcommand{\cW}{\mathcal{W}}	\newcommand{\WW}{\mathbbm{W}}
	\newcommand{\cX}{\mathcal{X}}	\newcommand{\XX}{\mathbbm{X}}
	\newcommand{\cY}{\mathcal{Y}}	\newcommand{\YY}{\mathbbm{Y}}
	\newcommand{\cZ}{\mathcal{Z}}	\newcommand{\ZZ}{\mathbbm{Z}}

%% Gras et Gothique
	\newcommand{\bA}{\mathbf{A}}	\newcommand{\fA}{\mathfrak{A}}
	\newcommand{\bB}{\mathbf{B}}	\newcommand{\fB}{\mathfrak{B}}
	\newcommand{\bC}{\mathbf{C}}	\newcommand{\fC}{\mathfrak{C}}
	\newcommand{\bD}{\mathbf{D}}	\newcommand{\fD}{\mathfrak{D}}
	\newcommand{\bE}{\mathbf{E}}	\newcommand{\fE}{\mathfrak{E}}
	\newcommand{\bF}{\mathbf{F}}	\newcommand{\fF}{\mathfrak{F}}
	\newcommand{\bG}{\mathbf{G}}	\newcommand{\fG}{\mathfrak{G}}
	\newcommand{\bH}{\mathbf{H}}	\newcommand{\fH}{\mathfrak{H}}
	\newcommand{\bI}{\mathbf{I}}	\newcommand{\fI}{\mathfrak{I}}
	\newcommand{\bJ}{\mathbf{J}}	\newcommand{\fJ}{\mathfrak{J}}
	\newcommand{\bK}{\mathbf{K}}	\newcommand{\fK}{\mathfrak{K}}
	\newcommand{\bL}{\mathbf{L}}	\newcommand{\fL}{\mathfrak{L}}
	\newcommand{\bM}{\mathbf{M}}	\newcommand{\fM}{\mathfrak{M}}
	\newcommand{\bN}{\mathbf{N}}	\newcommand{\fN}{\mathfrak{N}}
	\newcommand{\bO}{\mathbf{O}}	\newcommand{\fO}{\mathfrak{O}}
	\newcommand{\bP}{\mathbf{P}}	\newcommand{\fP}{\mathfrak{P}}
	\newcommand{\bQ}{\mathbf{Q}}	\newcommand{\fQ}{\mathfrak{Q}}
	\newcommand{\bR}{\mathbf{R}}	\newcommand{\fR}{\mathfrak{R}}
	\newcommand{\bS}{\mathbf{S}}	\newcommand{\fS}{\mathfrak{S}}	
	\newcommand{\bT}{\mathbf{T}}	\newcommand{\fT}{\mathfrak{T}}	
	\newcommand{\bU}{\mathbf{U}}	\newcommand{\fU}{\mathfrak{U}}
	\newcommand{\bV}{\mathbf{V}}	\newcommand{\fV}{\mathfrak{V}}
	\newcommand{\bW}{\mathbf{W}}	\newcommand{\fW}{\mathfrak{W}}
	\newcommand{\bX}{\mathbf{X}}	\newcommand{\fX}{\mathfrak{X}}
	\newcommand{\bY}{\mathbf{Y}}	\newcommand{\fY}{\mathfrak{Y}}
	\newcommand{\bZ}{\mathbf{Z}}	\newcommand{\fZ}{\mathfrak{Z}}
	
%% Notations mathematiques
	\newcommand{\eps}{\varepsilon}
	\newcommand{\dd}{\mathop{}\!\mathrm{d}}
	\newcommand{\ie}{{\it i.e.}}
	\newcommand{\norm}[1]{\left\lVert #1\right\rVert}
	\newcommand{\abs}[1]{\left\lvert #1\right\rvert}
	\newcommand{\red}[1]{{\color{red} #1}}
	\newcommand{\blue}[1]{\textcolor{blue}{#1}}
	\newcommand{\orange}[1]{{\color{orange} #1}}
	\newcommand{\ind}{\mathbbm{1}}
	\newcommand{\1}{\mathbbm{1}}
	\newcommand{\indic}{\mathbbm{1}}
	\let\div\relax
	\DeclareMathOperator{\div}{div}
	\newcommand{\assign}{:=}

%% Environnements 
	\theoremstyle{plain}
		\newtheorem{theorem}{Theorem}[section]
		\newtheorem{corollary}[theorem]{Corollary}
		\newtheorem{lemma}[theorem]{Lemma}
		\newtheorem{proposition}[theorem]{Proposition}
		\newtheorem{conjecture}[theorem]{Conjecture}
		\newtheorem{remark}[theorem]{Remark}

%% Ennonces normaux
	\theoremstyle{definition}
		\newtheorem{definition}[theorem]{Definition}
		\newtheorem{notation}[theorem]{Notation}
		\newtheorem{example}[theorem]{Example}		
		
\begin{document}

\red{Remark
\begin{itemize}
  \item $\cM^{+, p}_1 =\cM^+_p$
\end{itemize}}

\red{Il semblerait que dans
{\cite{escobedoDustSelfsimilaritySmoluchowski2006}} les solutions pour
l'{\'e}quation de Smoluchowski avec $K_{\alpha}$ soit d{\'e}j{\`a}
{\'e}tudi{\'e}es (mais pas la convergence de Smoluchowski). Voir aussi
{\cite{escobedoGelationCoagulationFragmentation2002}},
{\cite{barikMassconservingSolutionsSmoluchowski2020}}.

En fait ils travailles avec
\[ K (x, y) = x^{- \alpha} y^{\beta} + x^{\beta} y^{- \alpha} \]
$- 1 \le \alpha \le \beta < 1$ $\alpha + \beta \in [0, 2]$, ce qui
n'est pas v{\'e}rifi{\'e} pour nous.

En revanche, {\cite{fournierExistenceSelfSimilarSolutions2005}} (Equation 1.7)
ils construisent une solution avec $K_{\alpha}$. Remarquons qu'on devrait
pouvoir du coup {\'e}tendre nos r{\'e}sultats dans ce cas l{\`a}.

Par ailleurs Norris montre tout, sans vitesse pour notre modèle}

\section{Introduction}

There is a nice survey on fragmentation equation
{\cite{deaconuProbabilisticRepresentationsFragmentation2023}}, celebrated
review {\cite{aldousDeterministicStochasticModels1999}} on Smoluchowski and
coagulation processes.

For existence results on Smoluchowski equation :
{\cite{laurencotClassContinuousCoagulationFragmentation2000}}
{\cite{lambExistenceUniquenessResults2004}} (and the references therein) : our
kernel does not fulfill the conditions. One can also see
{\cite{niethammerSelfsimilarSolutionsFat2013}}.

\

Work by Norris in {\cite{norrisSmoluchowskisCoagulationEquation1999}}, which
fulfills all the conditions of our kernel. Nevertheless no speed of
convergence, and with another language. Existence, uniqueness and convergence
of Marcus-Lushnikov process towards Smoluchowski equation are done for the
weak topology (we are working in Wasserstein $1$ distance). No speed and no
other types of functionals.

\

For coagulation models with homogeneous rate and there link with
Smoluchowski equation : a series of work of Fournier and coauthors
{\cite{fournierStochasticCoalescenceHomogeneouslike2009}},
{\cite{fournierConvergenceMarcusLushnikov2004}}
{\cite{fournierStochasticCoalescents2006}}
{\cite{fournierDistanceCoagulation2006}}
{\cite{cepedaSmoluchowskisEquationRate2011a}}. Again, our kernel does not
fulfills the conditions. But also the work
{\cite{kyprianouUniversalityClassFragmentationcoalescence2018}} for the speed
of convergence in the constant kernel case

Nice survey work in {\cite{laurencotWeakCompactnessTechniques2015a}}

\

CLT : {\cite{deaconuPureJumpMarkov2002}} and
{\cite{kolokoltsovCentralLimitTheorem2008}} speed
{\cite{cepedaSmoluchowskisEquationRate2011a}} and
{\cite{kyprianouUniversalityClassFragmentationcoalescence2018}}

\

Stragegy of the proof : Use ideas from
{\cite{cepedaSmoluchowskisEquationRate2011a}} for the distance in
$\RR_+^{\ast}$ to have a Wasserstein distance adapted to our kernel.
Ideas from {\cite{norrisSmoluchowskisCoagulationEquation1999}} for existence
and uniqueness of the Smoluchowski equation, see also
{\cite{norrisClusterCoagulation2000}}. Ideas from
{\cite{kolokoltsovCentralLimitTheorem2008}} to use regularity with respect to
the initial condition of the Smoluchowski equation to compare the
Markus-Lushnikov process and solutions of the Smoluchowski equation. We also
use ideas of {\cite{kyprianouUniversalityClassFragmentationcoalescence2018}}
to derive a strong rate of convergence (in $1 / \sqrt{n}$). Finally we use
tools from {\cite{martiniKolmogorovEquationsSpaces2023}} to derive a rate of
convergence for non-linear functional of the empirical measure of the
Marcus-Lushnikov process.

\subsection{Plans of the Paper}

\subsection{Notations}

Let $(E, d_E)$ and $(F, d_F)$ be two metric spaces. We say that $f : E
\rightarrow F$ is Lipschitz continuous, and write $f \in  \text{ Lip } ((E, d_E)
; (F, d_F))$ if
\[ \llbracket f \rrbracket_{ \text{ Lip } ((E, d_E) ; (F, d_F))} \assign
   \underset{\begin{array}{c}
     x \neq y\\
     x, y \in E
   \end{array}}{\sup} \frac{d_F (f (x), f (y))}{d_E (x, y)} < + \infty . \]

\section{Functional spaces}

\subsection{A distance for the coagulation}

In this part, we define a topology on $\RR_+^{\ast}$ which is well
suited for the the kernels
\begin{equation}
  K_{\alpha} (x, y) = \frac{1}{x^{\alpha}} + \frac{1}{y^{\alpha}} .
  \label{eq:def-Kalpha}
\end{equation}
This will allow us to consider the convergence the empirical measure of the
coalescence process towards the Smoluchowski equation in some Wasserstein
spaces compatible with $K_{\alpha}$. We are strongly inspired by
{\cite{fournierDistanceCoagulation2006}}, where a smiliar distance is
constructed.

\begin{definition}
  Let $0 < \alpha < 1$. Let us define the distance on $\RR_+^{\ast}$
  via
  \[ d_{\alpha} (x, y) = \left| \frac{1}{x^{\alpha}} - \frac{1}{y^{\alpha}}
     \right| . \]
\end{definition}

\begin{proposition}
  Let $x_0 \in \RR$ and $r > 0$, then
  \[ B_{d_{\alpha}} (x_0, r) = \left\{\begin{array}{ll}
       \left( (r + x_0^{- \alpha})^{- \frac{1}{\alpha}}, (- r + x_0^{-
       \alpha})^{- \frac{1}{\alpha}} \right) &  \text{ if } r < x_0^{- \alpha}\\
       \left( (r + x_0^{- \alpha})^{- \frac{1}{\alpha}}, + \infty \right) &
        \text{ if } x_0^{- \alpha} \le r
     \end{array}\right. . \]
\end{proposition}

\begin{proof}
  Straightforward.
\end{proof}

\begin{corollary}
  The topology generated by $d_{\alpha}$ and the one generated by $| \cdot
  |$ coincide on $\RR_+^{\ast}$.
\end{corollary}

\begin{proof}
  The previous proposition shows that the open balls for $d_{\alpha}$ are open
  sets for $| \cdot |$. Now take $(a, b) \subset \RR_+^{\ast}$.
  Consider $r = \frac{a^{- \alpha} - b^{- \alpha}}{2}$ and $x_0 = \left(
  \frac{a^{- \alpha} + b^{- \alpha}}{2} \right)^{- \frac{1}{\alpha}}$, and
  \[ B_{d_{\alpha}} (r, x_0) = (a, b), \]
  which ends the proof.
\end{proof}

\begin{corollary}
  The space $(\RR_+^{\ast}, d_{\alpha})$ is a Polish space.
\end{corollary}

\begin{proof}
  Obviously $(\RR_+^{\ast}, | \cdot |)$ is homeomophic to
  $(\RR, | \cdot |)$, which is a locally compact metric space and
  countable at infinity, and thus is a Polish space. This implies that
  $(\RR_+^{\ast}, d_{\alpha})$ is a Polish space thanks to the previous
  Corollary.
\end{proof}

\begin{proposition}
  \label{prop:compact-dalpha}The closed intervals $[a, b]$ are compact sets
  for $d_{\alpha}$.
\end{proposition}

\begin{proof}
  First, remark that is $\RR_+^{\ast} \ni x_n \xrightarrow{d_a} x \in
  \RR_+^{\ast}$ if and only if $\left| \frac{1}{x_n^{\alpha}} -
  \frac{1}{x^{\alpha}} \right| \rightarrow 0$ if and only if $| x_n - x |
  \rightarrow 0$. Furthermore, remember that $[a, b] =
  \overline{B_{d_{\alpha}}} \left( \left( \frac{a^{- \alpha} + b^{-
  \alpha}}{2} \right)^{- \frac{1}{\alpha}}, \frac{a^{- \alpha} - b^{-
  \alpha}}{2} \right)$. Now, take a sequence $(x_n)_{n \ge 0}$ of
  elements of $[a, b]$. There exists a subsesquence and a point $x \in [a, b]$
  such that $| x_{k_n} - x | \rightarrow 0$ and $d_{\alpha} (x_{k_n}, x)
  \rightarrow 0$.
\end{proof}

\

In order to work with the Wasserstein topology on finite measures on
$(\RR_+^{\ast}, d_{\alpha})$, we need to study a little more the
Lipschitz continuous function of this space. First, remark that
\[ x \rightarrow \frac{1}{x^{\alpha}} \in  \text{ Lip } ((\RR_+^{\ast},
   d_{\alpha}) ; (\RR, | \cdot |)) . \]

In fact we have the following straightforward but useful property :

\begin{proposition}\label{prop:lip-isometry}
The spaces $Lip((\RR_+^*,d_\alpha);(\RR,|\cdot|))$ and $Lip((\RR_+^*,|\cdots|);(\RR,|\cdot|))$ are in bijection via the map 
\[f \in Lip((\RR_+^*,d_\alpha);(\RR,|\cdot|)) \to \left(x \to f\left(\frac{1}{x^{\frac{1}{\alpha}}}\right)\right) \in Lip((\RR_+^*,|\cdot|);(\RR,|\cdot|)),\]
with inverse
\[g \in Lip((\RR_+^*,|\cdot|);(\RR,|\cdot|)) \to \left(x \to g\left(\frac{1}{x^{\alpha}}\right)\right) \in Lip((\RR_+^*,d_\alpha);(\RR,|\cdot|)).\]
\end{proposition}

\begin{proof}
    Take $f \in Lip((\RR_+^*,d_\alpha);(\RR,|\cdot|))$, then 
    \begin{align*}
        \left|
            f\left(x^{-\frac{1}{\alpha}}\right)- f\left(y^{-\frac{1}{\alpha}}\right)
        \right| 
        \le  &
        \llbracket f \rrbracket_{Lip_{d_\alpha}} 
        d_{\alpha}
            \left(
                x^{-\frac{1}{\alpha}},
                y^{-\frac{1}{\alpha}}
            \right)
        \\ 
        = & 
        \llbracket f \rrbracket_{Lip_{d_\alpha}} |x-y|.
    \end{align*}
    The same computation holds for $f \in Lip((\RR_+^*,|\cdot|);(\RR,|\cdot|))$, one gets 
    \[\left|
            f\left(x^{-\alpha}\right)- f\left(y^{-\alpha}\right)
        \right| \le \llbracket f \rrbracket_{Lip_{|\cdot |}} d_\alpha(x,y)  .\]
        This ends the proof, using the fact that the two previously define maps are mutual inverses. 
\end{proof}
   
% \begin{proposition}
%   Take a Lipschitz continuous function $f \in  \text{ Lip } ((\RR_+^{\ast}, | \cdot |) ; (\RR, | \cdot |))$ then $f$
%   is a locally Lipschitz continuous from $(\RR_+, d_{\alpha})$ to
%   $(\RR, | \cdot |)$.
% \end{proposition}

% \begin{proof}
%   take $x > y \in \RR_+^{\ast}$, then
  
%   \begin{align*}
%     | f (x) - f (y) | \le & \llbracket f \rrbracket_{ \text{ Lip } ((\RR_+^{\ast}, | \cdot |) ; (\RR, | \cdot |))} | x - y
%     |\\
%     \le & \llbracket f \rrbracket_{ \text{ Lip } ((\RR_+^{\ast}, |
%     \cdot |) ; (\RR, | \cdot |))} \left| \frac{1}{x^{\alpha}} -
%     \frac{1}{y^{\alpha}} \right| \frac{| x - y |  (x y)^{\alpha} }{|
%     x^{\alpha} - y^{\alpha} |}\\
%     \le & \llbracket f \rrbracket_{ \text{ Lip } ((\RR_+^{\ast}, |
%     \cdot |) ; (\RR, | \cdot |))} d_{\alpha} (x, y) | x - y |^{1 -
%     \alpha} (x y)^{\alpha} .
%   \end{align*}
  
%   Here we have used the fact that $x^{\alpha} - y^{\alpha} \ge | x - y
%   |^{\alpha}$
  
%   In aparticular, for $x_0 \in \RR_+^{\ast}$ and $r = \frac{1}{2
%   x_0^{\alpha}}$, one gets $B_{d_{\alpha}} (x_0, r) = \left( \left(
%   \frac{2}{3} \right)^{\frac{1}{\alpha}} x_0, 2^{\frac{1}{\alpha}} x_0
%   \right)$, and for all $x > y \in B_{d_{\alpha}} (x_0, r)$,
%   \[ | x - y |^{1 - \alpha} (x y)^{\alpha} < \left( 2^{\frac{1}{\alpha}} -
%      \left( \frac{2}{3} \right)^{\frac{1}{\alpha}} \right)^{1 - \alpha} x_0^{1
%      - \alpha} 4 x_0^{2 \alpha} = 4 \left( 2^{\frac{1}{\alpha}} - \left(
%      \frac{2}{3} \right)^{\frac{1}{\alpha}} \right)^{1 - \alpha} x_0^{1 +
%      \alpha}, \]
%   and for all $x_0 \in \RR_+^{\ast}$, one gets
%   \[ \sup_{\substack{
%        x \neq y\\
%        x, y \in B_{d_{\alpha}} \left( x_0, \frac{1}{2 x_0^{\alpha}} \right)}} \frac{| f (x) - f (y) |}{d_{\alpha} (x, y)} \le 4
%      \left( 2^{\frac{1}{\alpha}} - \left( \frac{2}{3}
%      \right)^{\frac{1}{\alpha}} \right)^{1 - \alpha} x_0^{1 + \alpha}
%      \llbracket f \rrbracket_{ \text{ Lip } ((\RR_+^{\ast}, | \cdot |) ;
%      (\RR, | \cdot |))} \]
%   which proves the result.
% \end{proof}

\begin{proposition}
  Let $f \in  \text{ Lip } ((\RR_+^{\ast}, d_{\alpha}) ; (\RR, |
  \cdot |))$ such that $f (1) = 0$. Then
  \[ | f (x) | \le \llbracket f \rrbracket_{ \text{ Lip } ((\RR_+^{\ast}, d_{\alpha}) ; (\RR, | \cdot |))} \left( 1
     + \frac{1}{x^{\alpha}} \right) . \]
\end{proposition}

\begin{proof}
  The proof is a direct consequence of the definition of the distance.
\end{proof}

The following proposition will be of importance in the following, and quantify the rate of approximation of Hölder continuous functions (for $d_\alpha$) via Lipschitz continuous functions. This is a rather classical result when $d_\alpha$ is replaced by $|\cdot|$, but some arguments are necessary to state and prove it in the context of the wanted distance.

\begin{lemma}\label{lemma:holder-smooth-approx}
    Let $\kappa \in (0,1)$, $\delta>0$ and let $f\in \cC^{1-\kappa}((\RR_+^*,d_\alpha);(\RR,|\cdot|))$, that is 
    \[\llbracket f\rrbracket_{\cC^{1-\kappa}((\RR_+^*,d_\alpha);(\RR,|\cdot|))}:=\sup_{x\neq y \in \RR_+^*} \frac{|f(x)-f(y)|}{d_{\alpha}(x,y)^{1-\kappa}} < +\infty.\]
    There exists $f_\delta \in Lip((\RR_+^*,d_\alpha);(\RR,|\cdot|))$ such that 
    \[\sup_{x\in \RR_+^*} |f(x)-f_\delta(x)| \lesssim \llbracket f\rrbracket_{\cC^{1-\kappa}((\RR_+^*,d_\alpha);(\RR,|\cdot|))} \delta^{1-\kappa}\]
    and
    \[\llbracket f_\delta \rrbracket_{Lip((\RR_+^*,d_\alpha);(\RR,|\cdot|))} \lesssim \llbracket f\rrbracket_{\cC^{1-\kappa}((\RR_+^*,d_\alpha);(\RR,|\cdot|))} \delta^{-\kappa}.\]
\end{lemma}

\begin{proof}
    With a slight abuse of notations, we write $\cC^{1-\kappa}_{d_\alpha}=\cC^{1-\kappa}((\RR_+^*,d_\alpha);(\RR,|\cdot|))$ and $Lip_{d_\alpha}=Lip((\RR_+^*,d_\alpha);(\RR,|\cdot|))$.
    
    First, let us remark that if $f\in \cC^{1-\kappa}_{d_\alpha}$, then the function $g : x \to f\left(\frac{1}{x}\right)$ belongs to $\cC^{1-\kappa}((\RR_+^*,|\cdot|);(\RR,|\cdot|))$. Indeed, 
    \[
    |g(x)-g(y)| = \left|f\left(\frac{1}{x^{\frac{1}{\alpha}}}\right) - f\left(\frac{1}{y^{\frac1\alpha}}\right) \right| \le \llbracket f \rrbracket_{\cC^{1-\kappa}_{d_\alpha}} d_\alpha\left( \frac{1}{x^{\frac{1}{\alpha}}},\frac{1}{y^{\frac1\alpha}} \right)^{1-\kappa} = \llbracket f \rrbracket_{\cC^{1-\kappa}_{d_\alpha}} |x-y|^{1-\kappa}.
    \] 
    Since $f(x) =  g\left(\frac{1}{x^\alpha}\right)$
    we have the equality $\llbracket g \rrbracket_{\cC^{1-\kappa}_{|\cdot|}} = \llbracket f \rrbracket_{\cC^{1-\kappa}_{d_\alpha}}$. Hence, it is enough to prove the result for the functions $g : \cC^{1-\kappa}_{|\cdot|}$.

    Remark also that since 
    \[|f(x+h)-f(x)| \le \llbracket f \rrbracket_{\cC^{1-\kappa}_{d_\alpha}} |\left| \frac{1}{(x+h)^\alpha} - \frac{1}{x^\alpha} \right|^{1-\kappa} \lesssim \llbracket f \rrbracket_{\cC^{1-\kappa}_{d_\alpha}} x^{-\alpha(1-\kappa)}.\]
    This means that $f$ as a limit as $x\to +\infty$, call it $\ell$ and one also has
    \[ |f(x) - \ell| \le \llbracket f \rrbracket_{\cC^{1-\kappa}_{d_\alpha}} x^{-\alpha(1-\kappa)}.\]

    Let us define $\bar{g} : \RR \to \RR$ as follow :
    \[\bar g(x) = 
        \begin{cases}
            g(|x|) & \text{for } x\neq 0 \\
            \ell & \text{for } x=0
        \end{cases}.
    \] 
    Using the previous computation, we have $\bar{g} \in \cC^{1-\kappa}((\RR,|\cdot|);(\RR,|\cdot|))$, $\llbracket \bar{g} \rrbracket_{\cC^{1-\kappa}_{|\cdot|}(\RR;\RR)} \le \llbracket f \rrbracket_{\cC^{1-\kappa}_{d_\alpha}}$ and for $x\in \RR_+^*$, $f(x) = \bar{g}(x^{-\alpha})$.

    The following of the proof is then quite classical. Take $\delta>0$, $\rho_\delta : x \to \rho_\delta(x)= \frac{e^{-\frac{x^2}{2\delta^2}}}{\sqrt{2\pi \delta^2}}$ be the Gaussian density of variance $\delta^2$ and define 
    \[\bar{g}_\delta(x) = \rho_\delta * \bar{g}(x) = \int_{\RR} \rho_\delta(x-y) \bar{g}(y) \dd y.\]
    Notice that thanks to the Hölder continuity of $\bar{g}$, for all $x\in \RR_+^*$, 
    \begin{align*}
        \left|\bar{g}_\delta(x^{-\alpha}) - f(x) \right|= &\left|\bar{g}_\delta(x^{-\alpha}) - \bar{g}(x^{-\alpha}) \right|\\
        \le & \int_{\RR} \rho_\delta(y) \left|\bar{g}(x^{-\alpha}-y) - \bar{g}(x^{-\alpha}) \right| \dd y \\
        \le & \llbracket f \rrbracket_{\cC^{1-\kappa}_{d_\alpha}} \int_{\RR} \rho_\delta(y) y^{1-\kappa} \dd y \\
        \le & \delta^{1-\kappa} \llbracket f \rrbracket_{\cC^{1-\kappa}_{d_\alpha}}.
    \end{align*}
    Last but not least one can use Paley-Littlewood decomposition in order to characterize H\"older norm, see \cite[Chapter 2]{bahouriFourierAnalysisNonlinear2011} and in particular \cite[Theorem 2.36]{bahouriFourierAnalysisNonlinear2011}, on one has 
    \[\llbracket \bar{g}_\delta\rrbracket_{Lip((\RR,|\cdot|);(\RR,|\cdot|))} \lesssim \sup_{j\ge -1} 2^{j}\|\Delta_j \bar{g}_\delta\|_{L^\infty(\RR)} \quad \text{and}\quad \sup_{j\ge -1} 2^{j(1-\kappa)}\|\Delta_j \bar{g}\|_{L^\infty(\RR)} \lesssim \llbracket \bar{g}_\delta\rrbracket_{\cC^{1-\kappa}((\RR,|\cdot|);(\RR,|\cdot|))} \lesssim \llbracket f\rrbracket_{\cC^{1-\kappa}_{d_\alpha}}\]
    Take $\varphi$ being smooth, non negative, with support in an annulus such that for $j\ge 1$, $\Delta_j \bar{g}_\delta = \cF^{-1}(\varphi(2^{-j}\cdot \cF(\bar{g}_\delta))$, and take $\psi$, smooth, non-negative, with support in an annulus such that $\psi_{|\mathop{supp}(\varphi)} = 1$.
    The following computation holds : 
    \begin{align*}
        |\Delta_j \bar{g}_\delta(x)| = & |\cF^{-1}(\psi(2^{-j}\cdot)\cF(\bar{g}) \varphi(2^{-j}\cdot) \cF(\rho_\delta))| \\
        = & \left|(\Delta_j \bar{g})*\left( \cF^{-1}\left(\psi(2^{-j} \cdot) e^{-\frac{\delta^2 |\cdot|^2}{2}}\right)\right)(x)\right|\\
        \le & \|\Delta_j\bar{g}\|_{L^\infty(\RR)} \left\| \cF^{-1}\left(\psi(2^{-j} \cdot) e^{-\frac{\delta^2 |\cdot|^2}{2}}\right)\right\|_{L^1(\RR)}.
    \end{align*}
    Note also that thanks to a change of variables,  
    \[\left\| \cF^{-1}\left(\psi(2^{-j} \cdot) e^{-\frac{\delta^2 |\cdot|^2}{2}}\right)\right\|_{L^1(\RR)} = \left\| \cF^{-1}\left(\psi( \cdot) e^{-\frac{ 2^{2j}\delta^2 |\cdot|^2}{2}}\right)\right\|_{L^1(\RR)}.\]
    We also have 
    \begin{align*}
        \left|(1+|x|^2)\cF^{-1}\left(\psi( \cdot) e^{-\frac{2^{2j}\delta^2 |\cdot|^2}{2}}\right)(x)\right|
        = & \left|  \int_{\RR} \psi(\xi) \left(1-\frac{\partial^2 }{\partial \xi^2}\right)e^{i\xi x} e^{-\frac{2^{2j}\delta^2 \xi^2}{2}} \dd \xi\right| \\
        =& \left|  \int_{\RR} e^{i\xi x} \left(1-\frac{\partial^2}{\partial\xi^2}\right) \left(\psi(\cdot) e^{-\frac{2^{2j}\delta^2 |\cdot|^2}{2}} \right)(\xi) \dd \xi\right| \\
        \lesssim & (1+\delta 2^j+(\delta 2^j)^2) e^{-c\delta^2 2^{2j}} \\
        \lesssim & 2^{-j\kappa} \delta^{-\kappa}.
    \end{align*}
    This means that 
    \[\left\| \cF^{-1}\left(\psi(2^{-j} \cdot) e^{-\frac{\delta^2 |\cdot|^2}{2}}\right)\right\|_{L^1(\RR)} \lesssim 2^{-\kappa j} \delta^{-\kappa}\]
    Using the fact that $\|\Delta_j \bar{g}\|_{L^\infty(\RR)} \lesssim 2^{-(1-\kappa )j} \llbracket f \rrbracket_{\cC^{1-\kappa}_{d_\alpha}}$, one gets that 
    \[\sup_j 2^{j}\|\Delta_j \bar{g}_\delta\|_{L^\infty} \lesssim \delta^{-\kappa} \llbracket f \rrbracket_{\cC^{1-\kappa}_{d_\alpha}}.\]
    By defining $f_\delta = \bar{g}_\delta(x^{-\alpha})$, we have the wanted result. 
\end{proof}

% \begin{remark}\label{rem:equiv-sup}
%     Using the same techniques of proof as before, for a given $f\in Lip_{d_\alpha}$ with $f(0)  = 1$, for any $\delta>0$, there exists a function $g_\delta' : \RR_+^* \to \RR$ such that if we define 
%     \[f_\delta(x) = \int_1^{x^-\alpha} g'_\delta(y) \dd y, \]
%     one has
%     \[\sup_{x\in\RR_+^*} |f(x)-f_\delta(x)| \lesssim \delta \quad \text{and} \quad \llbracket f_\delta \rrbracket_{Lip_{d_\alpha}} \le \llbracket  f \rrbracket_{Lip_{d_\alpha}}.\]
% \end{remark}

\subsection{Topology on space of measures}

\subsubsection{Signed measure on metric space}

We refer to {\cite{bogachevMeasureTheory2007}}, Chapter 3, 7 and 8 and
{\cite{follandRealAnalysisModern2007}} Chapter 7 for complete details on
finite signed measure metric spaces.

\begin{definition}
  Let $(E, d_E)$ be a non empty polish space. A Signed measure on $E$ is a
  $\sigma$-additive functional $\mu$ from $\cB (E)$ the set of Borel
  sets on $E$ in $\RR$ such that $\mu (\emptyset) = 0$ and $| \mu (A) |
  < + \infty$ for all $A \in \cB (E)$. We denote this space by
  $\cM (E, d_E)$.
\end{definition}

\begin{proposition}[Hahn-Jordan decomposition]\label{prop:hahn-jordan}
  Let $\mu \in \cM (E, d_E)$, there exists two non-negative finite
  measure $\mu_+, \mu_-$ such that $ \text{ supp } \mu_+ \cap  \text{ supp } \mu_- =
  \emptyset$ and $ \text{ supp } \mu_+ \cup  \text{ supp } \mu_- = E$, and
  furthermore,
  \[ \mu = \mu_+ - \mu_- . \]
\end{proposition}

\begin{definition}
  Let $\mu \in \cM (E, d_E)$ and define the total variation of $\mu$
  as
  \[ | \mu | = \mu_+ + \mu_- \]
  and the total variation norm as
  \[ \| \mu \| = (\mu_+ + \mu_-) (E) . \]
\end{definition}

Finally le us remind the Riesz-Markov-Kakutani representation theorem
$\cM (E, d_E)$.

\begin{theorem}\label{theorem:M-as-topologocal-dual}
  Let $(E, d_E)$ be a non empty polish space. Let $C_0 (E ; \RR)$ be
  the set of all continuous function $f$ vanishing at infinity, namely for all
  $\eps > 0$ there exists a compact set $K \subset E$ such that $| f
  (x) | < \eps$ for $x \notin K$. We endowed $C_0 (E ; \RR)$
  with the sup-norm topology, which makes it a complete vector space and
  denotes by $C_0 (E ; \RR)^{\ast}$ its topological dual. Let $\mu \in
  \cM (E, d_E)$ and define $T_{\mu} \in C_0 (E ; \RR)^{\ast}$
  via
  \[ T_{\mu} f = \int_E f (x) \mu (\dd x) . \]
  Then $\mu \rightarrow T_{\mu}$ is a isometric isomorphism from $(\cM
  (E, d_E), \| \cdot \|)$ to $C_0 (E ; \RR)^{\ast}$. In particular
  $(\cM (E, d_E), \| \cdot \|)$ is a complete vector space and
  \[ \| \mu \| = \underset{\begin{array}{c}
       f \in C_0 (E ; \RR)\\
       \| f \|_{\infty, E} \le 1
     \end{array}}{\sup} | \langle f, \mu \rangle | . \]
\end{theorem}

\subsubsection{Wasserstein space of finite measures on general metric space}

We need to define the Wasserstein-1 topology on the space of \emph{finite}
measure. To do so we rely on {\cite{martiniKolmogorovEquationsSpaces2023}} and
Appendix $B$ of {\cite{claisseMeanFieldGames2023b}}.

\begin{definition}
  Let $\cM_1^+ (E, d_E)$ be the set of finite non-negative measures on
  a non empty polish space $(E, d_E)$ such that there exists $x_0 \in E$ (and
  thus for all $x_0 \in E$) with
  \[ \int_E d_E (x, x_0) \mu (\dd x) < + \infty . \]
\end{definition}

Let $\partial$ be a cimetery point and define $\bar{E} = E \cup \{ \partial
\}$. By defining $d_{\bar{E}} (x, \partial) = d_E (x, x_0) + 1$ we can extend
the definition of $d_{\bar{E}}$ on $\bar{E}$, such that $(\bar{E},
d_{\bar{E}})$ is still a Polish space.

\begin{definition}
  Let $m > 0$,
  \[ \cM_{1, m}^+ (\bar{E}, d_{\bar{E}}) = \{ \mu \in \cM_1^+
     (\bar{E}) ; \mu (X) = m \} \]
  and for $\mu, \nu \in \cM_{1, m}^+ (\bar{E}, d_{\bar{E}})$
  \[ W_{1, m} (\mu, \nu) \assign \inf_{\pi \in \Pi (\mu, \nu)} \int_{\bar{X}
     \times \bar{X}} d_{\bar{E}} (x, y) \pi (\dd x, \dd y), \]
  where $\Pi (\nu, \mu)$ denotes the collection of all non-negative measures
  on $\bar{E} \times \bar{E}$ with marginals $\mu$ and $\nu$.
\end{definition}

Let $\mu, \nu \in \cM_1^+ (E, d_E)$ and $m \ge \mu (E) \vee \nu
(E)$, and define
\begin{equation}
  \bar{\mu}_m = \mu (\cdot \cap E) + (m - \mu (E)) \delta_{\partial}
  (\cdot), \quad \bar{\nu}_m = \nu (\cdot \cap E) + (m - \nu (E))
  \delta_{\partial} (\cdot) \label{eq:defmum}
\end{equation}
and
\[ W_1 (\mu, \nu) = W_{1, m} (\bar{\mu}_m, \bar{\nu}_m) . \]
We have the following Kantorovitch-Rubinstein duality :

\begin{lemma}
  \label{lemma:monge-kanto}
  For all $\mu, \nu \in \cM_1^+ (E, d_E)$,
  \[ W_1 (\mu , \nu) = \left( \underset{\substack{
       \varphi \in  \text{ Lip } ((E, d_E) ; (\RR, | \cdot |))\\
       \varphi (x_0) = 0
}}{\sup} \langle \varphi, \mu - \nu \rangle \right) + | \langle
     1, \mu - \nu \rangle |, \]
  thus $W_1$ does not depend of the choice of $m$.
\end{lemma}

\begin{lemma}
  \label{lemma:CV-Wasserstein-linear}The space $(\cM_1^+ (E, d_E),
  W_1)$ is a Polish space and a sequence $(\mu_n)_{n \in \NN}$ in
  $\cM_1 (E, d_E)$ converges to $\mu \in \cM_1^+ (E, d_E)$ if
  and only if for all $\varphi \in C ((E, d_E) ; \RR)$ such that
  there exists $C > 0$ with $| \varphi (x) | \le C (1 + d_E (x, x_0))$,
  we have
  \[ \langle \varphi, \mu_n \rangle \rightarrow \langle \varphi, \mu \rangle .
  \]
\end{lemma}

\subsubsection{Wasserstein space of measures on $(\RR_+^{\ast},
d_{\alpha})$}

Since we are interested to the metric space $E =\RR_+^{\ast}$ with $d_E
= d_{\alpha}$, we will specify some fact on $\cM_1^+
(\RR_+^{\ast}, d_{\alpha})$.

\begin{proposition}
  \label{prop:equivalence-M1}Let $\mu \in \cM_1^+
  (\RR_+^{\ast}, d_{\alpha})$, if and only if $\mu$ is a non-negative
  measure on $\RR_+^{\ast}$ such that
  \[ \langle 1, \mu \rangle < + \infty \quad  \text{ and } \quad \langle x^{-
     \alpha}, \mu \rangle < + \infty . \]
\end{proposition}

\begin{proof}
  Suppose first that $\mu \in \cM_1^+ (R^{\ast}_+, d_{\alpha})$, $\mu$
  is non-negative and $\langle 1, \mu \rangle < + \infty$ and $\langle
  d_{\alpha} ( \cdot, 1), \mu \rangle < + \infty$. But
  \[ \int_{\RR^{\ast}_+} \frac{1}{x^{\alpha}} \mu (\dd x) \le
     \int_{\RR^{\ast}_+} \left( \left| \frac{1}{x^{\alpha}} - 1 \right|
     + 1 \right) \mu (\dd x) = \langle d_{\alpha} (\cdot, 1),
     \mu \rangle + \langle 1, \mu \rangle < + \infty . \]
  On the other side, if $\mu$ is obviously a finite non-negative measure.
  Furthermore,
  \[ \langle d_{\alpha} ( \cdot, 1), \mu \rangle \le \langle
     1, \mu \rangle + \langle x^{- \alpha}, \mu \rangle, \]
  which ends the proof.
\end{proof}

As mentioned before, the introduction of the distance $d_\alpha$ is strongly inspired by \cite{cepedaSmoluchowskisEquationRate2011a}. Let us give a bit more context. In \cite[Definition 2.2]{cepedaSmoluchowskisEquationRate2011a}, a distance between two finite non negative measures is defined as follow : for $\mu,\nu \in \cM^+_{1}$, 
\[d(\mu,\nu) = \int_{0}^{+\infty} x^{-(\alpha + 1)} \left| \mu((0,x])- \nu((0,x])\right|\dd x.\]
In fact, this distance is nearly the Wasserstein distance $W_1$. we have the following fact :
\begin{proposition}
    Let $\mu,\nu \in \cM^+_1$, then
    \[
    W_1(\mu, \nu)= \alpha d(\mu,\nu) + |\langle1, \mu-\nu \rangle|.
    \]
\end{proposition}

\begin{proof}
First, let us define for all $x\in\RR_+^*$,
\[g'(x) = sign\left(\mu\left(\left(0,x^{-\frac{1}{\alpha}}\right]\right) - \nu\left(\left(0,x^{-\frac{1}{\alpha}}\right]\right)\right),\]
and
\[g(x) = \int_1^x g'(y) \dd y \quad \text{and} \quad f(x) = g(x^{-\alpha}).\]
Obviously $g \in Lip_{|\cdot|}$ with $g(1) = 0$ and $\llbracket g \rrbracket_{Lip_{|\cdot|}}   \le 1$.
Furthermore, 
\begin{align*}
    \langle f , \mu-\nu\rangle = & \int_0^{+\infty} \int_0^{x^{-\alpha}} g'(y) \dd y \big(\mu(\dd x)- \nu(\dd x) \big) \\
    = & \int_0^{+\infty} g'(y) \int_0^{y^{-\frac{1}{\alpha}}}  \big(\mu(\dd x)-\nu(\dd x) \big) \dd y \\
    = & \alpha\int_0^{+\infty} y^{-(1+\alpha)} g'(y^{-\alpha}) \int_0^{y} \big(\mu(\dd x)-\nu(\dd x) \big) \dd y \\
    = & \alpha\int_0^{+\infty} y^{-(1+\alpha)} \big|\mu((0,y])-\nu((0,y]) \big| \dd y \\
    = & \alpha d(\mu,\nu).
\end{align*}
Thanks to Proposition \ref{prop:lip-isometry}, 
\[\sup_{\substack{\llbracket h\rrbracket_{Lip_{d_\alpha}} = 1 \\  h(1) = 0 }} \langle h, \mu - \nu \rangle \le \alpha d(\mu,\nu).\]
Furthermore, since $f\in Lip_{d_\alpha}$ and $f(1) = 0$, 
\[\alpha d(\mu,\nu) = \langle f \mu - \nu \rangle \le \sup_{\substack{\llbracket h\rrbracket_{Lip_{d_\alpha}} = 1 \\  h(1) = 0 }} \langle h, \mu - \nu \rangle.\]
This proves the wanted result thanks to Lemma \ref{lemma:monge-kanto}.
\end{proof}

Due to our hypothesis, we will have a lack of Lipschitz continuity, but rather Hölder continuity. The following proposition explain how this fact interact with the $W_1$.

\begin{proposition}
    Let $\mu,\nu \in \cM_1^+$ and $K>0$ such that $W_1(\mu,\nu) \le K$. Let $0<\kappa<1$ and suppose that $g\in \cC^{1-\kappa}((\RR_+^*,d_\alpha);\RR)$. Then
    \[|\langle f,\mu-\nu\rangle| \lesssim  W_1(\mu,\nu)^{1-\kappa}\left(\llbracket f\rrbracket_{\cC^{1-\kappa}_{d_\alpha}}\langle 1 , \mu + \nu \rangle^{\kappa}+|f(1)|W_1(\mu,\nu)^\kappa\right).\]
\end{proposition}

\begin{proof}
    Remark that by considering $g=sgn(f(1)) f - |f(1)|$  one gets
    \[|\langle f,\mu-\nu\rangle| \le |\langle g,\mu-\nu\rangle| + |f(1)| |\langle 1,\mu-\nu\rangle| \le |\langle g,\mu-\nu\rangle| + |f(1)| ,W_1(\mu,\nu)\]
    and we can suppose that $f(1)=0$ since $\llbracket f \rrbracket_{\cC^{1-\kappa}} = \llbracket g \rrbracket_{\cC^{1-\kappa}}$.
    
    Using Lemma \ref{lemma:holder-smooth-approx}, for $\delta>0$, there exists $f_\delta \in Lip_{d_\alpha}$ such that $\llbracket f_\delta\rrbracket_{\cC^{1-\kappa}_{d_\alpha}}\lesssim \llbracket f\rrbracket_{\cC^{1-\kappa}_{d_\alpha}}$, $\llbracket f_\delta\rrbracket_{Lip} \lesssim \delta^{-\kappa} \llbracket f\rrbracket_{\cC^{1-\kappa}_{d_\alpha}}$ and $\sup_{x\in\RR_+^*} |f_\delta(x)-f(x)| \lesssim \llbracket f\rrbracket_{\cC^{1-\kappa}_{d_\alpha}} \delta^{1-\kappa}$.

    We have in this setting
    \begin{align*}
        |\langle f,\mu-\nu\rangle| \le & 
            |\langle f-f_\delta,\mu-\nu\rangle| + |\langle f_\delta,\mu-\nu\rangle| \\
            \lesssim & \llbracket f\rrbracket_{\cC^{1-\kappa}_{d_\alpha}}  \langle 1, \mu + \nu \rangle \delta^{1-\kappa} + \llbracket f\rrbracket_{\cC^{1-\kappa}_{d_\alpha}} \delta^{-\kappa} W_1(\mu,\nu).
    \end{align*}
    One may chose $\delta = \frac{W_{1}(\mu,\nu)}{\langle 1, \mu + \nu \rangle}$, and one gets
    \[|\langle f,\mu-\nu\rangle| \lesssim \llbracket f\rrbracket_{\cC^{1-\kappa}_{d_\alpha}} W_1(\mu,\nu)^{1-\kappa} \langle 1, \mu + \nu \rangle^\kappa,\]
    which is the wanted result. 
\end{proof}

\subsubsection{Compact sets for Wasserstein and total variation topologies}

In the following we will strongly need compacity in $\cM_1^+
(\RR^{\ast}_+, d_{\alpha})$. We follow
{\cite{cardaliaguetNotesMeanField}}, Lemma 5.7 and
{\cite{claisseMeanFieldGames2023b}} Lemma B.3. to prove the following compacity criteria :

\begin{lemma}
  \label{lemma:compacity-wasserstein}Let $p > 1$. If $\cK \subset
  \cM_1^+ (\RR_+^{\ast}, d_{\alpha})$ satisfies
  \begin{equation}
    \underset{\mu \in \cK}{\sup} \left( \int_{\RR_+^*} \left( 1 +
    \frac{1}{x^{\alpha p}} + x \right) \mu (\dd x) \right) < + \infty,
    \label{eq:compacity-condition}
  \end{equation}
  then $\cK$ is relatively compact in $\cM_1^+
  (\RR_+^{\ast})$ and for any $\nu\in\overline{\cK}^{W_1}$ the closure of $\cK$,
  \[\left( \int_{\RR_+^*} \left( 1 +
    \frac{1}{x^{\alpha p}} + x \right) \nu (\dd x) \right) < + \infty.\]
\end{lemma}

\begin{proof}
  Take $m \ge 0$ large enough such that for all $\mu \in \cK$,
  $\mu (\RR_+^{\ast}) \le m$ (which exists thanks to the $1$ in
  Equation \eqref{eq:compacity-condition}, hence $\cK \subset
  \cM_{1, m}^+ (\overline{\RR_+^{\ast}})$, where as previously
  for any element $\mu \in \cK$ we extend it by $\bar{\mu}_m \in
  \cM_{1, m}^+ (\overline{\RR_+^{\ast}})$ as defined in
  Equation \eqref{eq:defmum}.
  
  Furthermore, in that setting, take $R > 1$ and we have
  
  \begin{align*}
    \mu ([R^{- 1}, R]^c) = & \bar{\mu}_m ([R^{- 1}, R]^c)\\
    = & \int_0^{R^{- 1}} 1 \mu (\dd x) + \int_R^{+ \infty} 1 \mu (\dd
    x)\\
    = & \int_0^{R^{- 1}} \frac{R^{\alpha p}}{R^{\alpha p}} \mu (\dd x) +
    \int_R^{+ \infty} \frac{R}{R} \mu (\dd x)\\
    \le & \frac{1}{R^{\alpha p}} \int_0^{R^{- 1}} x^{- \alpha p} \mu
    (\dd x) + \frac{1}{R} \int_R^{+ \infty} x \mu (\dd x)\\
    \lesssim & \left( \frac{1}{R^{\alpha p}} + \frac{1}{R} \right)
    \underset{\mu \in \cK}{\sup} \left( \int_{\RR_+^*} (1 + x^{- \alpha p} +
    x) \mu (\dd x) \right) .
  \end{align*}
  
  Hence, for all $\eps > 0$, a constant $R_0$ exists such that for all
  $R > R_0$,
  \[ \underset{\mu \in \cK}{\sup} ([R^{- 1}, R]^c) < \eps \]
  and $\cK$ is tight thanks to Proposition \ref{prop:compact-dalpha}.
  Thanks to Prohorov theorem, for any sequence $(\mu_n)_n$ of $\cK$, a
  measure $\mu \in \cM_1^+ (\RR_+^{\ast})$ and a subsequence
  (also denoted by $(\mu_n)_n$) exist such that for all bounded continuous
  function $\varphi$,
  \[ \langle \varphi, \mu_n \rangle \rightarrow \langle \varphi, \mu \rangle .
  \]
  Furthermore, we also have
  \[ \int_{\RR_+^*} \left( 1 + \frac{1}{x^{\alpha p}} + x \right) \mu
     (\dd x) \le K < + \infty, 
    \]
     which is the last point of the lemma. 
  Indeed, take
  \begin{equation} \label{eq:definition-phidelta}
  \varphi_{\delta} (x) = \frac{2}{\delta} \left( x - \frac{\delta}{2}
     \right) \mathbbm{1}_{\left[ \frac{\delta}{2}, \delta \right]} (x) +
     \mathbbm{1}_{[\delta, \delta^{- 1}]} (x) + \delta \left( \frac{2}{\delta}
     - x \right) \mathbbm{1}_{[\delta^{- 1}, 2 \delta^{- 1}]} (x).
    \end{equation}
  Remark that for all $x \in \RR_+^{\ast}$, $(\varphi_{\delta}
  (x))_{\delta}$ is non decreasing. Hence, thanks to monotone convergence
  theorem,
  \[ \int_{\RR^{\ast}_+} \varphi_{\delta} (x) \left( 1 +
     \frac{1}{x^{\alpha p}} + x \right) \mu (\dd x) \to_{\delta
     \rightarrow 0} \int_{\RR^{\ast}_+} \left( 1 + \frac{1}{x^{\alpha
     p}} + x \right) \mu (\dd x) . \]
  Furthermore, for fixed $\delta > 0$, $x \rightarrow \varphi_{\delta} (x)
  \left( 1 + \frac{1}{x^{\alpha p}} + x \right)$ is a continuous bounded
  function from $(\RR_+^{\ast}, d_{\alpha})$ to $\RR_+$. Hence
  \[ \int_{\RR^{\ast}_+} \varphi_{\delta} (x) \left( 1 +
     \frac{1}{x^{\alpha p}} + x \right) \mu_n (\dd x) \to_{n
     \rightarrow + \infty} \int_{\RR^{\ast}_+} \varphi_{\delta} (x)
     \left( 1 + \frac{1}{x^{\alpha p}} + x \right) \mu (\dd x) . \]
  Since $0 \le \varphi_{\delta} (x) \le 1$, we get that for all
  $\delta > 0$ and all $n \ge 0$,
  \[ 0 \le \int_{\RR^{\ast}_+} \varphi_{\delta} (x) \left( 1 +
     \frac{1}{x^{\alpha p}} + x \right) \mu_n (\dd x) \le K, \]
  letting $\delta \rightarrow 0$ we get the claim.
  
  Let $\eps > 0$ and $R$ as before. Let $\mu^R_n = \1_{[R^{-
  1}, R]} \mu_n$ and $\mu^R = \1_{[R^{- 1}, R]} \mu$. Remark first
  that for $\llbracket f \rrbracket_{ \text{ Lip } } \le 1$ with $f (1) =
  0$, one gets
  
  \begin{align*}
    | \langle f, \mu_n^R - \mu_n \rangle | = & \left| \int f (x)
    \1_{[R^{- 1}, R]^c} (x) \mu_n (\dd x) \right|\\
    \le & \left| \int d_{\alpha} (x, 1) \1_{[R^{- 1}, R]^c} (x)
    \mu_n (\dd x) \right|\\
    \le & \left| \int d_{\alpha} (x, 1)^p \mu_n (\dd x)
    \right|^{\frac{1}{p}} \left| \int \1_{[R^{- 1}, R]^c} (x) \mu_n
    (\dd x) \right|^{\frac{1}{q}},
  \end{align*}
  
  where $\frac{1}{p} + \frac{1}{q} = 1$.
  \[ | \langle 1, \mu_n^R - \mu_n \rangle | = \left| \int \1_{[R^{-
     1}, R]^c} (x) \mu_n (\dd x) \right| \]
  putting all together and using the tightness, one gets that there exists
  $R_0 > 0$, such that for all $n \ge 0$,
  \[ W_1 (\mu_n^R, \mu_n) \le \eps . \]
  and the same holds for $\mu^R$ and $\mu$.
  
  Finally, take a continuous function $f : \RR_+^{\ast} \rightarrow
  \RR$ such that $| f (x) | \lesssim 1 + d_{\alpha} (x, 1)$, and take
  $f_R$ a continuous and bounded function with support in $[(2 R)^{- 1}, 2 R]$
  such that $f_R = f$ on $[R^{- 1}, R]$. We have
  
  \begin{align*}
    \langle f, \mu_n^R - \mu^R \rangle = & \langle f_R, \mu_n - \mu \rangle
    \to_{n \rightarrow + \infty} 0.
  \end{align*}
  
  By Lemma \ref{lemma:CV-Wasserstein-linear}, $\mu_n^R \rightarrow \mu^R$ for
  $W_1$. Take $R$ as before, and take $n_0 > 0$ such that $W_1 (\mu_n^R,
  \mu^R) < \eps$ for $n \ge n_0$. We get
  \[ W_1 (\mu_n, \mu) \le W_1 (\mu_n^R, \mu_n) + W_1 (\mu_n^R, \mu^R) +
     W_1 (\mu^R, \mu) \le 3 \eps \]
  and the result is proved.
\end{proof}

Thanks to the previous result, it is natural to define the following sets :

\begin{definition}
  Let $p > 1$ and $\alpha > 0$ and $K > 0$ let us define
  \[ \cK^{+, K}_p \assign \left\{ \mu \in \cM_1^+
     (\RR_+^{\ast}, d_{\alpha}) : \int_{\RR_+} (1 + x^{- \alpha
     p} + x) \mu (\dd x) < K \right\}, \]
     and for $c>0$, 
     \[\cE^{+,K}_c = \assign \left\{ \mu \in \cM^+(\RR_+^{\ast}, d_{\alpha}) :  
      \int_{\RR_+} (x + \exp(c x^{- \alpha
     } ) \mu (\dd x) < K \right\}.\]
\end{definition}

The following Lemma is a direct consequence of Lemma \ref{lemma:compacity-wasserstein}.
\begin{theorem}\label{theorem:compacity-wasserstein}
    Let $p>1$, $c>0$ and $K>0$. The sets $\cK^{+,K}_p$ and $\cE^{+,K}_c$ are compact sets of $(\cM^+_1(\RR_+^*,d_\alpha),W_1)$.
\end{theorem}

\begin{proof}
Lemma \ref{lemma:compacity-wasserstein}, for all $K >
0$, $\cK^{+, K}_p$ is a relatively compact set of $\cM_1^+$, which is also close thanks to the second part of the same Lemma, hence it is compact. Furthermore, remark that for any $\tilde{c}<c$
\[x+1+x^{-2\alpha} \le \left(1+\frac{2}{c^2}\right)\left(x+ e^{c x^{-\alpha}}\right),\]
meaning that $\cE^{+,K}_c \subset \cK^{+,K(1+2c^{-2})}_2$ and thus is also relatively compact. 
One only has to prove that $\cE^{+,K}_c$ is closed to conclude.

This follows easily using the same techniques as in the proof of Lemma \ref{lemma:compacity-wasserstein}. Take $\varphi_{\delta}$ as defined in Equation \ref{eq:definition-phidelta}, take $(\mu_n)_{n\ge 0}$ a sequence of $\cE^{+,K}_c$ which converges to a measure $\mu\in \cM^+$. Remark that for $0<\delta\le 1$
\[x \to \varphi_\delta(x) e^{cx^{-\alpha}} \] 
is continuous from $(\RR_+^*,d_\alpha)$ to $\RR$ and satisfies the conditions of Lemma \ref{lemma:CV-Wasserstein-linear}, hence 
\begin{multline*}
    K\ge \int_0^{+\infty} (x+\exp(cx^{-\alpha}))  \mu_n(\dd x) \ge \int_0^{+\infty} \varphi_{\delta}(x)(x+\exp(cx^{-\alpha}))  \mu_n(\dd x) \\ \underset{n \to +\infty}{\to} \int_0^{+\infty} \varphi_{\delta}(x)(x+\exp(cx^{-\alpha}))  \mu(\dd x) \underset{\delta \to 0}{\to} \int_0^{+\infty} (x+\exp(cx^{-\alpha}))  \mu(\dd x),
\end{multline*}
where the last convergence holds thanks to the monotone convergence theorem.
We've just prove that $\mu \in \cE^{+,K}_c$, which ends the proof.
\end{proof}

In view of the rest of the paper, especially in the setting where we want to use this machinery, it is necessary to define the corresponding spaces for signed measures. 

\begin{definition}
  Let us define for $p \ge 1$ and $c,K > 0$,
  \[ \cM_1 = \{ \mu \in \cM (\RR_+^{\ast}, d_{\alpha})
     : | \mu | \in \cM_1^+ \}, \quad \cK_p^K = \{ \mu \in
     \cM (\RR_+^{\ast}, d_{\alpha}) : | \mu | \in
     \cK_p^{+, K} \}  \]
  and for $c>0$,
  \[ \cE_c^{K} = \{ \mu \in \cM (\RR_+^{\ast},
     d_{\alpha}) : | \mu | \in \cK_c^{+,K} \}. \]
\end{definition}

\begin{proposition}
  For any $p \ge 1$ and $c,K > 0$ the sets $\cK_p^K$ and $\cE^{K}_c$ are closed
  $\| \cdot \|$.
\end{proposition}

\begin{proof}
  Let $(\mu_n)_{n \ge 0}$ be a sequence of elements of $\cK_p^K$ (respectively $\cE^K_c$)
  which converge toward $\mu \in \cM$ for the total variation norm $\|
  \cdot \|$. Take $\varphi_{\delta}$ as in Equation \eqref{eq:definition-phidelta}, then for all $\delta > 0$ the functions
  \[ x \rightarrow \varphi_{\delta} (x) (x + x^{- \alpha p} + 1) \quad \text{and} \quad  x \rightarrow \varphi_{\delta} (x) (x + exp^{cx^{-\alpha}})\]
  belong to $ C_0
     ((\RR_+^{\ast}, d_{\alpha}) ; \RR_+) $.
  Hence,
  \[ \int_{\RR^{\ast}_+} \varphi_{\delta} (x) (x + x^{- \alpha p} + 1)
     \mu_n (\dd x) \to_{n \rightarrow + \infty}
     \int_{\RR^{\ast}_+} \varphi_{\delta} (x) (x + x^{- \alpha p} + 1)
     \mu (\dd x), \]
 and the same holds when replacing $(x+x^{-\alpha p} +1)$ by $(x + \exp(cx^{-\alpha}))$.
 
  Remark also that for any measurable $A \subset \RR_+^{\ast}$,
  \[ | \mu_n | (A) \le | \mu_n - \mu | (A) + | \mu | (A) \quad
      \text{ and } \quad | \mu | (A) \le | \mu_n - \mu | (A) + | \mu_n |
     (A) \]
  hence,
  \[ \big| | \mu_n | (A) - | \mu | (A) \big| \le | \mu_n - \mu | (A) \le
     \| \mu_n - \mu \| . \]
  Taking the supremum on $A$ and using {\cite{bogachevMeasureTheory2007}},
  (3.1.4),
  \[ \| | \mu_n | - | \mu | \| \le 2 \| \mu_n - \mu \| . \]
  and since $\varphi_{\delta} \le 1 $,
  \[ K \ge \left| \int_{\RR^{\ast}_+} \varphi_{\delta} (x) (x +
     x^{- \alpha p} + 1) | \mu_n | (\dd x) \right| \to_{n
     \rightarrow + \infty} \int_{\RR^{\ast}_+} \varphi_{\delta} (x)
     (x + x^{- \alpha p} + 1) | \mu | (\dd x) \]
  and by letting $\delta \rightarrow 0$, thanks to the monotone theorem, $\mu \in \cK_p^K$. The same techniques holds for $x+e^{cx^{-\alpha}}$ instead of $(x+x^{-\alpha p} +1)$ , which ends
  the proof.
\end{proof}

\subsubsection{Compacity for signed measures}

Finally, we will need some compacity in the space of signed measure. We rely here on the weak-* topology generated by the space of Lipschitz continuous function. Subsequently, we also give an alternative formulation of the Wasserstein distance. Let us endow the space $Lip\big((\RR_+^*,d_\alpha);(\RR,|\cdot|)\big)$ with the norm 
\[\|f\|_{Lip_{d_\alpha}} = \llbracket f \rrbracket_{Lip((\RR,d_\alpha);(\RR,|\cdot|))} \vee |f(1)|.\]

Let us denote by 
\[Lip^*_{d_\alpha} = Lip((\RR_+^*,d_\alpha);\RR)^*\]
the topological dual of the space of Lipschitz continuous functions, with  norm
\[ \|T\|_{Lip^*_{d_\alpha}} = \sup_{\|f\|_{Lip_{d_\alpha}}\le 1} T(f). \]

\begin{proposition}\label{proposition:dual}
The space $\cM_1$ is a subspace of $Lip_{d_\alpha}^*$, when identifying the points $\mu \in \cM_1$ with their corresponding linear form $T_\mu$ (as define in Theorem \ref{theorem:M-as-topologocal-dual}), and given for any $f\in Lip_{d_\alpha}$ by $T_\mu(f) = \langle f,\mu \rangle$.
\end{proposition}

\begin{proof}
    Let $\mu \in \cM_1^+$. Let us remind that it means that  
    \[\int_{\RR_+^* } (1 + x + x^{-\alpha}) |\mu|(\dd x) < + \infty.\]
    Let $f\in Lip_{d_\alpha}$. Remark that for all $x\in \RR_+^*$, $|f(x)| \le \llbracket f \rrbracket_{Lip_{d_\alpha}} (x^{-\alpha} + 1) + |f(1)|\le 2 \|f\|_{Lip_{d_\alpha}}(x^{-\alpha} + 1)$. This leads to the following inequalities :
    \[|T_mu(f)| \le 2\|f\|_{Lip_{d_\alpha}} \int_{\RR_+^*}( 1 + x^{-\alpha} )\mu(\dd x).\]
    This implies that 
    \[\|T_\mu\|_{Lip_{d_\alpha}} \le 2 \int_{\RR_+^*}( 1+x^{-\alpha})\mu(\dd x)<+\infty,\]
    which is the claim.
\end{proof}

Furthermore, we have the following proposition, which allows to link the Wasserstein distance on positive signed measures and the $\|\cdot\|_{Lip_{d_\alpha}^*}$ norm.

\begin{proposition}
    Let us define for $T\in Lip_{d_\alpha}^*$ the following quantity :
    \[\|T\|_{W_1} = \sup_{\substack{g\in Lip_{d_\alpha}\\g(1)=0}} T(g) + |T(1)|,\]
    where $1$ stands for the constant function equal to $1$. Then $\|\cdot\|_{Lip_{d_\alpha}^*}$ and $\|\cdot\|_{W_1}$ are equivalent norms. 
\end{proposition}

\begin{proof}
Suppose that $T \in Lip_{d_\alpha}^*$. We have 

\begin{align*}
T(f) = & T(f-f(1))+f(1)T(1) \\
\le & \llbracket f \rrbracket_{Lip_{d_\alpha}} \sup_{\substack{\llbracket g\rrbracket_{Lip_{d_\alpha}} \le 1 \\ g(1) = 0}} T(g)  + |f(1)| |T(1)| \\
\le & \|f\|_{Lip_{d_{\alpha}}} \|T\|_{W_1}.
\end{align*}
Taking the supremum for $\|f\|_{Lip_{d_{\alpha}}} \le 1$, we have 
$\|T\|_{Lip^*_{d_\alpha}} \le  \|T\|_{W_1}$.

Remark that $1\in Lip_{d_\alpha}$ and $\|1\|_{Lip_{d_\alpha}} = 1$, meaning that for any $T\in Lip_{d_\alpha}^*$,
$|T(1)| \le \|T\|_{Lip^*_{d_\alpha}}$.
Take $g\in Lip_{d_\alpha}$ with $\llbracket g\rrbracket_{Lip_{d_\alpha}} \le 1$ and $g(0) = 0$. This implies that $\|g\|_{Lip_{d_\alpha}} \le 1$, and we have
\[T(g) + |T(1)| \le 2\|T\|_{Lip^*_{d_\alpha}}.\]
taking the supremum on $g$, we have that $\|\cdot\|_{W_1}\le 2\|\cdot\|_{Lip^*_{d_\alpha}}$, which proves the result. 
\end{proof}

A straightforward but rather important corollary is the following :
\begin{corollary}
    On $\cM^+_1$, $d_{W_1}$ and $(\mu,\nu)\to\|\mu-\nu\|_{Lip_{d_\alpha}^*}$ are equivalent distances.      
\end{corollary}

\begin{proof}
    The proof is a direct consequence of the previous Proposition about the equivalence of norms and the Mong-Kantorovitch duality (Lemma \ref{lemma:monge-kanto}).
\end{proof}

\begin{proposition}
    Les $p>1$ and $c,K >0$. The spaces $\cK^K_p$ and $\cE^{K}_c$ are compact in  $Lip^*_{d_\alpha}$ for the weak-* topology. Namely, for any sequence $(\mu_n)_{n\in \NN}$ of $\cK^K_p$ (respectively $\cE^K_c$), a subsequence (also denoted by $(\mu_n)_{n\in \NN}$) and a $\mu \in \cK^K_p$ (respectively $\mu \in \cC^{K}_c$) exist such that for all $f\in Lip_{d_\alpha}$, 
    \[\langle f,\mu_n \rangle \underset{n\to + \infty}{\longrightarrow} \langle f, \mu \rangle.\]
\end{proposition}

\begin{proof}
    Let us remark that for all $n\in \NN$, $\mu_n = \mu_n^+ - \mu_n^-$, where this decomposition is unique and given by Proposition \ref{prop:hahn-jordan}.
    Note that $\mu_n^+$ and $\mu_n^-$ also belongs to $\cM^{+,K}_{p}$ (respectively $\cE^{+,K}_{c}$. It implies that there is two measures $\mu^+,\mu^- \in \cK^{+,K}_p$ (respectively in $\cE^{+,K}_c$) and subsequences of $(\mu_n^+)_{n\ge 0}$ and $(\mu_n^-)_{n\ge 0}$ (again denoted abusively by $(\mu_n^+)_{n\ge 0}$ and $(\mu_n^-)_{n\ge 0}$) such that 
    \[W_(\mu^+_n,\mu^+) \to 0 \quad \text{and} \quad W_(\mu^-_n,\mu^-)\to 0.\]
    Since $\mathop{supp} \mu_n^+\cap \mathop{supp} \mu^-_n = \emptyset$, so does $\mathop{supp}\mu^+$ and $\mathop{supp}\mu^-$. Let us define $\mu= \mu^+ - \mu^-$. We obviously have $\mu \in \cK^K_p$ (respectively $\cE^K_c$.
    
    Furthermore for any $f\in Lip_{d_\alpha}$, one gets 
    \begin{equation*}
        |\langle f, \mu_n - \mu \rangle | \le |\langle f, \mu_n^+ - \mu^+ \rangle |+|\langle f, \mu_n^- - \mu^- \rangle | 
         \lesssim \|f\|_{Lip_{d_\alpha}} \big( W_1(\mu_n^+\mu^+)+ W_1(\mu_n^-\mu^-) \big).
    \end{equation*}
    This implies that 
    \[\|\mu_n - \mu\|_{Lip^*_{d_\alpha}} \lesssim W_1(\mu_n^+\mu^+)+ W_1(\mu_n^-\mu^-) \to_{n\to +\infty} 0. \]
    This ends the proof when reminding that the convergence for the norm of the dual space implies the weak-* convergence. 
\end{proof}

\begin{remark}
    We actualy proved something stronger, that $\cK^K_c$ and $\cE^K_c$ are compact sets of $Lip_{d\alpha}^*$.
\end{remark}

\red{Question : est-il possible de voir $T$ comme une forme linéaire continue de $C^0_b$ ? Peut-être  : 
Let $\varphi_\delta$ as define in Equation \eqref{eq:definition-phidelta}. 

 Je sais pas, pas forcément. Par exemple, dans le cas des fonction continues sur $\RR$, $T_\eps$ 
 \[T_\eps = \frac1\eps \int_0^\eps f(x) \dd x\]
 est définie pour toute fonction continue. Remarquons également que si $\|f\|_{C^1} = \|f\|_{\infty}+\|f'\|_{\infty}$, alors $T \in (C^1_b)^*$. Remarque also that  
 \[T_\eps \to \]
}

{\color{blue} Pas forcément pertinent}

Finally, the last result of this section combines Arzela Ascoli theorem and the previouses compacity criterion.

\begin{proposition}
    Let $(E,d)$ be a metric space, $0<\kappa\le1$ let $\cK \subset C\big((E,d); (\cM_1,\|\cdot\|_{Lip^*_{d_\alpha}})\big)$. Suppose that a constant $K>0$ and a constant $p>1$ exists (respectively $c>0$ exists) such that
    \begin{itemize}
        \item For all $f\in \cK$ and all $x\in E$, $f(x) \in \cK^{K}_p$ (respectively $\cE^K_c$).
        \item For all $f\in \cK$, 
        \[\sup_{x\neq y \in E} \frac{\|f(x)-f(y)\|_{Lip^*_{d_\alpha}}}{d(x,y)^\kappa} \le K.\]
    \end{itemize}
    Then $\cK$ is relatively compact on $C\big((E,d); (\cM_1,\|\cdot\|_{Lip^*_{d_\alpha}})\big)$ endowed with the topology of uniform convergence on all compact sets. 
\end{proposition}

\begin{proof}
    The proof is just a consequence of the compacity 
\end{proof}

\subsection{Linear derivatives in the Wasserstein space of non-negative finite
measures}

\

\begin{definition}
  Let $(E, d_E)$ be a Polish space and $F : (\cM^+_1 (E, d_E), W_1)
  \rightarrow \RR$ be a continuous function. We say that $F$ admit a
  flat derivative, also known as linear functional derivative if there exists
  a function
  \[ \frac{\delta F}{\delta m} \in C (\cM_1^+ \times E ; \RR)
  \]
  sucht that
  \begin{enumerate}
      \item For all $\nu,\mu\in \cM_1^+$, $z\to \sup_{\lambda \in [0,1]} \left|\frac{\delta F}{\delta m} (\lambda (\mu- \nu) + \nu;z)\right| \in L^1(\mu)\cap L^1(\nu)$,
      \item For all $\nu,\mu\in \cM_1^+$
      \[ F (\mu) - F (\nu) = \int_0^1 \int_E \frac{\delta F}{\delta m} (\ell (\mu
     - \nu) + \nu ;  z) (\mu - \nu) (\dd z) \dd
     \ell . \]
  \end{enumerate}
  We write $F \in C^1 (\cM_1^+ ; \RR)$.  
\end{definition}

\begin{remark}
  The justification of the previous Definition is as follows. If $(E,d)$ is as previously, even though  $Lip
  \big(E,d); \RR\big)$ endowed with the norm $\|f\|_{Lip}  = \llbracket f \rrbracket_{Lip}\vee |f(x_0)|$  is  not a relfexive space, we would like to think that
  \[ \big(Lip\big((E,d); \RR\big)^{\ast}\big)^{\ast} =(\cM_1)^{\ast} \approx  Lip\big((E,d); \RR\big). \]
  In this setting, if $F$ is Fr{\'e}chet differentiable, then $D F (\mu) \in
  \cM_1^{\ast}$ and one would like to have a ``gradient'', namely to
  have a function, $\frac{\delta F}{\delta m} (\mu) \in Lip \big((E,d); \RR)$
  such that for all $\mu, \nu \in \cM_+$
  \[ F (\mu) - F (\nu) = D F (\mu) \cdot (\mu - \nu) + o (\| \mu - \nu \|) =
     \left\langle \frac{\delta F}{\delta m} (\mu), (\mu - \nu) \right\rangle +
     o (\| \mu - \nu \|) . \]
  We cannot hope that such a gradient exists, and even more that it is regular. Nevertheless, the previous definition of flat derivative to such functions.   
\end{remark}

It appears that
  whenever $\frac{\delta F}{\delta m}$ exists, it is possible to see it as a
  directional derivative, as the following proposition shows.

\begin{proposition}
  Suppose that $F \in C^1 (\cM_1^+ ; \RR)$, then
  \[ \frac{\delta F}{\delta m} (\mu; z) = \lim_{\eps\to 0} \frac{F (\mu + \eps
     \delta_z) - F (\mu)}{\eps} . \]
  \begin{proof}
    The proof is an obvious consequence of the definition. Remark first that
    
    \begin{align*}
      \frac{F (\mu + \eps \delta_z) - F (\mu)}{\eps} = &
      \int_0^1 \int_E \frac{\delta F}{\delta m} (\eps \ell \delta_z +
      \mu ; \widetilde{z}) \delta_z (\dd \tilde{z}) \dd \ell\\
      = & \int_0^1 \frac{\delta F}{\delta m} (\eps \ell \delta_z + \mu
      ; z) \dd \ell .
    \end{align*}
    
    Furthermore remind that $\frac{\delta F}{\delta m}$ is continuous. It implies that for $z\in E$, $t \in [0,1]\to \frac{\delta F}{\delta m}(t\delta_z + \mu;z) - \frac{\delta F}{\delta m}(\mu;z)$ is continuous in $t=0$ and null in $0$. Let $\delta>0$ and $\kappa>0$ such that for all $t\le \kappa$, 
    \[\left|\frac{\delta F}{\delta m}(t\delta_z + \mu;z) - \frac{\delta F}{\delta m}(\mu;z)\right| \le \delta.\]
    Take $\eps\le\kappa$, then for all $\ell \in [0,1]$, $\ell \eps \le \kappa$ and 
    \[\left|\int_0^1 \frac{\delta F}{\delta m} (\eps \ell \delta_z + \mu
      ; z) - \frac{\delta F}{\delta m} ( \mu
      ; z) \dd \ell \right| \le \delta.\]
      Hence
    \[\left|\frac{F (\mu + \eps \delta_z) - F (\mu)}{\eps} - \frac{\delta F}{\delta m} ( \mu
      ; z)\right| \le \delta,\]
      and the results follows by sending $\eps\to 0$.
  \end{proof}
\end{proposition}


\begin{definition}
  Let $\varphi \in C ((\cM^+_1, W_1) ; (\cM^+_1, W_1))$ be a
  continuous function. We say that $\varphi$ admits a weak flat derivative if there
  exists a function $\frac{\delta \varphi}{\delta m} \in C (\cM^+_1
  \times E ; \cM_1)$ such that for all $f \in C_b (E ; \RR)$,
  $\mu \rightarrow \langle f, \varphi (\mu) \rangle \in C^1 (\cM^+_1 ;
  \RR)$ and we have
  \[ \frac{\delta (\langle f, \varphi \rangle)}{\delta m} (\mu ; z) =
     \left\langle f, \frac{\delta \varphi}{\delta m} (\mu ; z) \right\rangle .
  \]
  We write $\varphi \in C^1 (\cM_1^+ ; \cM_1^+)$.
\end{definition}

\begin{remark}
  In that setting
  \[ \int_E \left\langle f, \frac{\delta \varphi}{\delta m} (\mu ; z)
     \right\rangle \nu (\dd z) = \left\langle f, \int_E \frac{\delta
     \varphi}{\delta m} (\mu ; z) \nu (\dd z) \right\rangle, \]
  which is a direct consequence of the \red{boundedness} of $\frac{\delta
  \varphi}{\delta m}$ in $z$ and Fubini theorem.
  \red{Il y a possiblement un problème ici. En effet, il faudrait définir ce que veux dire 
  \[\int_{E} \frac{\delta \varphi}{ \delta m}(\mu;z)\nu(\dd z).\]
  One can use the notion of Böchner integral \blue{reference needed} (it would be easier in $(\cM,\|\cdot\|_{VT})$. Nevertheless, remind that $W_1(\mu,\nu) \equiv \|\mu - \nu\|_{Lip^*}$, which is a Banach space. Furthermore, since for all $f\in Lip$,
  \[z\to\sup_{\lambda \in [0,1]}\left|\left\langle f , \frac{\delta \varphi}{\delta m}(\lambda (\mu-\nu) + \nu;z)\right\rangle \in L^1(\nu),\right|\]
  in particular 
  \[z\to \left\langle f , \frac{\delta \varphi}{\delta m}(\mu;z)\right\rangle \in L^1(\nu).\]}
\end{remark}

We can know state and prove the following chain rule formula :

\begin{lemma}
  \label{lemma:chain-rule}Suppose that $\varphi \in \varphi \in C^1
  (\cM_1^+ ; \cM_1^+)$ and $F \in C^1 (\cM^+_1 ;
  \RR)$. Suppose furthermore that for all $y \in E$, $\mu \rightarrow
  \frac{\delta F}{\delta m} (\mu ; y)$ and $\mu \rightarrow \frac{\delta
  \varphi}{\delta m} (\mu ; z)$ are uniformely continuous. Then $F \circ
  \varphi \in C^1 (\cM^+_1 ; \RR)$
  \[ \frac{\delta (F \circ \varphi)}{\delta m} (\mu ; z) = \left\langle
     \frac{\delta F}{\delta m} (\varphi (\mu)) ; \frac{\delta \varphi}{\delta
     m} (\mu ; z) \right\rangle . \]
\end{lemma}

\begin{proof}
  Let $n \ge 1$, remark that $F (\varphi (\mu)) - F (\varphi (\nu))$
  
  \begin{align*}
    = & \sum_{k = 0}^{n - 1} F \left( \varphi \left( \frac{k + 1}{n} (\mu -
    \nu) + \nu \right) \right) - F \left( \varphi \left( \frac{k}{n} (\mu -
    \nu) + \nu \right) \right)\\
    = & \sum_{k = 0}^{n - 1} F \left( \int_0^1 \int_E \frac{\delta
    \varphi}{\delta m} \left( \frac{r + k}{n} (\mu - \nu) + \nu ; y \right)
    \frac{(\mu - \nu)}{n} (\dd y) \dd r + \varphi \left( \frac{k}{n}
    (\mu - \nu) + \nu \right) \right)\\
    & \qquad - F \left( \varphi \left( \frac{k}{n} (\mu - \nu) + \nu \right)
    \right)\\
    = & \sum_{k = 0}^{n - 1} \int_0^1 \int_E F_{n, k} (\ell, z) \left(
    \int_0^1 \int_E \frac{\delta \varphi}{\delta m} \left( \frac{r_2 + k}{n}
    (\mu - \nu) + \nu ; y_2 \right) \frac{(\mu - \nu)}{n} (\dd y_2) \dd
    r_2 \right) (\dd z) \dd \ell,
  \end{align*}
  
  where
  
  \begin{multline*}
    F_{n, k} (\ell, z)\\
    = \frac{\delta F}{\delta m} \left( \ell \left( \int_0^1 \int_E
    \frac{\delta \varphi}{\delta m} \left( \frac{r_1 + k}{n} (\mu - \nu) + \nu
    ; y_1 \right) \frac{(\mu - \nu)}{n} (\dd y_1) \dd r_1 \right) +
    \varphi \left( \frac{k}{n} (\mu - \nu) + \nu \right) ; z \right) .
  \end{multline*}
  
  Remark that since $\frac{\delta \varphi}{\delta m}$ is continuous and
  bounded in $y$, and since
  \[ \ell \int_0^1 \frac{\delta \varphi}{\delta m} \left( \frac{r_1 + k}{n}
     (\mu - \nu) + \nu ; y_1 \right) \frac{(\mu - \nu)}{n} (\dd y_1) \dd
     r_1 \to_{n \rightarrow + \infty} 0 \]
  uniformely in $\ell$, Furthermore, since $\left( \varphi \left( \frac{k}{n}
  (\mu - \nu) + \nu \right) \right)_{0 \le k \le n - 1}$ is
  bounded, for all $\delta > 0$, an integer $n_0 \ge 0$ exists such that
  for all $\ell \in [0, 1]$ and all $n \ge n_0$, the $W_1$ distance
  between
  \[ \ell \int_0^1 \frac{\delta \varphi}{\delta m} \left( \frac{r_1 + k}{n}
     (\mu - \nu) + \nu ; y_1 \right) \frac{(\mu - \nu)}{n} (\dd y_1) \dd
     r_1 + \varphi \left( \frac{k}{n} (\mu - \nu) + \nu \right) \]
  and
  \[ \varphi \left( \frac{k}{n} (\mu - \nu) + \nu \right) \]
  is smaller that $\delta$.
  
  Since for all $z \in E$, $\mu \rightarrow \frac{\delta F}{\delta m} (\mu ;
  z)$ is uniformly continuous, for all $\eps > 0$ and all $z \in E$,
  there exists $\delta > 0$ such that whenever $W_1 (\mu, \nu) \le
  \delta$, $\left| \frac{\delta F}{\delta m} (\mu ; z) - \frac{\delta
  F}{\delta m} (\nu ; z) \right| \le \eps$. Hence, for all $n
  \ge n_0$ and all $k \in \{ 0, \cdots, n \}$ and $\ell \in [0, 1]$,
  \[ \left| F_{n, k} (\ell ; z) - \frac{\delta F}{\delta m} \left( \varphi
     \left( \frac{k}{n} (\mu - \nu) + \nu \right) ; z \right) \right|
     \le \eps \]
  Furthermore
  
  \begin{multline*}
    \left| \int_E \left( \int_0^1 \int_E \frac{\delta \varphi}{\delta m}
    \left( \frac{r_2 + k}{n} (\mu - \nu) + \nu ; y_2 \right) \frac{(\mu -
    \nu)}{n} (\dd y_2) \dd r_2 \right) (\dd z) \right|\\
    \le \frac{1}{n} \left( \underset{\begin{array}{c}
      r \in [0, 1]\\
      y \in E
    \end{array}}{\sup} \left| \frac{\delta \varphi}{\delta m} (r \mu + (1 - r)
    \nu ; y) \right| (E) \right) (\mu + \nu) (E) .
  \end{multline*}
  
  Hence, if we define
  
  \begin{multline*}
    G_n (\mu, \nu) = \sum_{k = 0}^{n - 1} \int_E \frac{\delta F}{\delta m}
    \left( \varphi \left( \frac{k}{n} (\mu - \nu) + \nu \right) ; z \right)\\
    \times \left( \int_0^1 \int_E \frac{\delta \varphi}{\delta m} \left(
    \frac{r_2 + k}{n} (\mu - \nu) + \nu ; y_2 \right) \frac{(\mu - \nu)}{n}
    (\dd y_2) \dd r_2 \right) (\dd z)
  \end{multline*}
  
  And for all $n \ge n_0$,
  \[ | F (\varphi (\mu)) - F (\varphi (\nu)) - G_n (\mu , \nu) |
     \le \eps \left( \underset{\begin{array}{c}
       r \in [0, 1]\\
       y \in E
     \end{array}}{\sup} \left| \frac{\delta \varphi}{\delta m} (r \mu + (1 -
     r) \nu ; y) \right| (E) \right) (\mu + \nu) (E) . \]
  Using the same techniques, one can show that there exists $n_1 \ge
  n_0$ such that for all $n \ge n_1$,
  \[ | G_n (\mu , \nu) - H_n (\mu, \nu) | \le \eps
     \underset{\begin{array}{c}
       r \in [0, 1]\\
       y \in E
     \end{array}}{\sup} \left| \frac{\delta F}{\delta m} (r \mu + (1 - r) \nu
     ; z) \right| (\mu + \nu) (E), \]
  where
  
  \begin{multline*}
    H_n (\mu, \nu) = \sum_{k = 0}^{n - 1} \int_E \frac{\delta F}{\delta m}
    \left( \varphi \left( \frac{k}{n} (\mu - \nu) + \nu \right) ; z \right)\\
    \times \left( \int_E \frac{\delta \varphi}{\delta m} \left( \frac{k}{n}
    (\mu - \nu) + \nu ; y_2 \right) \frac{(\mu - \nu)}{n} (\dd y_2) \right)
    (\dd z) .
  \end{multline*}
  
  Since $\frac{\delta \varphi}{\delta m}$ and $\frac{\delta F}{\delta m}$ are
  continuous, one gets that
  \[ \ell \rightarrow \int_E \frac{\delta F}{\delta m} (\varphi (\ell (\mu -
     \nu) + \nu) ; y) \left( \int_E \frac{\delta \varphi}{\delta m} (\ell (\mu
     - \nu) + \nu ; z) (\mu - \nu) (\dd z) \right) (\dd y) \]
  is continuous, and using standard result on Riemann sum , we get that
  \[ H_n (\mu, \nu) \rightarrow \int_0^1 \int_E \frac{\delta F}{\delta m}
     (\varphi (\ell (\mu - \nu) + \nu) ; y) \left( \int_E \frac{\delta
     \varphi}{\delta m} (\ell (\mu - \nu) + \nu ; z) (\mu - \nu) (\dd z)
     \right) (\dd y) \dd \ell . \]
  The boundedness hypothesis allows us to conclude, using Fubini theorem.
\end{proof}

\begin{lemma}
  \label{lemma:chain-rule-flow}Suppose that $F \in C^1 (\cM^+_1 ;
  \RR)$ and that for all $y \in E$, $\mu \rightarrow \frac{\delta
  F}{\delta m} (\mu ; y)$. Suppose furthermore that $\mu \in C ([0, T] ;
  \cM^+_1)$ and that ther exists $\mu' \in C ([0, T] ;
  \cM^+_1)$ such that for all $f \in C^0_b (E ; \RR)$,
  \[ \langle f, \mu (t) \rangle - \langle f, \mu (t) \rangle = \int_s^t
     \langle f, \mu' (r) \rangle \dd r. \]
  Then $F \circ \mu \in C^1 ([0, T] ; \RR)$ and
  \[ (F \circ \mu)' (t) = \left\langle \frac{\delta F}{\delta m} (\mu (t) ;
     \cdot), \mu' (t) \right\rangle . \]
\end{lemma}

\begin{proof}
  We use the same (but simpler) arguments as in the previous proof. Remark
  that for $n \ge 1$,
  
  \begin{align*}
    F (\mu (t)) - F (\mu (s)) = & \sum_{k = 0}^{n - 1} F \left( \mu \left(
    \frac{k + 1}{n} (t - s) + s \right) \right) - F \left( \mu \left(
    \frac{k}{n} (t - s) + s \right) \right)\\
    & \sum_{k = 0}^{n - 1} F \left( \int_0^1 \mu' \left( \frac{k + r}{n} (t -
    s) + s \right) (t - s) \dd r + \mu \left( \frac{k}{n} (t - s) + s
    \right) \right)\\
    & \qquad - F \left( \mu \left( \frac{k}{n} (t - s) + s \right) \right)\\
    = & \sum_{k = 0}^{n - 1} \int_E \frac{\delta F}{\delta m} \left( \ell
    \int_0^1 \mu' \left( \frac{k + r}{n} (t - s) + s \right) \frac{(t - s)}{n}
    \dd r + \mu \left( \frac{k}{n} (t - s) + s \right) \right)\\
    & \qquad \times \left( \int_0^1 \mu' \left( \frac{k + r}{n} (t - s) + s
    \right) \frac{(t - s)}{n} \dd r \right) (\dd z)
  \end{align*}
  
  Since $\mu'$ is continuous, it is uniformely continuous on $[s, t]$ and
  thanks to the hypothesis on $F$, the same computations show that the limit
  as $n \rightarrow + \infty$ of the previous expression is the same as the
  limit when $n \rightarrow + \infty$ of
  \[ \sum_{k = 0}^{n - 1} \int_E \frac{\delta F}{\delta m} \left( \mu \left(
     \frac{k}{n} (t - s) + s \right) \right) \left( \mu' \left( \frac{k}{n} (t
     - s) + s \right) \frac{(t - s)}{n} \right) (\dd z) . \]
  and we get that
  \[ F (\mu (t)) - F (\mu (s)) = \int_s^t \left\langle \frac{\delta F}{\delta
     m} (\mu (r) ; \cdot), \mu' (r) \right\rangle \dd r. \]
  The result the follows using the continuity of $\frac{\delta F}{\delta m}$,
  its boundedness in its second variable, the continuity of $\mu'$.
\end{proof}

\begin{corollary}
  Let $L \in C (\cM_1^+ ; \cM_1^+)$ and $F \in C ([0, T]
  \times \cM_1^+ ; \RR)$. Suppose furthermore that there exists
  a function $\varphi \in C ([0, T] \times \cM_1^+ ; \cM_1^+)$
  such that for
  \begin{itemize}
    \item For all $t \in [0, T]$, $\mu \rightarrow F (t, \mu) \in C^1
    (\cM_1^+ ; \RR)$ and $\mu \rightarrow \varphi (t, \mu) \in
    C (\cM_1^+ ; \cM_1^+)$ such that for all $t \in [0, T]$
    and all $z \in E$, $\mu \rightarrow \frac{\delta F}{\delta m} (t, \mu ;
    z)$ and $\mu \rightarrow \frac{\delta \varphi}{\delta m} (t, \mu ; z)$ are
    uniformely continuous.
    
    \item For all $\mu \in \cM_1^+$ and all $t \rightarrow F (t, \mu)
    \in C^1 (\cM_1^+ ; \RR)$
    
    \item For all $f \in C_b (E ; \RR_+)$,
    \[ \langle f, \varphi (t, \mu) \rangle = \langle f, \mu \rangle + \int_0^t
       \langle f, L (\varphi (r, \mu)) \rangle \dd r. \]
    \item For all $s, t \in [0, T]$ such that $s + t \in [0, T]$,
    \[ \varphi (t + s, \mu) = \varphi (t, \varphi (s, \mu)) . \]
  \end{itemize}
  Let us define
  \[ u (t, \mu) = F (t, \varphi (t, \mu)) . \]
  Then $u \in C ([0, T] \times \cM_1^+ ; \RR)$, for all $\mu
  \in \cM_1^+$, $t \rightarrow u (t, \mu) \in C^1 ([0, T] ;
  \RR)$, for all $t \in [0, T]$, $\mu \rightarrow u (t, \mu) \in C^1
  (\cM_1^+ ; \RR)$ and $u$ solves the following partial
  differential equation :
  \[ \partial_t u (t, \mu) = \left\langle \frac{\delta u}{\delta m} (t, \mu ;
     \cdot), L (\mu) \right\rangle, \quad u (0, \mu) = F (0, \mu) . \]
\end{corollary}

\begin{proof}
  The continuity of $u$ is straightforward since $F$ and $\varphi$ are
  continuous. Furthermore, using the previous Lemmas \ref{lemma:chain-rule}
  and \ref{lemma:chain-rule-flow}, $u$ admits a linear derivative in its
  measure variable for all time and is differentiable in time for all fixed
  measure.
  
  Now let $t \in [0, T]$ and $h > 0$. Remark first that for $\mu \in
  \cM_1^+$,
  \[ \left\langle f, \frac{\varphi (t + h, \mu) - \varphi (t, \mu)}{h}
     \right\rangle = \frac{1}{h} \int_t^{t + h} \langle f, L (\varphi (r,
     \mu)) \rangle \dd r, \]
  and by letting $h \rightarrow 0$, since $L$ is continuous and $\varphi$ is
  continuous in time, we get that $\partial_t u (t, \mu)$ exists and
  \[ \partial_t u (t, \mu) = L (\varphi (t, \mu)) . \]
  Remark also that since
  \[ \varphi (t, \varphi (s, \mu)) = \varphi (t + s, \mu) \]
  one gets by differentiating in $s$ using the previous Lemma
  \ref{lemma:chain-rule-flow},
  \[ \left\langle \frac{\delta \varphi}{\delta m} (t, \varphi (s, \mu) ;
     \cdot), \partial_s \varphi (s, \mu) \right\rangle = \partial_s \varphi
     (t + s, \mu) = L (\varphi (t + s, \mu)) . \]
  Taking $s = 0$, in the previous expression, one gets $\partial_s \varphi (s,
  \mu) = L (\varphi (0, \mu)) = L (\mu)$ and
  \[ \left\langle \frac{\delta \varphi}{\delta m} (t, \mu ; \cdot), L (\mu)
     \right\rangle = L (\varphi (t, \mu)) . \]
  Hence, we get, using Fubini theorem and Lemma \ref{lemma:chain-rule},
  for all $t \in [0, T]$, $\mu \rightarrow F (t, \varphi (t, \mu)) \in C^1
  (\cM_+^1 ; \RR)$, $\frac{\delta F (t, \varphi (t,
  \cdot))}{\delta m} (\mu ; z) = \left\langle \frac{\delta F}{\delta m} (t,
  \varphi (t, \mu) ; \cdot) ; \frac{\delta \varphi}{\delta m} (t, \mu ; z)
  \right\rangle$ and,
  
  \begin{align*}
    \left\langle \frac{\delta F}{\delta m} (t, \varphi (t, \mu) ; z), L
    (\varphi (t, \mu)) \right\rangle = & \int_E \frac{\delta F}{\delta m} (t,
    \varphi (t, \mu) ; y) \left( \int_E \frac{\delta \varphi}{\delta m} (t,
    \mu ; z) L (\mu) (\dd z) \right) (\dd y)\\
    = & \int_E \frac{\delta F (t, \varphi (t, \cdot))}{\delta m} (\mu, z) L
    (\mu) (\dd z) .\\
    = & \left\langle \frac{\delta u}{\delta m} (t, \mu ; z), L (\mu)
    \right\rangle .
  \end{align*}
  
  We also get that
  
  \begin{align*}
    \frac{F (t, \varphi (t + h, \mu)) - F (t, \varphi (t, \mu))}{h} = &
    \int_0^1 \int_E \frac{\delta F}{\delta m} (t, \ell (\varphi (t + h, \mu) -
    \varphi (t, \mu)) + \varphi (t, \mu) ; z)\\
    & \qquad \times \left( \frac{\varphi (t + h, \mu) - \varphi (t, \mu)}{h}
    \right) (\dd z) \dd \ell\\
    = & \int_0^1 \frac{1}{h} \int_t^{t + h} \int_E \frac{\delta F}{\delta m}
    (t, \ell (\varphi (t + h, \mu) - \varphi (t, \mu)) + \varphi (t, \mu) ;
    z)\\
    & \qquad \times L (\varphi (r, \mu)) (\dd z) \dd r \dd \ell
  \end{align*}
  
  Furthermore, since $\frac{\delta F}{\delta m}$ is continuous and bounded in
  its second variable, and since $L$ and $t \rightarrow \varphi (t, \mu)$ are
  continuous, one gets that the limit of the right hand side when $h
  \rightarrow 0$ and is equal to
  \[ \int_E \frac{\delta F}{\delta m} (t, \varphi (t, \mu) ; z) L (\varphi (t,
     \mu)) (\dd z) \]
  which implies that
  \[ \partial_t u (t, \mu) = \left\langle \frac{\delta F}{\delta m} (t,
     \varphi (t, \mu) ; \cdot), L (\varphi (t, \mu)) \right\rangle . \]
  Putting alltogether, we have just proved that
  \[ \partial_t u (t, \mu) = \left\langle \frac{\delta u}{\delta m} (t, \mu ;
     \cdot) ; L (\mu) \right\rangle, \quad u (0, \mu) = F (0 , \mu), \]
  which is the wanted result.
\end{proof}

\red{Ici, il faudrait montrer que si le flot $\varphi$, en en
particulier se d{\'e}riv{\'e}e seconde, a de bonnes propri{\'e}t{\'e}s, en
particulier que si
\[ \mu \rightarrow \frac{\delta \varphi}{\delta m} (t, \mu ; z) \]
est $\gamma$-H{\"o}lder continue et que $\mu \rightarrow \frac{\delta
F}{\delta m} (t, \mu ; z)$ est Lipschitz continue, alors $\mu \rightarrow
\frac{\delta u}{\delta m} (t, \mu ; z)$ est $\gamma$-H{\"o}lder continue, et
on voudrait que ce soit {\'e}ventuellement uniforme en $t$ et $z$. {\c c}a
semble possible, {\'e}ventuellement en demandant plus de r{\'e}gularit{\'e} en
$z$ pour $\frac{\delta F}{\delta m} (t, \mu ; z)$. En effet

\begin{align*}
  \frac{\delta u}{\delta m} (t, \mu ; z) - \frac{\delta u}{\delta m} (t, \nu ;
  z) = & \left\langle \frac{\delta F}{\delta m} (t, \varphi (t, \mu) ;
  \cdot), \frac{\delta \varphi}{\delta m} (t, \mu ; z) - \frac{\delta
  \varphi}{\delta m} (t, \nu ; z) \right\rangle\\
  & + \left\langle \frac{\delta F}{\delta m} (t, \varphi (t, \mu) ; \cdot)
  - \frac{\delta F}{\delta m} (t, \varphi (t, \mu) ; \cdot), \frac{\delta
  \varphi}{\delta m} (t, \nu ; z) \right\rangle
\end{align*}

il faudrait que
\[ y \rightarrow \frac{\delta F}{\delta m} (t, \varphi (t, \mu) ; y) \]
soit Lipschitz, et on aurait

\begin{multline*}
  \left| \left\langle \frac{\delta F}{\delta m} (t, \varphi (t, \mu) ;
  \cdot), \frac{\delta \varphi}{\delta m} (t, \mu ; z) - \frac{\delta
  \varphi}{\delta m} (t, \nu ; z) \right\rangle \right|\\
  \le \left( \left\llbracket \frac{\delta F}{\delta m} (t, \varphi (t,
  \mu) ; \cdot) \right\rrbracket_{ \text{ Lip } } + \left| \frac{\delta
  F}{\delta m} (t, \varphi (t, \mu) ; 1) \right| \right) W_1 \left(
  \frac{\delta \varphi}{\delta m} (t, \mu ; z), \frac{\delta \varphi}{\delta
  m} (t, \nu ; z) \right)
\end{multline*}

}

\red{
Plan de la preuve :
\begin{itemize}
  \item Existence et unicit{\'e} en temps court dans $\cE^K_c$ (Banach
  contraction Theorem avec $\| \cdot \|_{ \text{ TV } }$)
  
  \item Lemme pour la positivit{\'e} des solutions de Smoluchowski
  
  \item Bornes a priori : les solutions sont dans $C ([0, T] ;
  (\cE^{+, K}_c, W_1))$ pour un certain $K$ et pour le m{\^e}me $c$
  que la condition initiale
  
  \item Continuit{\'e} des solutions comme un flot pour $1 > \kappa > 0$
  quelconque
  \[ \varphi \in C^{1, \kappa} ([0, T] \times (\cE^{+, K}_c, W_1) ;
     (\cE^{+, K}_c, W_1)) \]
  \item Existence et unicit{\'e} locale de la d{\'e}riv{\'e} de Fr{\'e}chet de
  $\varphi$ dans
  \[ D \varphi \in C ([0, T] ; \cL_c ((\cE^K_c, \| \cdot
     \|_{ \text{ TV } }) ; (\cE^K_c, \| \cdot \|_{ \text{ TV } }))) \]
  \item Re lemme pour la positivit{\'e} des solutions, bornes a priori.
  
  \item Existence et unicit{\'e} de \ $t \rightarrow \delta_{\mu} \varphi (\mu
  ; z)$ et bornes a priori (m{\^e}me preuve que pr{\'e}c{\'e}dement)
  
  \item 
\end{itemize}
}

\section{Bihari lemma and log Lipschitz equations on metric space}

\begin{lemma}[Bihari's lemma]
  Let us define for any $x \in \RR_+^{\ast}$
  \[ l (x) = x \left( 1 + 0 \vee \log \left( \frac{1}{x} \right) \right),
     \quad l (0) = 0. \]
  Suppose that $y \in C ([0, T] ; \RR_+)$ and $a, b \in \RR_+$
  are such that
  \[ y (t) \le a + b \int_0^t l (y (s)) \dd s. \]
  Then following bounds hold.
  \[ y (t) \le C \left\{\begin{array}{ll}
       a e^{b t} &  \text{ if } a \ge 1\\
       e^{b t} \frac{1}{1 - \log (a)} &  \text{ if } e^{- (e^{b t} - 1)}
       \le a < 1\\
       2 a^{e^{- b t}} &  \text{ if } a < e^{- (e^{b t} - 1)}
     \end{array}\right. . \]
  In particular for $a = 0 $, $y (t) = 0$. 
\end{lemma}

\begin{proof}
  Let us define for $y > 0$
  \[ H (y) = \int_1^y \frac{1}{l (x)} \dd x = \left\{\begin{array}{ll}
       \log (y) &  \text{ if } y \ge 1\\
       \log \left( \frac{1}{1 - \log (y)} \right) &  \text{ if } y < 1
     \end{array}\right. . \]
  
  
  Note also that
  \[ H^{- 1} (x) = \left\{\begin{array}{ll}
       e^x &  \text{ if } x \ge 0\\
       \exp (1 - e^{- x}) &  \text{ if } x < 0
     \end{array}\right. . \]
  Define for all $t \in [0, T]$,
  \[ Y (t) = a + b \int_0^t l (y (s)) \dd s. \]
  Since $l$ is increasing, we have
  \[ Y' (t) = b l (y (t)) \le b l (Y (t)) . \]
  We suppose first that $a > 0$, by integrating, reminding that $Y (0) = a$
  and that $H^{- 1}$ is increasing,
  \[ H (Y (t)) - H (a) \le b t \quad  \text{ and } \quad Y (t) \le
     H^{- 1} (H (a) + b t) . \]
  Now, suppose that $a \ge 1$, then
  \[ Y (t) \le e^{\log (a) + b t} = a e^{b t} . \]
  Suppose now that $a < 1$ and $0 \le H (a) + b t = \log \left(
  \frac{1}{1 - \log (a)} \right) + b t$, namely $\exp (- (e^{b t} - 1))
  \le a$, one gets
  \[ Y (t) \le e^{b t} \frac{1}{1 - \log (a)} . \]
  Now suppose that $a < 1$ and $a < \exp (1 - \exp (b t))$ then
  \[ Y (t) \le \exp \left( 1 - \exp \left( - \log \left( \frac{1}{1 -
     \log (a)} \right) - b t \right) \right) = \exp (1 - e^{- b t} (1 - \log
     (a))) \le e^{1 - e^{- b t}} a^{e^{- b t}} \le 2 a^{e^{- b t}}
  \]
  which ends the proof whenever $a \neq 0$. For $a = 0$, remark that $Y' (t) =
  b l (y (t)) \ge 0$ and $Y$ is non-decreasing. Let $t_0 = \inf (t : Y
  (t) > 0) \wedge T$. If $t_0 < T$, let $t_0 < t_1 < T$, and $Y (t_1) > 0$
  since $Y$ is non-decreasing, continuous and $t_0$ is the last time where $Y
  (t_0) = 0$. One can do the same as before, and we get for $t \ge t_0$,
  \[ Y (t) \le H^{- 1} (H (Y (t_1)) + b (t - t_1)) . \]
  By letting $t_1 \rightarrow t_0$, one gets that $Y (t_1) \rightarrow Y (t_0)
  = 0$, and $H (Y (t_1)) + b (t - t_1) \rightarrow - \infty$. Since $\lim_{x
  \rightarrow X - \infty} H^{- 1} (x) = 0$, one gets that for $t \ge
  t_0$, $Y (t) = 0$, hence $t_0 = t$ and $Y (t) = 0$ and $y (t) = 0$.
\end{proof}

\begin{theorem}
  Suppose that $F : B \rightarrow B$ \red{Faire dans un espace
  m{\'e}trique}
\end{theorem}

The proof is taken from

\begin{proof}
  \ d
\end{proof}

\begin{lemma}
  Let $p \ge 2,$ and suppose that $\eta \in C ([0, T] ;
  \cM^+_p)$. Let $K, c > 0$ and Suppose furthermore that $\mu, \nu \in
  C ([0, T] ; \cE^{+, K}_c)$ solve the following equation :
  \[ \langle f, \mu_t \rangle = \langle f, \mu_0 \rangle + \int_0^t
     \int_{\RR_+^{\ast}} \left( \int_{\RR^{\ast}_+} K_{\alpha}
     (x, y) (f (x + y) - f (x) - f (y)) \eta_t (\dd y) \right) \mu_s
     (\dd x) \dd s. \]
  Then
  \[ W_1 (\mu_t, \nu_t) \le W_1 (\mu_0, \nu_0) + C \int_0^t l (W_1
     (\mu_s, \nu_s)) \dd s, \]
  where
  \[ l (x) = x \left( 1 + 1 \vee \log \left( \frac{1}{x} \right) \right) \]
  and $C$ is a constant which depends on $\eta$, $c$ and $K$.
\end{lemma}

\begin{proof}
  Take $f \in  \text{ Lip } (\RR_+^{\ast}, d_{\alpha})$ with $\llbracket f
  \rrbracket_{ \text{ Lip } } \le 1$ and $f (1) = 0$, $1 \ge \delta >
  0$ and define
  \[ F_{\delta} (t, x) = \int_{\RR^{\ast}_+} K_{\alpha} (x, y) (f (x +
     y) - f (x) - f (y)) \eta_t (\dd y)  \text{ for } x \ge \delta \]
  and
  \[ F_{\alpha} (t, x) = \int_{\RR^{\ast}_+} K_{\alpha} (\delta, y) (f
     (x + y) - f (x) - f (y)) \eta_t (\dd y)  \text{ for } x < \delta . \]
  Remark that for $x, \tilde{x} \ge \delta$ and $y \in
  \RR_+^{\ast}$,
  
  \begin{align*}
    | x^{- \alpha} f (x + y) - \tilde{x}^{- \alpha} f (\tilde{x} + y) |
    \le & | x^{- \alpha} - \tilde{x}^{- \alpha} | | f (x + y) | +
    \tilde{x}^{- \alpha} | f (x + y) - f (\tilde{x} + y) |\\
    \le & d_{\alpha} (x, \tilde{x}) \left| \frac{1}{(x + y)^{\alpha}} -
    1 \right| + \tilde{x}^{- \alpha} | f (x + y) - f (\tilde{x} + y) |\\
    \le & 3 \delta^{- \alpha} d_{\alpha} (x, \tilde{x}),
  \end{align*}
  
  where we have used that $\left| \frac{1}{(x + y)^{\alpha}} - 1 \right|
  \le x^{- \alpha} + 1 \le 2 \delta^{- \alpha}$ since $x \ge
  \delta$ and $\delta \le 1$. Hence for $x, \tilde{x} \ge \delta$,
  
  \
  
  \begin{align*}
    | F_{\delta} (t, x) - F_{\delta} (t, \tilde{x}) | \le &
    \int_{\RR^{\ast}_+} | x^{- \alpha} f (x + y) - \tilde{x}^{- \alpha}
    f (\tilde{x} + y) | \eta_t (\dd y) + | x^{- \alpha} f (x) -
    \tilde{x}^{- \alpha} f (\tilde{x}) | \int_{\RR^{\ast}_+} \eta_t
    (\dd y)\\
    & + | x^{- \alpha} - \tilde{x}^{- \alpha} | \int_{\RR^{\ast}_+} |
    f (y) | \eta_t (\dd y) + \int_{\RR^{\ast}_+} y^{- \alpha} | f (x
    + y) - f (\tilde{x} + y) | \eta_t (\dd y)\\
    & + | f (x) - f (\tilde{x}) | \int_{\RR^{\ast}_+} y^{- \alpha}
    \eta_t (\dd y) \\
    \le & 9 \delta^{- \alpha} d_{\alpha} (x, \tilde{x}) \left(
    \int_{\RR_+^{\ast}} 1 + y^{- \alpha} \eta_t (\dd y) \right)\\
    \le & 9 \delta^{- \alpha} \underset{t \in [0, T]}{\sup} \left(
    \int_{\RR_+^{\ast}} 1 + y^{- \alpha} \eta_t (\dd y) \right)
    d_{\alpha} (x, \tilde{x}) .
  \end{align*}
  
  For $x, \tilde{x} < \delta$,
  \[ | F_{\delta} (t, x) - F_{\delta} (t, \tilde{x}) | \le 9 \delta^{-
     \alpha} \underset{t \in [0, T]}{\sup} \left( \int_{\RR_+^{\ast}} 1
     + y^{- \alpha} \eta_t (\dd y) \right) d_{\alpha} (x, \tilde{x}) . \]
  and for $\tilde{x} < \delta \le x$, remark that $d_{\alpha} (x,
  \delta) + d_{\alpha} (\delta, \tilde{x}) = d_{\alpha} (x, \tilde{x})$ and
  \[ | F_{\delta} (t, x) - F_{\delta} (t, \tilde{x}) | \le | F_{\delta}
     (t, x) - F_{\delta} (t, \delta) | + | F_{\delta} (t, \delta) - F_{\delta}
     (t, \tilde{x}) | \le 9 \delta^{- \alpha} \underset{t \in [0,
     T]}{\sup} \left( \int_{\RR_+^{\ast}} 1 + y^{- \alpha} \eta_t
     (\dd y) \right) d_{\alpha} (x, \tilde{x}) . \]
  Hence, $F_{\alpha}$ is Lipschitz
  
  \ 
\end{proof}




\section{Smoluchowski equation}

In this part, we focus ourselves on the Smoluchowski equation with kernel
$K_{\alpha}$ as defined in Equation \eqref{eq:def-Kalpha}. We are strongly
inspired by the presentation of {\cite{kolokoltsovCentralLimitTheorem2008}}
and the method of proof of
{\cite{norrisSmoluchowskisCoagulationEquation1999}}.

Let us consider the following Smoluchowski coagulation equation, which is a
equation in $\cM (\RR_+^{\ast})$ the space of non-negative
finite measure of $\RR_+^{\ast}$ and may be formally written for any
$\mu_0 \in \cM (\RR_+^{\ast})$ as
\begin{equation}
  \mu_t = \mu_0 + \frac{1}{2} \int_0^t \int_0^x K_{\alpha} (x, y) \mu_t (x -
  y) \mu_t (y) \dd y \dd r. \label{eq:smoluchowsli}
\end{equation}
In order to give a meaning to the solutions of the previous equation, one may
use its weak form.

\begin{definition}
  A continuous function $\mu : [0, T] \rightarrow (\cM_1^+
  ((\RR_+^{\ast}, d_{\alpha})), W_1)$ is a solution to the Smoluchowski
  Equation \eqref{eq:smoluchowsli} if for any $f \in C_b
  ((\RR_+^{\ast}, d_{\alpha}) ; \RR)$,
  \begin{equation}
    \langle f, \mu_t \rangle = \langle f, \mu_0 \rangle + \frac{1}{2} \int_0^t
    \int_{\RR_+^{\ast}} \int_{\RR_+^{\ast}} K_{\alpha} (x, y) (f
    (x + y) - f (y) - f (x)) \mu_r (\dd y) \mu_r (\dd x) \dd r.
    \label{eq:smolu-weak}
  \end{equation}
\end{definition}

In order to prove existence, uniqueness and regularity of Smoluchowski
equation, we will first solve it locally in time in $\cM_1$, the prove
that the solution is necessarely non-negative, and will lie in
$\cM_1^+$. Finally, will will prove that the solution is global in
time.

\begin{definition}
  \label{def:Lalpha}Let us define $L_{\alpha} : \cM_1 \rightarrow
  \cM_1$ such that for all $f \in C_b ((\RR_+, d_{\alpha}) ;
  \RR)$,
  \[ \langle f, L_{\alpha} (\mu) \rangle = \frac{1}{2}
     \int_{\RR^{\ast}_+} \int_{\RR^{\ast}_+} K_{\alpha} (x, y)
     (f (x + y) - f (x) - f (y)) \mu (\dd x) \mu (\dd y) . \]
  And for all $f \in C_b ((\RR_+, d_{\alpha}) ; \RR)$ let us
  define $K_{\alpha} f$ for all $(x, y) \in \RR_+^{\ast}$,
  \[ K_{\alpha} f (x, y) = K_{\alpha} (x, y) (f (x + y) - f (x) - f (y)), \]
  such that we have the following identity
  \[ \langle f, L_{\alpha} (\mu) \rangle = \langle K_{\alpha} f, \mu \otimes
     \mu \rangle . \]
\end{definition}

With the previous definition, a solution of the Smoluchowski equation is a
continuous function from $[0, T]$ to $\cM_1$, $t \rightarrow \mu_t$
such that
\[ \langle f, \mu_t \rangle = \langle f, \mu_0 \rangle + \int_0^t \langle f,
   L_{\alpha} (\mu_s) \rangle \dd s = \langle f, \mu_0 \rangle + \int_0^t
   \langle K_{\alpha} f, \mu_s \otimes \mu_s \rangle \dd s. \]
We have the following straightforward proposition, which allows us to write
the previous equation.

\begin{lemma}
  \label{lemma:jensen-Kalpha}Let $p \ge q > \frac{- 1}{\alpha}$ and let
  $\mu \in \cK^+_p$ be non zero, then
  \[ \langle x^{- \alpha p}, \mu \rangle \ge \langle x^{- \alpha q}, \mu
     \rangle^{\frac{\alpha p + 1}{\alpha q + 1}} \langle x, \mu \rangle^{-
     \frac{\alpha (p - q)}{\alpha q + 1}} \]
\end{lemma}

\begin{proof}
  First, remark that since $\mu$ is a finite non-negative measure,
  $\tilde{\mu}$ defined for all bounded and continuous $f$ as
  \[ \langle f, \tilde{\mu} \rangle = \int_0^{+ \infty} f (x) \frac{x \mu
     (\dd x)}{\langle x, \mu \rangle} \]
  is a probability measure (unless $\langle x, \mu \rangle = 0$ which would
  imply that $\mu = 0$). We have
  
  \
  
  \begin{align*}
    \langle x^{- \alpha p}, \mu \rangle = & \langle x, \mu \rangle \langle
    x^{- (\alpha p + 1)}, \tilde{\mu} \rangle\\
    = & \langle x, \mu \rangle \left\langle x^{- (\alpha q + 1) \frac{\alpha p
    + 1}{\alpha q + 1}}, \tilde{\mu} \right\rangle .
  \end{align*}
  
  Thanks to the requirement on $p$ and $q$, $\frac{\alpha p + 1}{\alpha q + 1}
  \ge 1$ and $y \rightarrow y^{\frac{\alpha p + 1}{\alpha q + 1} }$ is a
  convexe function. Hence
  
  \begin{align*}
    \langle x^{- \alpha p}, \mu \rangle \ge & \langle x, \mu \rangle^{-
    \frac{\alpha p + 1}{\alpha q + 1} + 1} \left\langle \frac{x}{x^{\alpha q +
    1}}, \mu \right\rangle^{\frac{\alpha p + 1}{\alpha q + 1}}\\
    = & \langle x, \mu \rangle^{- \frac{\alpha (p - q)}{\alpha q + 1}} \langle
    x^{- \alpha q}, \mu \rangle^{\frac{\alpha p + 1}{\alpha q + 1}}
  \end{align*}
  
  which is the wanted result. 
\end{proof}

\begin{proposition}
  \label{prop:def-Lalpha}The operator $L_{\alpha}$ is well defined from
  $\cM_1$ to $\cM$, furthermore if $\langle x, | \mu | \rangle
  < + \infty$ then $\langle x, L_{\alpha} (\mu) \rangle = 0$.
\end{proposition}

\begin{proof}
  First, remark that for any Borel set $A \in \RR_+^{\ast}$, and any
  $x, y \in \RR_+^{\ast}$,
  \[ - 2 \le \mathbbm{1}_A (x + y) - \mathbbm{1}_A (y) - \mathbbm{1}_A
     (x) \le 1. \]
  Remark also that
  \[ \frac{1}{2} \int_{\RR_+^{\ast}} \int_{\RR_+^{\ast}}
     K_{\alpha} (x, y) \mu (\dd x) \mu (\dd y) = \langle x^{- \alpha},
     \mu \rangle \langle 1, \mu \rangle . \]
  Hence, for any Borel set $A \in \RR_+^{\ast}$,
  \[ | \langle \mathbbm{1}_A, L_{\alpha} (\mu) \rangle | \le \langle
     x^{- \alpha}, | \mu | \rangle \langle 1, | \mu | \rangle, \]
  where we remind that $| \mu |$ denotes the total variation of the signed
  measure $\mu$. This last inequality guaranties that $L_{\alpha} (\mu)$ is
  well-defined. Using the fact that for a sequence $(A_i)_{i \ge 0}$ of
  disjoints Borel sets of $\RR_+^{\ast}$ and for all $x, y \in
  \RR_+^{\ast}$,
  \[ \left| \mathbbm{1}_{\bigcup_{i \ge 0} A_i} (x + y) -
     \mathbbm{1}_{\bigcup_{i \ge 0} A_i} (x) - \mathbbm{1}_{\bigcup_{i
     \ge 0} A_i} (y) \right| \le 2 \]
  and
  \[ \mathbbm{1}_{\bigcup_{i \ge 0} A_i} (x + y) -
     \mathbbm{1}_{\bigcup_{i \ge 0} A_i} (x) - \mathbbm{1}_{\bigcup_{i
     \ge 0} A_{i^2}} (y) = \sum_{i \ge 0} \mathbbm{1}_{A_i} (x +
     y) - \mathbbm{1}_{A_i} (x) - \mathbbm{1}_{A_i} (y), \]
  one gets that for $N \ge 0$,
  \[ \left| \frac{1}{2} K_{\alpha} (x, y) \left( \sum_{i = 0}^N
     \mathbbm{1}_{A_i} (x + y) - \mathbbm{1}_{A_i} (x) - \mathbbm{1}_{A_i} (y)
     \right) \right| \le K_{\alpha} (x, y) \in L^1 (\mu \otimes \mu), \]
  and thanks to the dominating convergence theorem,
  \[ \underset{N \rightarrow + \infty}{\lim} \sum_{i = 0}^N \langle
     \mathbbm{1}_{A_i}, L_{\alpha} (\mu) \rangle = \left\langle
     \mathbbm{1}_{\bigcup_{i \ge 0} A_i}, L_{\alpha} (\mu) \right\rangle
     . \]
  Furthermore,
  
  \begin{align*}
    \sum_{i = 0}^N | \langle \mathbbm{1}_{A_i}, L_{\alpha} (\mu) \rangle |
    \le & \int_{\RR^{\ast}_+} \int_{\RR^{\ast}_+} \sum_{i
    = 0}^N K_{\alpha} (x, y) (\mathbbm{1}_{A_i} (x + y) + \mathbbm{1}_{A_i}
    (x) + \mathbbm{1}_{A_i} (y)) | \mu | \otimes | \mu | (\dd x, \dd
    y)\\
    \le & \langle x^{- \alpha}, | \mu | \rangle \langle 1, | \mu |
    \rangle,
  \end{align*}
  
  which implies that
  \[ \left\langle \mathbbm{1}_{\bigcup_{i \ge 0} A_i}, L_{\alpha} (\mu)
     \right\rangle = \underset{N \rightarrow + \infty}{\lim} \sum_{i = 0}^N
     \langle \mathbbm{1}_{A_i}, L_{\alpha} (\mu) \rangle = \sum_{i \ge
     0} \langle \mathbbm{1}_{A_i}, L_{\alpha} (\mu) \rangle, \]
  and $L_{\alpha} (\mu)$ is a finite signed measure.
  
  Finally suppose that $\langle x , | \mu | \rangle < + \infty$, in Lemma
  \ref{lemma:jensen-Kalpha} by taking $q = 1 - \frac{1}{\alpha}$ and $p = 1$,
  \[ \langle x^{- (\alpha - 1)}, | \mu | \rangle \le \langle x^{-
     \alpha}, | \mu | \rangle^{\frac{\alpha + 1}{\alpha}} \langle 1, | \mu |
     \rangle^{- \frac{1}{\alpha}} < + \infty, \]
  which implies that
  \[ (x, y) \rightarrow \left( \frac{1}{x^{\alpha}} + \frac{1}{y^{\alpha}}
     \right) (x + y) \in L^1 (| \mu | \otimes | \mu |), \]
  Fubini theorem allows us to conclude that $(x \rightarrow x) \in L^1
  (L_{\alpha} (\mu))$ and that $\langle x, L_{\alpha} (\mu) \rangle = 0$.
\end{proof}

In this section, we are interested by the well-posedness of the previous
equation. Indeed, we will show the following Theorem:

\red{

\begin{theorem}
  For all $\delta > 0$, $\alpha > 0$ and $T > 0$ there exists a unique
  function
  \[ \varphi : C^2 ((\cK_{2 + \delta}^+, W_1) ; C ([0, T] ;
     (\cK_{2 + \delta}^+, W_1))) \]
  such that for all $\mu, \nu \in \cM_{2 + \delta}$ and all $f \in
   \text{ Lip } ((\RR_+^{\ast}, d_{\alpha}) ; \RR)$,
  \[ \langle f, \varphi (\mu)_t \rangle = \langle f, \mu \rangle + \frac{1}{2}
     \int_0^t \langle f, L_{\alpha} (\varphi (\mu)_r) \rangle \dd r =
     \langle f, \mu \rangle + \frac{1}{2} \int_0^t \langle K_{\alpha} f,
     \varphi (\mu)_r \otimes \varphi (\mu)_r \rangle \dd r, \]
  furthermore for all $t \in [0, T]$
  
  \begin{align*}
    \langle f, \varphi (\mu)_t - \varphi (\nu)_t \rangle = & \int_0^1
    \left\langle f, \frac{\delta \varphi}{\delta m} (\ell (\mu - \nu) + \nu)
    (z) \right\rangle (\mu - \nu) (\dd z) \dd \ell .\\
    & \langle f, \langle \delta_{\mu} \varphi (\nu)_t, (\mu - \nu) \rangle
    \rangle + \int_0^1 (1 - \lambda) \langle f, \langle \delta_{\mu \mu}^2
    \varphi (\lambda (\mu - \nu) + \nu)_t, (\mu - \nu)^{\otimes 2} \rangle
    \rangle
  \end{align*}
  
  Finally, suppose that $F \in C^1 (\cM_1^{2 + \delta}, \RR)$,
  
  \begin{multline*}
    F (\varphi (\mu)_t) = F (\mu) + \frac{1}{2} \int_0^t
    \int_{\RR^{\ast}_+} \int_{\RR^{\ast}_+} K_{\alpha} (x, y)\\
    \times (\delta_{\mu} F (\varphi (\mu)_s) (x + y) - \delta_{\mu} F (\varphi
    (\mu)_s) (x) - \delta_{\mu} F (\varphi (\mu)_s) (y)) \varphi (\mu)_s
    (\dd y) \varphi (\mu)_s (\dd x) .
  \end{multline*}
\end{theorem}}

To prove the previous, we need several lemmas and definitions.

\begin{definition}
  Let $\delta, \alpha, T > 0$ and $\mu_0 \in \cM_1$, for $\mu \in C
  ([0, T] ; \cM_1)$ let us define $\Gamma (\mu)$ for all $t \in [0,
  T]$,
  \[ \langle f, \Gamma (\mu)_t \rangle = \langle f, \mu_0 \rangle + \int_0^t
     \langle f, L_{\alpha} (\mu_s) \rangle \dd s, \]
  where $L_{\alpha}$ is defined in Definition \ref{def:Lalpha}.
\end{definition}

\begin{proposition}
  \label{prop:gamma-well-defined}Let $K > 0$, the function $\Gamma$ is
  well-defined from $C ([0, T] ; (\cM_1^{1, K}, \| \cdot \|))$ to $C
  ([0, T], (\cM, \| \cdot \|))$.
\end{proposition}

\begin{proof}
  Thanks to the continuity of $t \in [0, T] \rightarrow \mu_t \in
  \cM_1$ and the definition of $L_{\alpha}$ (and in particular
  Proposition \ref{prop:def-Lalpha}), for all $t \in [0, T]$, $\Gamma (\mu)$
  is well defined as a function from $[0, T]$ to $\cM$. Furthermore,
  for $0 \le s \le t \le T$ and for any $f \in C_0
  ((\RR_+^{\ast}, d_{\alpha}) ; \RR)$ with $\| f \|_{\infty}
  \le 1$,
  
  \begin{align*}
    | \langle f, \Gamma (\mu)_t - \Gamma (\mu)_s \rangle | = & \left| \int_s^t
    \langle f, L (\mu_r) \rangle \dd r \right|\\
    \le & \int_s^t \int_{\RR^{\ast}_+} \int_{\RR^{\ast}_+}
    K_{\alpha} (x, y) | (f (x + y) - f (x) - f (y)) | | \mu_r | (\dd x) |
    \mu_r | (\dd y) \dd r\\
    \le & \int_s^t \langle 1, | \mu_r | \rangle \langle x^{- \alpha}, |
    \mu_r | \rangle \dd r
  \end{align*}
  
  Note that since $\mu \in C ([0, T] ; \cM^{1, K}_1)$, for all $r \in
  [0, T]$, $\langle 1, | \mu_r | \rangle \langle x^{- \alpha}, | \mu_r |
  \rangle \lesssim K^2$ and
  \[ | \langle f, \Gamma (\mu)_t \rangle - \langle f, \Gamma (\mu)_s \rangle |
     \lesssim K^2 | t - s | . \]
  taking the supremum on $f \in C_0 ((\RR_+^{\ast}, d_{\alpha}) ;
  \RR)$ with $\| f \|_{\infty} \le 1$, one gets that
  \[ \| \Gamma (\mu)_t - \Gamma (\mu)_s \| \lesssim K^2 | t - s | \]
  and $\Gamma (\mu) \in C ([0, T] ; \cM)$.
\end{proof}

\begin{proposition}
  \label{prop:contraction}Let $\eps > 0$ and $K > 0$ and $0 < \alpha
  \le 1$. Define
  \[ \cC_{\eps, K} = \{ \mu \in \cM_1^{1, K} :
      \text{ supp } \mu \subset [\eps, + \infty) \} . \]
  and take $\mu_0 \in \bigcup_{K > 0} \cC_{\eps, K}$ . A
  constant $K_1 = K_1 (\mu_0) > 0$ and a time $\tau_1 (\eps) = \tau_1
  (\mu_0, \eps)$ exists such that
  \[ \Gamma \in  \text{ Lip } (C ([0, \tau_1 (\eps)] ;
     (\cC_{\eps, K_1}, \| \cdot \|)) ; C ([0, \tau_1
     (\eps)] ; (\cC_{\eps, K_1}, \| \cdot \|))) \]
  with its Lipschitz constant smaller than $\frac{1}{2}$.
\end{proposition}

\begin{proof}
  Remark that $\{ \mu \in \cM:  \text{ supp } \mu \subset [\eps, +
  \infty) \}$ is closed in $(\cM, \| \cdot \|)$, hence
  $\cC_{\eps, K}$ is also closed in this space. Furthermore,
  remark that if $x, y \ge \eps$, then $x + y \ge
  \eps$. Suppose that $A \subset (0, \eps)$ is measurable and
  $ \text{ supp } \mu \subset [\eps, + \infty)$, $\langle \mathbbm{1}_A,
  L_{\alpha} (\mu) \rangle = 0$ and $\langle \mathbbm{1}_A, \mu_0 \rangle =
  0$. Hence, $\langle \mathbbm{1}_A, \Gamma (\mu)_t \rangle = 0$ and
  $ \text{ supp } \Gamma (\mu)_t \subset [\eps, + \infty)$. Furthermore,
  \[ \langle 1, | \Gamma (\mu)_t | \rangle \le \langle 1, | \mu_0 |
     \rangle + \int_0^t \langle 1, | \mu_r | \rangle \langle x^{- \alpha}, |
     \mu_r | \rangle \dd r \le \langle 1, | \mu_0 | \rangle + t K^2,
  \]
  and using the fact that $ \text{ supp } \mu_r \subset [\eps, + \infty)$,
  which leads to $\langle x^{- \alpha}, | \mu_r | \rangle \le
  \eps^{- \alpha} \langle 1, | \mu_r | \rangle$, Jensen inequality for
  $y \rightarrow y^{1 - \alpha}$ and the fact that $\langle 1, | \mu_r |
  \rangle \le K$ and that $\langle x, | \mu_r | \rangle \le K$,
  one gets
  
  \begin{align*}
    \langle x, | \Gamma (\mu)_t | \rangle \le & \langle x, | \mu_0 |
    \rangle + 2 \int_0^t \langle x, | \mu_r | \rangle \langle x^{- \alpha}, |
    \mu_r | \rangle + \langle x^{1 - \alpha}, | \mu_r | \rangle \langle 1, |
    \mu_r | \rangle \dd r\\
    \le & \langle x, | \mu_0 | \rangle + 2 t K^2 \eps^{- \alpha}
    + 2 \int_0^t \langle x, | \mu_r | \rangle^{1 - \alpha} \langle 1, | \mu_r
    | \rangle^{1 + \alpha} \dd r\\
    \le & \langle x, | \mu_0 | \rangle + 2 t K^2 (1 + \eps^{-
    \alpha}) .
  \end{align*}
  
  The same argument allows us to write for a certain constant $c_{\alpha} >
  0$,
  \[ \langle x^{- \alpha}, | \Gamma (\mu)_t | \rangle \le \langle x^{-
     \alpha}, | \mu_0 | \rangle + c_{\alpha} \int_0^t \langle 1, | \mu_r |
     \rangle \langle x^{- 2 \alpha}, | \mu_r | \rangle \dd r \le
     \langle x^{- \alpha}, | \mu_0 | \rangle + c_{\alpha} K^2 t \eps^{-
     2 \alpha} . \]
  Putting all together, ones gets that a constant $c_{\alpha} > 0$ exists such
  that
  \[ \langle 1 + x + x^{- \alpha}, | \Gamma (\mu)_t | \rangle \le
     \langle 1 + x + x^{- \alpha}, | \mu_0 | \rangle + t c_{\alpha} K^2 (1 +
     \eps^{- \alpha} + \eps^{- 2 \alpha}) . \]
  One can then choose $K_0 = 2 \langle 1 + x + x^{- \alpha}, | \mu_0 |
  \rangle$ and
  \[ \tau_0 (\eps) = \frac{1}{4 c_{\alpha} \langle 1 + x + x^{-
     \alpha}, | \mu_0 | \rangle (1 + \eps^{- \alpha} + \eps^{- 2
     \alpha})}, \]
  and we have
  \[ \underset{t \in [0, \tau_0 (\eps)]}{\sup} \langle 1 + x + x^{-
     \alpha}, | \Gamma (\mu)_t | \rangle \le K_0, \]
  which proves that $\Gamma$ is a function from $C ([0, \tau_0 (\eps)],
  (\cC_{\eps, K_0}, \| \cdot \|))$ to itself. Furthermore,
  take $f \in C_0 ((R_+^{\ast}, d_{\alpha}) ; \RR)$, we get for $\mu
  \in C ([0, \tau_0 (\eps)], (\cC_{\eps, K_0}, \|
  \cdot \|))$ and $r \in [0, \tau_0 (\eps)]$,
  
  \begin{align*}
    \langle f, L_{\alpha} (\mu_r) \rangle = & \frac{1}{2}
    \int_{\RR^{\ast}_+} \int_{\RR^{\ast}_+} \left(
    \frac{1}{x^{\alpha}} + \frac{1}{y^{\alpha}} \right) (f (x + y) - f (x) - f
    (y)) \mu_r \otimes \mu_r (\dd x, \dd y)\\
    = & \int_{\RR^{\ast}_+} \int_{\RR^{\ast}_+}
    \frac{1}{x^{\alpha}} f (x + y) \mu_r \otimes \mu_r (\dd x, \dd y)\\
    & - \int_{\RR^{\ast}_+} x^{- \alpha} f (x) \mu_r (\dd x)
    \langle 1, \mu_r \rangle - \int_{\RR^{\ast}_+} f (x) \mu_r (\dd
    x) \langle y^{- \alpha}, \mu_r \rangle\\
    = & \int_{\RR^{\ast}_+} \left( \int_{\RR^{\ast}_+}
    \frac{1}{x^{\alpha}} \theta (x) f (x + y) \mu_r (\dd x) \right) \theta
    (y) \mu_r (\dd y)\\
    & - \int_{\RR^{\ast}_+} x^{- \alpha} \theta (x) f (x) \mu_r
    (\dd x) \langle 1, \mu_r \rangle - \int_{\RR^{\ast}_+} f (x)
    \mu_r (\dd x) \langle y^{- \alpha}, \mu_r \rangle,
  \end{align*}
  
  where $\theta \in C ((\RR^{\ast}_+, d_{\alpha}) ; \RR)$ with
  $\theta = 1$ on $[\eps, \infty)$, $0 \le \theta \le 1$
  and $\theta = 0$ on $\left( 0, \frac{\eps}{2} \right)$. Now remark
  that
  \[ x \rightarrow x^{- \alpha} \theta (x) f (x) \in C_0
     ((\RR_+^{\ast}, d_{\alpha}) ; \RR) \quad  \text{ and } \quad x
     \rightarrow f (x) \in C_0 ((\RR_+^{\ast}, d_{\alpha}) ;
     \RR) . \]
  Furthermore, for $y \le \frac{\eps}{2}$,
  \[ \left( \int_{\RR^{\ast}_+} \frac{1}{x^{\alpha}} \theta (x) f (x +
     y) \mu_r (\dd x) \right) \theta (y) = 0 \]
  and
  \[ \left| \left( \int_{\RR^{\ast}_+} \frac{1}{x^{\alpha}} \theta (x)
     f (x + y) \mu_r (\dd x) \right) \theta (y) \right| \le K_0 \in
     L^1 (\mu_r) . \]
  Finally, let $\delta > 0$ and $A \subset \RR_+^{\ast}$ such that for
  all $z \notin A$, $| f (z) | \le \delta$. Since $A$ is a compact set,
  it is bounded, and a constant $a > 0$ exists such that for $z > a$, $z
  \notin A$, and for $y > z$,
  \[ \left| \left( \int_{\RR^{\ast}_+} \frac{1}{x^{\alpha}} \theta (x)
     f (x + y) \mu_r (\dd x) \right) \theta (y) \right| \le K_0
     \delta . \]
  Hence for $z \notin \left[ \frac{\eps}{2}, a \right]$, $\left| \left(
  \int_{\RR^{\ast}_+} \frac{1}{x^{\alpha}} \theta (x) f (x + y) \mu_r
  (\dd x) \right) \theta (y) \right| \le K_0 \delta$ and thanks to
  the domintated convergence theorem,
  \[ y \rightarrow \left( \int_{\RR^{\ast}_+} \frac{1}{x^{\alpha}}
     \theta (x) f (x + y) \mu_r (\dd x) \right) \theta (y) \in C_0
     ((\RR_+^{\ast}, d_{\alpha}) ; \RR) . \]
  The same argment allows us to state that
  \[ x \rightarrow \frac{1}{x^{\alpha}} \theta (x) \left(
     \int_{\RR^{\ast}_+} f (x + y) \theta (y) \nu_r (\dd y) \right)
     \in C_0 ((\RR_+^{\ast}, d_{\alpha}) ; \RR) \]
  with its sup norm bounded by $\eps^{- \alpha} K_0$.
  
  Now take $\tau_1 (\eps) \le \tau_0 (\eps)$ and \ $\mu,
  \nu \in C ([0, \tau_1 (\eps)], (\cC_{\eps, K_0}, \|
  \cdot \|))$. One gets for any $t \in [0, \tau_1 (\eps)]$
  
  \begin{align*}
    | \langle f, \Gamma (\mu)_t - \Gamma (\nu)_t \rangle | \le &
    \int_0^t | \langle f, (L_{\alpha} (\mu_r) - L_{\alpha} (\nu_r)) \rangle |
    \dd r\\
    \le & \int_0^t \left| \int_{\RR^{\ast}_+} \left(
    \int_{\RR^{\ast}_+} \frac{1}{x^{\alpha}} \theta (x) f (x + y) \mu_r
    (\dd x) \right) \theta (y) (\mu_r (\dd y) - \nu_r (\dd y))
    \right| \dd r\\
    & + \int_s^t \left| \int_{\RR^{\ast}_+} \frac{1}{x^{\alpha}}
    \theta (x) \left( \int_{\RR^{\ast}_+} f (x + y) \theta (y) \nu_r
    (\dd y) \right) (\mu_r (\dd x) - \nu_r (\dd x)) \right| \dd
    r\\
    & + \int_0^t \left| \int_{\RR^{\ast}_+} x^{- \alpha} \theta (x) f
    (x) (\mu_r (\dd x) - \nu_r (\dd x)) \right| | \langle 1, \mu_r
    \rangle | \dd r\\
    & + \int_s^t \left| \int_{\RR^{\ast}_+} x^{- \alpha} \theta (x) f
    (x) \nu_r (\dd x) \right| | \langle 1, \mu_r - \nu_r \rangle | \dd
    r\\
    & + \int_0^t \left| \int_{\RR^{\ast}_+} f (x) \mu_r (\dd x) -
    \nu_r (\dd x) \right| | \langle y^{- \alpha}, \mu_r \rangle | \dd
    r\\
    & + \int_0^t \left| \int_{\RR^{\ast}_+} f (x) \nu_r (\dd x)
    \right| \left| \int_{\RR^{\ast}_+} \theta (y) y^{- \alpha} (\mu_r -
    \nu_r) (\dd y) \right| \dd r\\
    \lesssim & (1 + \eps^{- \alpha} + K_0) K_0 \int_0^t \| \nu_r -
    \mu_r \| \dd r
  \end{align*}
  
  And we get that
  \[ \underset{t \in [0, \tau_1 (\eps)]}{\sup} \| \Gamma (\mu)_t -
     \Gamma (\nu)_t \| \lesssim \tau_1 (\eps) (1 + \eps^{-
     \alpha}) (1 + K_0)^2 \underset{t \in [0, \tau_0 (\eps)]}{\sup} \|
     \nu_t - \mu_t \| . \]
  One can then chose $C \tau_1 (\eps) (1 + \eps^{- \alpha}) (1 +
  K_0)^2 \le \frac{1}{2}$ where $C > 0$ is the constant in the previous
  equality. A choice of $\tau_1 (\eps)$ which fulfills the two
  conditions on $\tau_1 (\eps)$ is for example
  \[ \tau_1 (\eps) = \frac{C_{\alpha}}{(1 + \eps^{- \alpha} +
     \eps^{- 2})^2 (1 + 2 \langle 1 + x + x^{- \alpha}, \mu_0
     \rangle)^2} \]
  where $C_{\alpha}$ is a constant which only depends on $\alpha$. We can also
  chose $K = K_0 =  2 \langle 1 + x + x^{- \alpha}, \mu_0 \rangle$, and one
  gets the wanted result.
\end{proof}

\begin{remark}
  \label{remark:expression-of-small-times}In the previous proof, we have just
  shown local well-posedness of Smoluchowski equation with kernel $K_{\alpha}$
  on a time interval $[0, \tau_1 (\eps)]$ where
  \[ \tau_1 (\eps) = \frac{C_{\alpha}}{(1 + \eps^{- \alpha} +
     \eps^{- 2})^2 (1 + 2 \langle 1 + x + x^{- \alpha}, \mu_0
     \rangle)^2} \]
  and where the solutions stay on $\cC_{\eps, K_1}$ with
  \[ K_1 = 2 \langle 1 + x + x^{- \alpha}, \mu_0 \rangle . \]
\end{remark}

\begin{theorem}
  \label{theorem:solution-smolu-positive}Let $p > 1$, $\mu_0 \in
  \cM^{+, 1}_1$, $\eps > 0$ and $\mu_0^{\eps} =
  \mathbbm{1}_{[\eps, + \infty)} \mu_0$. A constant $K_2 = K_2 (\mu_0)
  > 0$ and a time $\tau_2 = \tau_2 (\mu_0, \eps) > 0$ exist such that
  there is a unique function $\mu^{\eps} \in C ([0, \tau_2],
  (\cC_{\eps, K_2}))$ such that $\mu^{\eps} = \Gamma
  (\mu^{\eps})$. Furthermore, $\mu^{\eps}$ is a path of
  \emph{non-negative} measures, and
  \[ \tau_2 = \tau_2 (\eps, \langle 1 + x + x^{- \alpha}, \mu_0
     \rangle) \quad  \text{ and } \quad K_2 = K_2 (\langle 1 + x + x^{- \alpha},
     \mu_0 \rangle) . \]
\end{theorem}

\begin{proof}
  First, remark that since $\mu_0 \in \cM^{+, 1}_1$, we have by
  definition
  \[ \langle 1 + x + x^{- \alpha}, \mu_0^{\eps} \rangle \le
     \langle 1 + x + x^{- \alpha}, \mu_0 \rangle < + \infty, \]
  where the previous bound does not depend on $\eps > 0$.
  
  Hence there exists such that $\mu_0 \in \cC_{\eps, \langle 1
  + x + x^{- \alpha}, \mu_0 \rangle}$. Take $K_1$ and $\tau_1 (\eps)$
  as defined in Proposition \ref{prop:contraction} (with expressions in Remark
  \ref{remark:expression-of-small-times}), then on $C ([0, \tau_1
  (\eps)] ; (\cC_{\eps, K_1}, \| \cdot \|))$, $\Gamma$
  is a contraction, and since $\cC_{\eps, K_1}$ is a closed
  subset of $(\cM, \| \cdot \|)$, $C ([0, \tau_1 (\eps)] ;
  (\cC_{\eps, K_1}, \| \cdot \|))$ is a closed subset of $C
  ([0, \tau_0 (\eps)] ; (\cM, \| \cdot \|))$ and a unique
  fixed point $\mu^{\eps} \in C ([0, \tau_1 (\eps)] ;
  (\cC_{\eps, K_1}, \| \cdot \|))$ exists.
  
  One needs to prove that $\mu^{\eps}$ is a path of non-negative
  measures, using the fact that $\mu_0$ and thus $\mu_0^{\eps}$ are
  non-negative measures. To do so, we rely on a method of proof inspired by
  {\cite{norrisSmoluchowskisCoagulationEquation1999}}. Consider the path of
  measures $\nu^{\eps}$ such that for all $t \in [0, \tau_1
  (\eps)]$, $\nu^{\eps}_t \ll \mu^{\eps}_t$ and
  \[ \frac{\dd \nu^{\eps}}{\dd \mu^{\eps}} (y) = \exp
     \left( \int_0^t \int_{\RR^{\ast}_+} K_{\alpha} (x, y)
     \mu_r^{\eps} (\dd x) \right) \]
  We have, using the fact thet $\Gamma (\mu^{\eps})_t =
  \mu^{\eps}_t$ for $t \in [0, \tau_1 (\eps)]$,
  
  \begin{align*}
    \frac{\dd}{\dd t} \langle f, \nu^{\eps}_t \rangle = &
    \int_{\RR^{\ast}_+} \int_{\RR^{\ast}_+} \frac{\dd
    \nu^{\eps}}{\dd \mu^{\eps}} (y) K_{\alpha} (x, y) f (y)
    \mu_t^{\eps} (\dd x) \mu^{\eps}_t (\dd y)\\
    & + \frac{1}{2} \int_{\RR^{\ast}_+} \int_{\RR^{\ast}_+}
    \frac{\dd \nu^{\eps}}{\dd \mu^{\eps}} (y) K_{\alpha}
    (x, y) f (x + y) \mu_t^{\eps} (\dd x) \mu^{\eps}_t
    (\dd y)\\
    & - \int_{\RR^{\ast}_+} \int_{\RR^{\ast}_+} \frac{\dd
    \nu^{\eps}}{\dd \mu^{\eps}} (y) K_{\alpha} (x, y) f (y)
    \mu_t^{\eps} (\dd x) \mu^{\eps}_t (\dd y)\\
    = & \frac{1}{2} \int_{\RR^{\ast}_+} \int_{\RR^{\ast}_+}
    F_{\alpha}^{\eps} (t, x, y) f (x + y) \nu_t^{\eps} (\dd
    x) \nu^{\eps}_t (\dd y),
  \end{align*}
  
  where
  \[ F_{\alpha}^{\eps} (t, x, y) = K_{\alpha} (x, y) \exp \left(
     \int_0^t \int_{\RR^{\ast}_+} (K_{\alpha} (z, x + y) - K_{\alpha}
     (z, x) - K_{\alpha} (z, y)) \mu_r^{\eps} (\dd z) \dd r
     \right) . \]
  
  
  Let us define for $\nu \in C ([0, \tau_1 (\eps)] ; (\cM_1^1,
  \| \cdot \|))$ $\tilde{\Gamma} (\nu)$ as follow :
  \[ \langle f, \tilde{\Gamma} (\nu)_t \rangle = \langle f,
     \mu_0^{\eps} \rangle + \frac{1}{2} \int_0^t
     \int_{\RR^{\ast}_+} \int_{\RR^{\ast}_+}
     F_{\alpha}^{\eps} (t, x, y) f (x + y) \nu_r (\dd x) \nu_r
     (\dd y) \dd r. \]
  First remark that, thanks to its expression, for any path of
  \emph{non-negative} measures $\nu \in C ([0, T], (\cM_1^{+, 1},
  \| \cdot \|))$ and since $\mu_0 \in \cM^+$ is a
  \emph{non-negative} measure, for any $t \in [0, \tau_1 (\eps)]$,
  the measure $\tilde{\Gamma} (\nu)_t \in \cM^+$. Remark also that
  $\tilde{\Gamma} (\nu^{\eps}) = \nu^{\eps}$.
  
  Using the same arguments as in the proof of Proposition
  \ref{prop:contraction}, one could get that for a suitable time $\tau_2
  (\eps) \le \tau_1 (\eps)$ and for $K_2 = 2 \langle 1 + x
  + x^{- \alpha}, \mu_0 \rangle$ independent of $\eps$,
  $\tilde{\Gamma}$ is a contraction from $C ([0, \tau_2 (\eps)],
  (\cC_{\eps, K_2}, \| \cdot \|))$ to itself. Following the
  same reasoning as in the previous proof, one could show that $\tau_2
  (\eps)$ depends only on $K_2 = 2 \langle 1 + x + x^{- \alpha}, \mu_0
  \rangle$ and $\eps$. Again, thanks to the contraction mapping
  theorem, a unique fixed point $\tilde{\nu}^{\eps}$ for
  $\tilde{\Gamma}$ in $C ([0, \tau_2 (\eps)],
  (\cC_{\eps, K_2}, \| \cdot \|))$ exists. Furthermore,
  thanks to the continuity of $\tilde{\Gamma}$,
  \[ \tilde{\nu}^{\eps} = \underset{n \rightarrow + \infty}{\lim}
     \tilde{\nu}^{\eps, n}, \]
  where $\tilde{\nu}^{\eps, n + 1} = \Gamma (\tilde{\nu}^{\eps,
  n})$ and $\tilde{\nu}^{\eps, 0}_t = \mu^{\eps}_0$ for all $t
  \in [0, \tau_2 (\eps)]$, and since $\mu_0$ and thus
  $\mu_0^{\eps}$ are non-negative measures, for all $n \ge 0$, \
  $\tilde{\nu}^{\eps, n}$ are paths of non-negative measure, and thanks
  to the properties of $\tilde{\Gamma}$, so does $\tilde{\nu}^{\eps}$.
  
  The previous computation shows that $\tilde{\Gamma} (\nu^{\eps}) =
  \nu^{\eps}$ and since $\mu_0$ (and thus $\mu_0^{\eps}$) are
  \emph{non-negative} measures, and thanks to the previous discussion,
  $\nu^{\eps} = \tilde{\nu}^{\eps}$ for $t \in [0, \tau_2
  (\eps)]$.
  
  Finally,
  \[ \frac{\dd \mu^{\eps}_t}{\dd \nu^{\eps}} (y) = \exp
     \left( - \int_0^t \int_{\RR^{\ast}_+} K_{\alpha} (x, y)
     \mu_r^{\eps} (\dd x) \right) > 0 \]
  and $\mu^{\eps}_t$ is a non-negative measure for all $t \in [0,
  \tau_2 (\eps)]$.
\end{proof}

\begin{theorem}
  Let $p \ge 2$ and suppose that $\mu_0 \in \cM^{+, p}_1$ and
  define $\mu^{\eps}_0 = \mathbbm{1}_{[\eps, + \infty)} \mu_0$.
  For any $T > 0$, there exists a unique path of non-negative measures
  $\mu^{\eps}$ solution of the Smoluchowski equation
  \[ \mu^{\eps}_t = \mu^{\eps}_0 + \int_0^t L_{\alpha}
     (\mu^{\eps}_r) \dd r. \]
  Furthermore $\mu^{\eps} \in  \text{ Lip } ([0, T] ; (\cM^{+, p,
  \langle 1 + x + x^{- \alpha p}, \mu_0 \rangle}_1, W_1))$ and
  \[ \llbracket \mu^{\eps} \rrbracket_{ \text{ Lip } } \le c_{\alpha,
     p} \langle 1 + x + x^{- \alpha p}, \mu_0 \rangle^2, \]
  for a constant $c_{\alpha, p} > 0$ independent of $\mu_0$ and $\eps$.
  
\end{theorem}

\begin{proof}
  Thanks to the previous theorem, there exists $\tau_2 = \tau_2 (\eps,
  \langle 1 + x + x^{- \alpha}, \mu_0 \rangle)$ and $K_2 = K_2 (\langle 1 + x
  + x^{- \alpha}, \mu_0 \rangle)$ such that $\mu^{\eps}$ is the unique
  non-negative solution of the previous equation in $C ([0, \tau_2] ;
  (\cC_{\eps, K_2}, \| \cdot \|))$. Remark first that we have
  for all $t \le \tau_2$,
  \[ \langle 1, \mu^{\eps}_t \rangle = \langle x, \mu^{\eps}_0
     \rangle \le \langle x, \mu_0 \rangle . \]
  Furthermore, one gets
  \begin{equation}
    \langle 1, \mu^{\eps}_t \rangle = \langle 1, \mu^{\eps}_0
    \rangle - \frac{1}{2} \int_0^t \langle 1, \mu^{\eps}_r \rangle
    \langle x^{- \alpha}, \mu^{\eps}_r \rangle \dd r \le
    \langle 1, \mu^{\eps}_0 \rangle \le \langle 1, \mu_0 \rangle
    . \label{eq:invariance-mueps1}
  \end{equation}
  Since for all $p \ge 1$,
  \[ \frac{1}{x^{p \alpha}} + \frac{1}{y^{p \alpha}} \ge 2^{\alpha p +
     1} \frac{1}{(x + y)^{\alpha p}}, \]
  one gets that
  
  \begin{align*}
    \frac{1}{x^{\alpha p}} + \frac{1}{y^{\alpha p}} - \frac{1}{(x + y)^{\alpha
    p}} = & (1 - 2^{- (\alpha p + 1)}) \left( \frac{1}{x^{\alpha p}} +
    \frac{1}{y^{\alpha p}} \right) + 2^{- (\alpha p + 1)} \left(
    \frac{1}{x^{\alpha p}} + \frac{1}{y^{\alpha p}} - \frac{2^{\alpha p +
    1}}{(x + y)^{\alpha p}} \right)\\
    \ge & (1 - 2^{- (\alpha p + 1)}) \left( \frac{1}{x^{\alpha p}} +
    \frac{1}{y^{\alpha p}} \right) .
  \end{align*}
  
  Hence
  
  \begin{align*}
    \langle x^{- \alpha p}, \mu^{\eps}_t \rangle = & \langle x^{-
    \alpha p}, \mu^{\eps}_0 \rangle - \frac{1}{2} \int_0^t \left(
    \frac{1}{x^{\alpha}} + \frac{1}{y^{\alpha}} \right) \left(
    \frac{1}{x^{\alpha p}} + \frac{1}{y^{\alpha p}} - \frac{1}{(x + y)^{\alpha
    p}} \right) \mu^{\eps}_r (\dd x) \mu^{\eps}_r (\dd
    y)\\
    \le & \langle x^{- \alpha p}, \mu^{\eps}_0 \rangle - \frac{(1
    - 2^{- (\alpha p + 1)})}{2} \int_0^t \left( \frac{1}{x^{\alpha}} +
    \frac{1}{y^{\alpha}} \right) \left( \frac{1}{x^{\alpha p}} +
    \frac{1}{y^{\alpha p}} \right) \mu^{\eps}_r (\dd x)
    \mu^{\eps}_r (\dd y)\\
    \le & \langle x^{- \alpha p}, \mu^{\eps}_0 \rangle - (1 -
    2^{- (\alpha p + 1)}) \int_0^t \langle x^{- \alpha (p + 1)},
    \mu_r^{\eps} \rangle \langle 1, \mu_r^{\eps} \rangle +
    \langle x^{- \alpha p}, \mu_r^{\eps} \rangle \langle x^{- \alpha},
    \mu_r^{\eps} \rangle \dd r\\
    \le & \langle x^{- \alpha p}, \mu_0 \rangle .
  \end{align*}
  
  Finally, we get
  \[ \langle 1 + x + x^{- \alpha}, \mu^{\eps}_{\tau_2} \rangle
     \le \langle 1 + x + x^{- \alpha}, \mu_0 \rangle . \]
  One can then start again the dynamic on $[\tau_2, T]$ with initial condition
  $\mu^{\eps}_{\tau_2}$, which fulfills the same conditions as $\mu_0$.
  The previous inequality and the previous Theorem
  \ref{theorem:solution-smolu-positive} shows that one can uniquely define
  $\mu^{\eps}$ on $[\tau_2, 2 \tau_2]$ as a path a non-negative
  measures. By a direct induction, one can uniquely define $\mu^{\eps}$
  on $[0, n \tau_2]$ for all $n \ge 0$, and thus on any interval $[0,
  T]$.
  
  Now, one gets for all $s \le t \in [0, T]$,
  
  \begin{align*}
    | \langle 1, \mu^{\eps}_t \rangle - \langle 1, \mu^{\eps}_s
    \rangle | = & \frac{1}{2} \int_s^t \langle 1, \mu_r^{\eps} \rangle
    \langle x^{- \alpha}, \mu_r^{\eps} \rangle \dd r\\
    \le & \frac{1}{2} \langle 1, \mu_0^{\eps} \rangle \langle
    x^{- \alpha}, \mu_0^{\eps} \rangle (t - s) .
  \end{align*}
  
  Furthermore, for $f \in  \text{ Lip } ((\RR_+^{\ast}, d_{\alpha}) ;
  \RR)$ with $f (1) = 0$ and $\llbracket f \rrbracket \le 1$, we
  have
  \[ | f (x + y) - f (x) - f (y) | \le \left| \frac{1}{(x + y)^{\alpha}}
     - 1 \right| + \left| \frac{1}{x^{\alpha}} - 1 \right| + \left|
     \frac{1}{y^{\alpha}} - 1 \right| \le 3 \left( \frac{1}{x^{\alpha}}
     + \frac{1}{y^{\alpha}} \right) \]
  and
  \[ | \langle f, \mu^{\eps}_t \rangle - \langle f, \mu^{\eps}_s
     \rangle | \le \frac{3}{2} \int_s^t \left( \frac{1}{x^{\alpha}} +
     \frac{1}{y^{\alpha}} \right)^2 \mu^{\eps}_r (\dd x)
     \mu^{\eps}_r (\dd y) \le 3 \int_s^t \langle 1,
     \mu_r^{\eps} \rangle \langle x^{- 2 \alpha}, \mu_r^{\eps}
     \rangle \dd r \]
  and
  \[ W_1 (\mu^{\eps}_t, \mu^{\eps}_s) = \sup \langle f,
     \mu^{\eps}_t \rangle - \langle f, \mu^{\eps}_s \rangle + |
     \langle 1, \mu^{\eps}_t \rangle - \langle 1, \mu^{\eps}_s
     \rangle | \le 3 \langle 1, \mu_0 \rangle (\langle x^{- \alpha},
     \mu_0 \rangle + \langle x^{- 2 \alpha}, \mu_0 \rangle) | t - s | . \]
  Since $\mu_0 \in \cM_1^{+, p}$ for $p > 2$, a constant $c_{\alpha,
  p} > 0$ exists such that $\langle x^{- \alpha} + x^{- 2 \alpha}, \mu_0
  \rangle \le \langle x^{- \alpha p} + x + 1, \mu_0 \rangle$, and we
  have
  \[ W_1 (\mu^{\eps}_t, \mu^{\eps}_s) \le c_{\alpha, p} |
     t - s | \langle 1 + x + x^{- \alpha p}, \mu_0 \rangle^2, \]
  which is the wanted result.
\end{proof}

Using the same method of proof as Theorem
\ref{theorem:solution-smolu-positive}, one can prove the following
proposition, which will prove useful in the following :

\begin{proposition}
  \label{prop:positivity-smoluchowski}Suppose for $p > 1$ that $\nu_0
  \le \mu_0 \in \cM_1^{+, p}$, which means that for any Borel
  set $A \subset \RR_+^{\ast}$, $\nu_0 (A) \le \mu_0 (A)$. Then
  for all $t \in [0, T]$, $\nu_t \le \mu_t$, where $(\nu_t)_t$
  (respectively $(\mu_t)_t$) is the solution of the Smoluchowski Equation with
  kernel $K_{\alpha}$ and initial value $\nu^{\eps}_0$ (respectively
  $\mu_0^{\eps}$) as constructed in Theorem
  \ref{theorem:solution-smolu-positive}.
\end{proposition}

\begin{proof}
  Define
  \[ \theta^{\eps}_t (x) = \exp \left( \int_0^t
     \int_{\RR_+^{\ast}} K_{\alpha} (x, y) (\mu^{\eps}_r (\dd
     y) + \nu^{\eps}_r (\dd y)) \dd r \right) \]
  and consider
  \[ h_t^{\eps} = \theta^{\eps}_t (x) (\mu^{\eps}_t -
     \nu^{\eps}_t) . \]
  Remark that
  \[ \frac{\dd}{\dd t} \langle f, \mu^{\eps}_t -
     \nu^{\eps}_t \rangle = \frac{1}{2} \int_{\RR^{\ast}_+}
     \int_{\RR^{\ast}_+} K (x, y) (f (x + y) - f (x) - f (y))
     (\mu^{\eps}_t + \nu^{\eps}_t) (\dd y)
     (\mu^{\eps}_t - \nu^{\eps}_t) (\dd x) . \]
  We have
  
  \begin{align*}
    \frac{\dd}{\dd t} \langle f, h_t^{\eps} \rangle = &
    \int_{\RR_+^{\ast}} \int_{\RR_+^{\ast}} K_{\alpha} (x, y)
    (\mu^{\eps}_t + \nu^{\eps}_t) (\dd y)
    \theta^{\eps}_t (x) f (x) (\mu^{\eps}_t -
    \nu^{\eps}_t) (\dd x)\\
    & + \frac{1}{2} \int_{\RR_+^{\ast}} \int_{\RR_+^{\ast}}
    K_{\alpha} (x, y) \theta^{\eps}_t (x + y) f (x + y)
    (\mu^{\eps}_t + \nu^{\eps}_t) (\dd y)
    (\mu^{\eps}_t - \nu^{\eps}_t) (\dd x)\\
    & - \int_{\RR_+^{\ast}} \int_{\RR_+^{\ast}} K_{\alpha} (x,
    y) \theta^{\eps}_t (x) f (x) (\mu^{\eps}_t +
    \nu^{\eps}_t) (\dd y) (\mu^{\eps}_t -
    \nu^{\eps}_t) (\dd x)\\
    = & \frac{1}{2} \int_{\RR_+^{\ast}} \int_{\RR_+^{\ast}}
    K_{\alpha} (x, y) \frac{\theta^{\eps}_t (x +
    y)}{\theta^{\eps}_t (x)} f (x + y) (\mu^{\eps}_t +
    \nu^{\eps}_t) (\dd y) (\mu^{\eps}_t -
    \nu^{\eps}_t) (\dd x) .
  \end{align*}
  
  Hence, we have $H (h^{\eps}) = h^{\eps}$ where for any
  suitable path of non-negative measures $(h_t)_t$, $H$ is defined for
  suitable $f$ by
  
  \begin{multline*}
    \langle f, H (h)_t \rangle = \langle f, \mu^{\eps}_0 -
    \nu^{\eps}_0 \rangle + \int_0^t \int_{\RR^{\ast}_+}
    \int_{\RR^{\ast}_+} K_{\alpha} (x, y) \frac{\theta^{\eps}_t
    (x + y)}{\theta^{\eps}_t (x)} f (x + y) (\mu^{\eps}_t +
    \nu^{\eps}_t) (\dd y) h_r (\dd x) \dd r.
  \end{multline*}
  
  Since $\mu^{\eps}_0 - \nu^{\eps}_0$ is a non-negative measure,
  so does $H (h)_t$ for any $t \in [0, T]$. One can then argue the the
  previous equation has a unique solution (using the same method as in the
  proof of Theorem \ref{theorem:solution-smolu-positive}), and
  $h^{\eps}$ is a path of non-negative measure, and so does
  $\mu^{\eps} - \nu^{\eps}$.
\end{proof}

\subsection{Regularity of the approximate solution with respect to the initial
condition}

\begin{theorem}
  \label{theorem:flow-lipschitz}Let $T > 0$. Let $K > 0$, \red{$p
  \ge 2$}, $\eps > 0$ and $\mu_0, \nu_0 \in \cM_1^{+, p,
  K}$. Let $\mu^{\eps}, \nu^{\eps} \in  \text{ Lip } ([0, T] ;
  \cM_1^{+, p, K})$ be the solution of the Smoluchowski equation with
  initial condition $\mu^{\eps}_0 = \mathbbm{1}_{[\eps, +
  \infty)} \mu_0$ (respectively $\nu^{\eps}_0 =
  \mathbbm{1}_{[\eps, + \infty)} \nu_0$). Then
  \[ \underset{t \in [0, T]}{\sup} W_1 (\mu^{\eps}_t,
     \nu^{\eps}_t) \le W_1 (\mu^{\eps}_0,
     \nu^{\eps}_0) e^{c K (1 + \eps^{- 1}) T} . \]
\end{theorem}

\begin{proof}
  First, remark that
  
  \begin{align*}
    \langle 1, \mu^{\eps}_t - \nu^{\eps}_t \rangle = & \langle
    1, \mu^{\eps}_0 - \nu^{\eps}_0 \rangle - \frac{1}{2}
    \int_0^t \langle 1, \mu^{\eps}_r - \nu^{\eps}_r \rangle
    \langle x^{- \alpha}, \mu^{\eps}_r \rangle \dd r\\
    & - \frac{1}{2} \int_0^t \langle 1, \nu^{\eps}_r \rangle \langle
    x^{- \alpha}, \mu^{\eps}_r - \nu^{\eps}_r \rangle \dd r
  \end{align*}
  
  Remark that since $x \rightarrow x^{- \alpha}$ is Lipschitz continuous from
  $(\RR_+^{\ast}, d_{\alpha})$ to $\RR$, and using the fact from
  the previous Theorem that $\sup_t \langle 1 + x^{- \alpha} + x,
  \mu^{\eps}_t \rangle \le c_{p, \alpha} K$, we get that a
  constant $c > 0$ exists, which depends on $K, \alpha, p$ such that
  \[ | \langle 1, \mu^{\eps}_t - \nu^{\eps}_t \rangle |
     \le | \langle 1, \mu^{\eps}_0 - \nu^{\eps}_0 \rangle
     | + c \left( \int_0^t | \langle 1, \mu^{\eps}_r -
     \nu^{\eps}_r \rangle | \dd r + W_1 (\mu_r^{\eps},
     \nu_r^{\eps}) \dd r \right) . \]
  Take $f \in  \text{ Lip } ((\RR_+^{\ast}, d_{\alpha}) ; \RR)$
  with $\llbracket f \rrbracket_{ \text{ Lip } } \le 1$. Let us remind that
  we have
  
  \begin{align*}
    \langle f, \mu^{\eps}_t \rangle = & \langle f, \mu^{\eps}_0
    \rangle + \int_0^t \int_{\RR^{\ast}_+} \int_{\RR^{\ast}_+}
    x^{- \alpha} f (x + y) \mu_r^{\eps} (\dd x) \mu_r^{\eps}
    (\dd y) \dd r\\
    & - \int_0^t \langle x^{- \alpha}, \mu^{\eps}_r \rangle \langle f,
    \mu_r^{\eps} \rangle \dd r\\
    & - \int_0^t \int_{\RR^{\ast}_+} x^{- \alpha} f (x)
    \mu_r^{\eps} (\dd x) \langle 1, \mu_r^{\eps} \rangle
    \dd r
  \end{align*}
  
  and
  
  \begin{align*}
    \langle f, \mu^{\eps}_t - \nu^{\eps}_t \rangle = & \langle
    f, \mu^{\eps}_0 - \nu_0^{\eps} \rangle +\\
    & \int_0^t \int_{\RR^{\ast}_+} \int_{\RR^{\ast}_+} x^{-
    \alpha} f (x + y) (\mu_r^{\eps} - \nu^{\eps}_r) (\dd x)
    \mu_r^{\eps} (\dd y) \dd r\\
    & + \int_0^t \int_{\RR^{\ast}_+} \int_{\RR^{\ast}_+} x^{-
    \alpha} f (x + y) \nu_r^{\eps} (\dd x) (\mu_r^{\eps} -
    \nu^{\eps}_r) (\dd y) \dd r\\
    & - \int_0^t \langle x^{- \alpha}, \mu^{\eps}_r -
    \nu^{\eps}_r \rangle \langle f, \mu_r^{\eps} \rangle \dd
    r - \int_0^t \langle x^{- \alpha}, \nu^{\eps}_r \rangle \langle f,
    \mu_r^{\eps} - \nu^{\eps}_r \rangle \dd r\\
    & - \int_0^t \int_{\RR^{\ast}_+} x^{- \alpha} f (x)
    (\mu_r^{\eps} - \nu^{\eps}_r) (\dd x) \langle 1,
    \mu_r^{\eps} \rangle \dd r\\
    & - \int_0^t \int_{\RR^{\ast}_+} x^{- \alpha} f (x)
    \nu_r^{\eps} (\dd x) \langle 1, \mu_r^{\eps} -
    \nu^{\eps}_r \rangle \dd r.
  \end{align*}
  
  Note that $y \rightarrow \int_{\RR^{\ast}_+} x^{- \alpha} f (x + y)
  \nu_r^{\eps} (\dd x)$ is Lipschitz continuous and that
  \[ \left\llbracket \int_{\RR^{\ast}_+} x^{- \alpha} f (x + \cdot)
     \nu_r^{\eps} (\dd x) \right\rrbracket_{ \text{ Lip } } \le
     \langle x^{- \alpha}, \nu^{\eps}_r \rangle \le K. \]
  Remark also that since $\mu^{\eps}_r$ and $\nu^{\eps}_r$ have
  support included in $[\eps, + \infty)$, If \ (from Proposition
  \ref{prop:contraction} and Theorem \ref{theorem:solution-smolu-positive}),
  for all $y \in \RR_+^{\ast}$
  \[ x \rightarrow x^{- \alpha} f (x) \theta (x) \quad  \text{ and } \quad x
     \rightarrow x^{- \alpha} f (x + y) \theta (x) \theta (y) \]
  are Lipschitz continuous on $[\eps, + \infty)$, with their Lipschitz
  constants bounded by
  \[ c (1 + \eps^{- \alpha}) \]
  for some constant $c > 0$. Furthermore, again thanks to the support of the
  measures, for $0 < \eps < 1$
  \[ W_1 (\mu^{\eps}_r, \nu^{\eps}_r) =
     \underset{\begin{array}{c}
       f : \RR_+^{\ast} \rightarrow \RR  \text{ measurable } \\
       f_{| [\eps, + \infty)} \in  \text{ Lip } (([\eps, + \infty),
       d_{\alpha}) ; \RR)\\
       \llbracket f \rrbracket_{ \text{ Lip } } \le 1, f (1) = 1
     \end{array}}{\sup} \langle f, \mu^{\eps}_r - \nu^{\eps}_r
     \rangle + | \langle 1, \mu^{\eps}_r - \nu^{\eps}_r \rangle
     | . \]
  The previous discussion leads to
  
  \begin{align*}
    \langle f, \mu^{\eps}_t - \nu^{\eps}_t \rangle \le &
    \langle f, \mu^{\eps}_0 - \nu_0^{\eps} \rangle + c K (1 +
    \eps^{- \alpha}) \int_0^t W_1 (\mu^{\eps}_r,
    \nu^{\eps}_r) \dd r
  \end{align*}
  
  and putting alltogether, we get
  \[ W_1 (\mu_t^{\eps}, \nu_t^{\eps}) \le W_1
     (\mu_t^{\eps}, \nu_t^{\eps}) + c K (1 + \eps^{-
     \alpha}) \int_0^t W_1 (\mu^{\eps}_r, \nu^{\eps}_r) \dd
     r. \]
  Gr{\"o}nwall lemma allows us to conclude. 
\end{proof}

\begin{theorem}
  Let $z \in \RR_+^{\ast}$, \red{$p \ge 2$}, $K > 0$,
  $\eps > 0$, $\mu_0 \in \cM^{+, p, K}_1$, $\mu_0^{\eps}
  = \mathbbm{1}_{[\eps, + \infty)} \mu_0$ and $\mu^{\eps} \in
   \text{ Lip } ([0, T] ; (\cM^{+, p, K}_1, W_1))$ be the unique solution
  of the Smoluchowski equation.
  
  Let $z > \eps$, we are looking to the solution $A^{\eps} (z)
  \in C ([0, T] ; \red{(\cM_1, W_1)})$ define for all bounded
  measurable functions as
  \[ \langle f, A^{\eps}_r (z) \rangle = f (z) + \int_0^t
     \int_{\RR_+^{\ast}} \left( \int_{\RR_+^{\ast}} K_{\alpha}
     (x, y) (f (x + y) - f (x) - f (y)) \mu^{\eps}_r (\dd x) \right)
     A^{\eps}_r (z) (\dd y) \dd r \]
  \red{There is a unique solution to the previous equation.
  Furthermore, this solution is Lipschitz continuous in time, regular in $z$
  (uniformely de preference)}
\end{theorem}

\begin{proof}
  Take $f \in C_0 ((\RR_+^{\ast}, d_{\alpha}) ; \RR)$. Since
  $\mu^{\eps}$ has support included on $[\eps, + \infty)$, as in
  the proof of Proposition \ref{prop:contraction}, take $0 \le
  \theta^{\eps} \le 1$ a continuous function such that
  $\theta^{\eps} (y) = 0$ for $y < \frac{\eps}{2}$ and $\theta
  (y) = 1$ for $y \ge \eps$. For all $t \in [0, T]$ and all $y
  \in \RR_+^{\ast}$, and for any $(A_t)_t \in C ([0, T] ;
  (\cC_{\eps, K}, \| \cdot \|))$, let us define
  
  \begin{align*}
    \langle f, \Lambda (A)_t \rangle = & f (z) + \int_0^t
    \int_{\RR_+^{\ast}} \left( \int_{\RR_+^{\ast}} K_{\alpha}
    (x, y) (f (x + y) - f (x) - f (y)) \mu^{\eps}_r (\dd x) \right)
    A_r (z) (\dd y) \dd r\\
    = & f (z) + \int_0^t \int_{\RR_+^{\ast}} \int_{\RR_+^{\ast}}
    K_{\alpha} (x, y) f (x + y) \mu^{\eps}_r (\dd x) A_r (z) (\dd
    y) \dd r\\
    & - \int_0^t \int_{\RR_+^{\ast}} \int_{\RR_+^{\ast}}
    K_{\alpha} (x, y) f (y) \mu^{\eps}_r (\dd x) A_r (z) (\dd y)
    \dd r\\
    & - \int_0^t \langle 1, A_r (z) \rangle \langle x^{- \alpha} f,
    \mu^{\eps}_r \rangle \dd r - \int_0^t \langle y^{- \alpha}, A_r
    (z) \rangle \langle f, \mu^{\eps}_r \rangle \dd r.\\
    = & f (z) + \int_0^t \left( \theta (y) \int_{\RR_+^{\ast}}
    K_{\alpha} (x, y) f (x + y) \mu^{\eps}_r (\dd x) \right) A_r (z)
    (\dd y) \dd r\\
    & - \int_0^t \int_{\RR_+^{\ast}} \theta (y) \left(
    \int_{\RR_+^{\ast}} K_{\alpha} (x, y) f (y) \mu^{\eps}_r
    (\dd x) \right) A_r (z) (\dd y) \dd r\\
    & - \int_0^t \langle 1, A_r (z) \rangle \langle x^{- \alpha} f,
    \mu^{\eps}_r \rangle \dd r - \int_0^t \int_{\RR_+^{\ast}}
    y^{- \alpha} \theta (y) A_r (z) (\dd y) \langle x^{- \alpha} f,
    \mu^{\eps}_r \rangle \dd r.
  \end{align*}
  
  Here we have crucially use the fact that $ \text{ supp } A_r (z) \subset
  [\eps, + \infty)$. Using the same bounds as Proposition
  \ref{prop:contraction}, one gets that $\Lambda$ is a contraction for on $C
  ([0, \tau], \cC_{\eps, K}, \| \cdot \|)$ for suitable
  $\tau$ and $K$. Indeed, $\Lambda (A)_0 = \delta_z$ and $z \ge
  \eps$ and one gets that $\Lambda$ as a unique fixed point, as in
  Theorem \ref{theorem:solution-smolu-positive}. Furthermore, since $\Lambda$
  is linear, the solution exists for all time. Now, let $\eta > 0$ and let
  $(\tilde{\mu}^{\eps}_t)_t$ be the unique solutions of the
  smoluchowski equation with initial condition respectively
  $\tilde{\mu}^{\eps}_0 = (\mu_0 + \eta \delta_z)^{\eps} =
  \mu_0^{\eps} + \eta \delta_z$ and $(\mu^{\eps}_t)_t$ with
  initial condition $\mu^{\eps}_0$. We have for $f \in C_0
  ((\RR_+^{\ast} d_{\alpha}) ; \RR)$ with $\| f \|_{\infty}
  \le 1$,
  \[ \frac{\dd}{\dd t} \langle f, \tilde{\mu}^{\eps}_t -
     \mu^{\eps}_t - \eta A^{\eps}_t (z) \rangle = \]
  \begin{align*}
    & \frac{1}{2} \int_{\RR_+^{\ast}} \int_{\RR_+^{\ast}}
    K_{\alpha} (x, y) (f (x + y) - f (x) - f (y)) (\tilde{\mu}^{\eps}_t
    + \mu^{\eps}_t) \otimes (\tilde{\mu}^{\eps}_t -
    \mu^{\eps}_t) (\dd x, \dd y)\\
    & - \frac{\eta}{2} \int_{\RR_+^{\ast}} \int_{\RR_+^{\ast}}
    K_{\alpha} (x, y) (f (x + y) - f (x) - f (y)) (\tilde{\mu}^{\eps}_t
    + \mu^{\eps}_t) \otimes (\eta A_t^{\eps} (z)) (\dd x,
    \dd y)\\
    & - \frac{1}{2} \eta \int_{\RR_+^{\ast}}
    \int_{\RR_+^{\ast}} K_{\alpha} (x, y) (f (x + y) - f (x) - f (y))
    (\tilde{\mu}^{\eps}_t - \mu^{\eps}_t) \otimes
    A_t^{\eps} (z) (\dd x, \dd y)\\
    = & \frac{1}{2} \int_{\RR_+^{\ast}} \int_{\RR_+^{\ast}}
    K_{\alpha} (x, y) (f (x + y) - f (x) - f (y)) (\tilde{\mu}^{\eps}_t
    + \mu^{\eps}_t) \otimes (\tilde{\mu}^{\eps}_t -
    \mu^{\eps}_t - \eta A^{\eps}_t (z)) (\dd x, \dd y)\\
    & - \frac{1}{2} \eta \int_{\RR_+^{\ast}}
    \int_{\RR_+^{\ast}} K_{\alpha} (x, y) (f (x + y) - f (x) - f (y))
    (\tilde{\mu}^{\eps}_t - \mu^{\eps}_t) \otimes
    A_t^{\eps} (z) (\dd x, \dd y) .
  \end{align*}
  
  Thanks to the previous Theorem \ref{theorem:flow-lipschitz} and the previous
  computations, one has
  \[ \| \tilde{\mu}^{\eps}_t - \mu^{\eps}_t \| \le
     c_{\eps} \eta, \quad \| A_t^{\eps} (z) \| \le
     c_{\eps} \]
  and the last line is bounded by $c_{\eps} \eta^2$, where
  $c_{\eps}$ is a constant which depends on $\eps$, but not on
  $z \ge \eps$ neither on $\eta$ and $\mu_0$. The previous
  computations gave, by integrating them,
  \[ \| \tilde{\mu}^{\eps}_t - \mu^{\eps}_t - \eta
     A^{\eps}_t (z) \| \le c_{\eps} \left( \int_0^t \|
     \tilde{\mu}^{\eps}_r - \mu^{\eps}_r - A^{\eps}_r (z)
     \| \dd r + \eta^2 \right), \]
  and
  \[ \underset{t \in [0, T]}{\sup} \| \tilde{\mu}^{\eps}_t -
     \mu^{\eps}_t - \eta A^{\eps}_t (z) \| \le C
     (\eps, T) \eta^2 . \]
  This gives that, in $C ([0, T] ; (\cM, \| \cdot \|))$,
  \[ \underset{\eta \rightarrow 0}{\lim} \frac{\tilde{\mu}^{\eps}_t -
     \mu^{\eps}_t}{\eta} = A^{\eps}_t (z) . \]
  Notice that $(\mu_0 + \eta \delta_z)^{\eps} = \mu_0^{\eps} +
  \eta \delta_z \ge \mu_0^{\eps}$ in the sense of Proposition
  \ref{prop:positivity-smoluchowski}. This leads, using this Proposition
  \ref{prop:positivity-smoluchowski} to the fact that for any $t \in [0, T]$,
  $\frac{\tilde{\mu}^{\eps}_t - \mu^{\eps}_t}{\eta}$ is a
  non-negative measure, and so does $A^{\eps}_t (z)$.
  
  Remark then that
  \[ \langle x, A^{\eps}_t (z) \rangle = z, \]
  and using Lemma \ref{lemma:jensen-Kalpha} for $A^{\eps}_t (z)$ with
  $p = 1$ and $q = 0$, one gets
  
  \begin{align*}
    \frac{\dd}{\dd t} \langle 1, A_t (z) \rangle = & - \langle x^{-
    \alpha}, \mu^{\eps}_t \rangle \langle 1, A_t (z) \rangle - \langle
    1, \mu^{\eps}_t \rangle \langle y^{- \alpha}, A_t (z) \rangle .\\
    \le & - \langle x^{- \alpha}, \mu^{\eps}_t \rangle \langle 1,
    A_t (z) \rangle - z^{- \alpha} \langle 1, \mu^{\eps}_t \rangle
    \langle 1, A_t (z) \rangle^{1 + \alpha} .
  \end{align*}
  
  and
  \[ \langle 1, A_t (z) \rangle \le 1 \]
  and the same holds for any $p \ge 1$
  \[ \langle y^{- \alpha p}, A_t (z) \rangle \le z^{- \alpha p} . \]
  
\end{proof}

\red{f}Petite remarque ici : Il suffit de prendre un $f$ smooth
(\red{Lipschitz ?)} avec un support compact (\red{Ou pas
?)}, et de consid{\'e}rer
\[ \langle f, A^{\eps}_r (\lambda (\mu^{\eps}_0 -
   \nu^{\eps}_0) + \nu^{\eps}_0) (z) \rangle \]
cela devrait avoir un support compact et {\^e}tre smooth
(\red{Lipschitz sans support compact ?)} (pas trop difficile
normalement \red{Potentiellement difficile mais j'y crois}) en $z$,
et ce uniform{\'e}ment en $\eps$. Un coup de Arzela-Ascoli (Attention
aux d{\'e}pendances normalement g{\'e}rables en $\lambda$ et en $t$) donne
qu'il existe $F (t, \lambda, \mu_0, \nu_0)$ smooth {\`a}\quad support compact
(\red{Lipschitz ?}) tel que
\[ \langle f, \mu_t - \nu_t \rangle = \int_0^1 F (t, \lambda, \mu_0, \nu_0)
   (z) (\mu_0 - \nu_0) (\dd z) \dd \lambda . \]
On appelle alors $\varphi (\mu_0)$ cette solution, pas forc{\'e}ment unique,
et on va s'en servir pour comparer des choses.

\begin{align*}
  \langle f, \varphi () - \nu_t \rangle & 
\end{align*}

Dans ce cas, on regard
\[ f (t, \mu_0) = \langle f, \mu_{T - t} \rangle, \]
and we have
\[ f (t, \mu_0 + \eta \delta_z) - f (t, \mu_0) = \eta \int_0^1 F (T - t,
   \lambda, \mu_0 + \eta \delta_z, \mu_0) (z) \dd \lambda \]
et
\[ \langle f, A_{T - t} (z) \rangle = ? \delta_{\mu} f (t, \mu_0) (z) =
   \int_0^1 F (T - t, \lambda, \mu_0, \mu_0) (z) \dd \lambda . \]
On voudrait

\begin{multline*}
  0 = \partial_t f (t, \mu_0) +\\
  \frac{1}{2} \int_{\RR_+^{\ast}} \int_{\RR_+^{\ast}} K_{\alpha}
  (x, y) (\delta_{\mu} f (t, \mu_0) (y + x) - \delta_{\mu} f (t, \mu_0) (x) -
  \delta_{\mu} f (t, \mu_0) (y)) \mu_0 \otimes \mu_0 (\dd x, \dd y)
\end{multline*}

Remark also that
\[ \partial_t f (t, \mu_0) = - \langle f, L_{\alpha} (\mu_{T - t}) \rangle \]
On a
\[ \langle f, \varphi () - \rangle = \int_0^1 F (t, \lambda, \mu_0, \nu_0)
   (\mu_0 - \nu_0) (\dd z) \]
Cela suffit {\`a} l'unicit{\'e}, parce que on a en consid{\'e}rant $F (T - t,
\lambda, \mu_0, \nu_0)$
\[ \langle f, \mu_{T - t} - \nu_{T - t} \rangle = \int_0^1 F (T - t, \lambda,
   \mu_0, \nu_0) (\mu_0 - \nu_0) (\dd z) \dd \lambda, \]
en prenant $t = 0$, on a alors
\[ \langle f, \mu_T - \nu_T \rangle = \int_0^1 F (T, \lambda, \mu_0, \nu_0)
   (\mu_0 - \nu_0) (\dd z) \dd \lambda . \]
Et dans ce cas si $\nu_0 = \mu_0$,
\[ \langle f, \mu_T - \nu_T \rangle = 0 \]
et
\[ \mu_T = \nu_T . \]


\begin{theorem}
  \red{Let $z \in \RR_+^{\ast}$, \red{$p \ge
  2$}, $K > 0$, $\eps > 0$, $\mu_0, \nu_0 \in \cM^{+, p, K}_1$,
  $\mu_0^{\eps} = \mathbbm{1}_{[\eps, + \infty)} \mu_0$,
  $\nu_0^{\eps} = \mathbbm{1}_{[\eps, + \infty)} \nu_0$ and
  $\mu^{\eps}, \nu^{\eps} \in  \text{ Lip } ([0, T] ;
  (\cM^{+, p, K}_1, W_1))$ be the unique solution of Smoluchowski
  equations with initial conditions $\mu^{\eps}_0$ and
  $\nu^{\eps}_0$. Then for all $t \in [0, T]$,
  \[ \langle f, \mu^{\eps}_t - \nu^{\eps}_t \rangle = \int_0^1
     \langle f, A^{\eps}_r (\lambda (\mu^{\eps}_0 -
     \nu^{\eps}_0) + \nu^{\eps}_0) (z) \rangle
     (\mu^{\eps}_0 - \nu^{\eps}_0) (\dd z) . \]}
\end{theorem}

\begin{theorem}
  \
  
  \begin{multline*}
    \langle f, B^{\eps}_r (z_1, z_2) \rangle =\\
    \int_0^t \int_{\RR_+^{\ast}} \left( \int_{\RR_+^{\ast}}
    K_{\alpha} (x, y) (f (x + y) - f (x) - f (y)) A^{\eps}_r (z_2)
    (\dd x) \right) A^{\eps}_r (z_1) (\dd y) \dd r\\
    + \int_0^t \int_{\RR_+^{\ast}} \left( \int_{\RR_+^{\ast}}
    K_{\alpha} (x, y) (f (x + y) - f (x) - f (y)) \mu^{\eps}_r (z_2)
    (\dd x) \right) B^{\eps}_r (z_1, z_2) (\dd y) \dd r.
  \end{multline*}
\end{theorem}

\begin{align*}
  \langle f, A_t^{\eps} (z) \rangle = & f (z) +\\
  & \int_0^t \int_{\RR^{\ast}_+} \int_{\RR^{\ast}_+} x^{-
  \alpha} f (x + y) A_r^{\eps} (z) (\dd x) \mu_r^{\eps}
  (\dd y) \dd r\\
  & + \int_0^t \int_{\RR^{\ast}_+} \int_{\RR^{\ast}_+} x^{-
  \alpha} f (x + y) \mu_r^{\eps} (\dd x) A^{\eps}_r (z)
  (\dd y) \dd r\\
  & - \int_0^t \langle x^{- \alpha}, A^{\eps}_r (z) \rangle \langle f,
  \mu_r^{\eps} \rangle \dd r\\
  & - \int_0^t \langle x^{- \alpha}, \mu^{\eps}_r \rangle \langle f,
  A^{\eps}_r (z) \rangle \dd r\\
  & - \int_0^t \int_{\RR^{\ast}_+} x^{- \alpha} f (x)
  A^{\eps}_r (z) (\dd x) \langle 1, \mu_r^{\eps} \rangle
  \dd r\\
  & - \int_0^t \int_{\RR^{\ast}_+} x^{- \alpha} f (x)
  \mu_r^{\eps} (\dd x) \langle 1, A^{\eps}_r (z) \rangle
  \dd r.
\end{align*}

\

\red{Precise bounds on the moments of $\mu^{\eps}$

\begin{proof}
  \
  
  One also get, since $r \rightarrow \langle 1, \mu^{\eps}_r \rangle$
  and $r \rightarrow \langle x^{- \alpha}, \mu^{\eps}_r \rangle$ are
  continuous, that $t \rightarrow \langle 1, \mu^{\eps}_t \rangle$ is
  differentiable and that
  \[ \frac{\dd}{\dd t} \langle 1, \mu^{\eps}_t \rangle = -
     \frac{1}{2} \langle 1, \mu^{\eps}_t \rangle \langle x^{- \alpha},
     \mu^{\eps}_t \rangle . \]
  Using Lemma \ref{lemma:jensen-Kalpha} with $p = 1$ and $q = 0$, one gets
  \[ \frac{\dd}{\dd t} \langle 1, \mu^{\eps}_t \rangle \le
     \langle 1, \mu_t^{\eps} \rangle^{2 + \alpha} \langle x,
     \mu_t^{\eps} \rangle^{- \alpha} = \frac{- 1}{2 \langle x,
     \mu_0^{\eps} \rangle^{\alpha}} \langle 1, \mu_t^{\eps}
     \rangle^{2 + \alpha} . \]
  This gives that
  \[ \langle 1, \mu^{\eps}_t \rangle \le \frac{\langle 1,
     \mu^{\eps}_0 \rangle}{\left( 1 + \frac{(1 + \alpha) \langle 1,
     \mu^{\eps}_0 \rangle^{1 + \alpha}}{2 \langle x,
     \mu_0^{\eps} \rangle^{\alpha}} t \right)^{\frac{1}{1 + \alpha}}} .
  \]
  Finally, using the fact that
  \[ \frac{1}{x^{\alpha}} + \frac{1}{y^{\alpha}} \ge 2^{\alpha + 1}
     \frac{1}{(x + y)^{\alpha}}, \]
  we have that
  \[ \ge (2^{1 + \alpha} - 1) \frac{1}{(x + y)^{\alpha}} \]
  \begin{align*}
    \frac{1}{x^{\alpha}} + \frac{1}{y^{\alpha}} - \frac{1}{(x + y)^{\alpha}} =
    & (1 - 2^{- (\alpha + 1)}) \left( \frac{1}{x^{\alpha}} +
    \frac{1}{y^{\alpha}} \right) + 2^{- (\alpha + 1)} \left(
    \frac{1}{x^{\alpha}} + \frac{1}{y^{\alpha}} - \frac{2^{\alpha + 1}}{(x +
    y)^{\alpha}} \right)\\
    \le & (1 - 2^{- (\alpha + 1)}) \left( \frac{1}{x^{\alpha}} +
    \frac{1}{y^{\alpha}} \right)
  \end{align*}
  
  and
  
  \begin{align*}
    \langle x^{- \alpha}, \mu^{\eps}_t \rangle = & \langle x^{-
    \alpha}, \mu^{\eps}_0 \rangle - \frac{1}{2} \int_0^t \left(
    \frac{1}{x^{\alpha}} + \frac{1}{y^{\alpha}} \right) \left(
    \frac{1}{x^{\alpha}} + \frac{1}{y^{\alpha}} - \frac{1}{(x + y)^{\alpha}}
    \right) \mu^{\eps}_r (\dd x) \mu^{\eps}_r (\dd y)\\
    \le & \langle x^{- \alpha}, \mu^{\eps}_0 \rangle - \frac{(1 -
    2^{- (\alpha + 1)})}{2} \int_0^t \left( \frac{1}{x^{\alpha}} +
    \frac{1}{y^{\alpha}} \right)^2 \mu^{\eps}_r (\dd x)
    \mu^{\eps}_r (\dd y)\\
    \le & \langle x^{- \alpha}, \mu^{\eps}_0 \rangle - (1 - 2^{-
    (\alpha + 1)}) \int_0^t \langle x^{- \alpha}, \mu_r^{\eps}
    \rangle^2 + \langle x^{- 2 \alpha}, \mu_r^{\eps} \rangle \langle 1,
    \mu_r^{\eps} \rangle \dd r\\
    \le & \langle x^{- \alpha}, \mu^{\eps}_0 \rangle - (1 - 2^{-
    (\alpha + 1)}) \int_0^t \langle x^{- \alpha}, \mu_r^{\eps}
    \rangle^2 \dd r\\
    \le & \langle x^{- \alpha}, \mu_0 \rangle .
  \end{align*}
  
  Thanks to the same argument as for $t \rightarrow \langle 1,
  \mu_t^{\eps} \rangle$, the previous computation actually shows that
  \[ \frac{\dd}{\dd t} \langle x^{- \alpha}, \mu^{\eps}_t
     \rangle \le - (1 - 2^{- (\alpha + 1)}) \langle x^{- \alpha},
     \mu_t^{\eps} \rangle^2 \]
  and we have
  \[ \langle x^{- \alpha}, \mu^{\eps}_t \rangle \le \frac{\langle
     x^{- \alpha}, \mu^{\eps}_0 \rangle}{1 + (1 - 2^{- (\alpha + 1)})
     \langle x^{- \alpha}, \mu^{\eps}_0 \rangle t} . \]
  One can inject the previous inequality to have
  \[ \frac{\dd}{\dd t} \langle 1, \mu_t^{\eps} \rangle \ge
     - \frac{1}{2} \frac{\langle x^{- \alpha}, \mu^{\eps}_0 \rangle}{1
     + (1 - 2^{- (\alpha + 1)}) \langle x^{- \alpha}, \mu^{\eps}_0
     \rangle t} \langle 1, \mu_t^{\eps} \rangle \]
  giving
  
  \begin{align*}
    \langle 1, \mu^{\eps}_t \rangle \ge & \langle 1,
    \mu^{\eps}_0 \rangle \exp \left( - \frac{1}{(2 - 2^{- \alpha})} \ln
    (1 + (1 - 2^{- (\alpha + 1)}) \langle x^{- \alpha}, \mu^{\eps}_0
    \rangle t) \right)\\
    = & \frac{\langle 1, \mu^{\eps}_0 \rangle}{(1 + (1 - 2^{- (\alpha +
    1)}) \langle x^{- \alpha}, \mu^{\eps}_0 \rangle t)^{\frac{1}{2 -
    2^{- \alpha}}}} .
  \end{align*}
  
  Remark finally that one gets for any $p > 1$
  \[ \frac{1}{x^{p \alpha}} + \frac{1}{y^{p \alpha}} \ge 2^{1 + \alpha
     p} \frac{1}{(x + y)^{p \alpha}} \]
  and by a similar fashion as before,
  
  \begin{align*}
    \frac{\dd}{\dd t} \langle x^{- \alpha p}, \mu^{\eps}_t
    \rangle \le & - (1 - 2^{- (1 + \alpha p)}) (\langle x^{- \alpha (p +
    1)}, \mu^{\eps}_t \rangle \langle 1, \mu^{\eps}_t \rangle +
    \langle x^{- \alpha p}, \mu^{\eps}_t \rangle \langle x^{- \alpha},
    \mu^{\eps}_t \rangle)\\
    \le & - (1 - 2^{- (1 + \alpha p)}) \langle x^{- \alpha (p + 1)},
    \mu^{\eps}_t \rangle \langle 1, \mu^{\eps}_t \rangle\\
    \le & - (1 - 2^{- (1 + \alpha p)}) \frac{\langle 1,
    \mu^{\eps}_0 \rangle}{(1 + (1 - 2^{- (\alpha + 1)}) \langle x^{-
    \alpha}, \mu^{\eps}_0 \rangle t)^{\frac{1}{2 - 2^{- \alpha}}}}
    \langle x^{- \alpha (p + 1)}, \mu^{\eps}_t \rangle
  \end{align*}
  
  Hence, using Lemma \ref{lemma:jensen-Kalpha}, with $p = p + 1$ and $q = p$,
  one gets
  \[ \langle x^{- \alpha (p + 1)}, \mu^{\eps}_t \rangle \ge
     \langle x^{- \alpha p}, \mu^{\eps}_t \rangle^{\frac{\alpha (p + 1)
     + 1}{\alpha p + 1}} \langle x , \mu^{\eps}_0 \rangle^{-
     \frac{\alpha}{\alpha p + 1}} = \langle x^{- \alpha p},
     \mu^{\eps}_t \rangle^{1 + \frac{\alpha}{\alpha p + 1}} \langle x ,
     \mu^{\eps}_0 \rangle^{- \frac{\alpha}{\alpha p + 1}} \]
  and we have
  
  \begin{multline*}
    \langle x^{- \alpha p}, \mu^{\eps}_0 \rangle^{-
    \frac{\alpha}{\alpha p + 1}} - \langle x^{- \alpha p}, \mu^{\eps}_t
    \rangle^{- \frac{\alpha}{\alpha p + 1}} \le - \frac{2 - 2^{-
    \alpha}}{1 - 2^{- \alpha}} \left( p + \frac{1}{\alpha} \right) \frac{1 -
    2^{- (1 + \alpha p)}}{1 - 2^{- (1 + \alpha)}} \frac{\langle 1,
    \mu^{\eps}_0 \rangle}{\langle x^{- \alpha}, \mu^{\eps}_0
    \rangle}\\
    \times (1 + (1 - 2^{- (\alpha + 1)}) \langle x^{- \alpha},
    \mu^{\eps}_0 \rangle t)^{\frac{1 - 2^{- \alpha}}{2 - 2^{- \alpha}}}
  \end{multline*}
  
  leading to the existence of a constant $c_{\alpha, p}$ such that
  \[ \langle x^{- \alpha p}, \mu^{\eps}_t \rangle \le \langle
     x^{- \alpha p}, \mu^{\eps}_0 \rangle \left( 1 + c_{\alpha, p}
     \frac{\langle 1, \mu^{\eps}_0 \rangle \langle x^{- \alpha p},
     \mu^{\eps}_0 \rangle^{\frac{\alpha}{\alpha p + 1}}}{\langle x^{-
     \alpha}, \mu^{\eps}_0 \rangle^{\frac{1}{2 - 2^{- \alpha}}}}
     {t^{\frac{1 - 2^{- \alpha}}{2 - 2^{- \alpha}}}}  \right)^{- \left( p +
     \frac{1}{\alpha} \right)} . \]
  where we have used Lemma \ref{lemma:jensen-Kalpha} with $p = 1$ and $q = -
  1$.
\end{proof}}

\

\red{\red{ATTENTION ERREUR ICI}Now, let $K > 0$ and
considere that $\mu_0, \mu_t \in \cM^{2 + \delta, K}_1$. In order to
prove that $\Gamma$ is a function from $C^0 ([0, \tau] ; \cM^{2 +
\delta, K}_1)$ to itself for a certain $\tau \le T$, one has to prove
that it defines non-negative measures, and that the continuity in time holds.
Remark that for any Borel set $A \subset \RR_+^{\ast}$, one has
\[ \langle \mathbbm{1}_A, \Gamma (\mu)_t \rangle = \mu_0 (A) + \int_0^t
   \langle \mathbbm{1}_A, L_{\alpha} (\mu_s) \rangle \dd s. \]
Remark also that
\[ \langle \mathbbm{1}_A, L_{\alpha} (\mu_s) \rangle \ge - 2 \langle 1,
   \mu_s \rangle \langle x^{- \alpha}, \mu_s \rangle \]
Remark also that if $\mu \in C^0 ([0, \tau] ; \cM^{2 + \delta, K}_1)$,
then a constant $c_{\delta} > 0$ exists such that
\[ \langle x^{- \alpha}, \mu_s \rangle \le c_{\delta} K \quad
    \text{ and } \quad \langle 1, \mu_s \rangle \le K, \]
meaning that
\[ \langle \mathbbm{1}_A, \Gamma (\mu)_t \rangle \ge \mu_0 (A) - 2
   c_{\delta} K^2 T. \]
It is then enough to take $\tau = \frac{\mu_0 (A)}{2 c_{\delta} K^2}$ to have
the non negativity of $\Gamma$.

\red{FIN ERREUR ICI}

Now, let us prove that $\Gamma (\mu)_t \in \cM_1$ for all $t \in [0,
\tau]$. Indeed, one has first that
\[ \langle 1, \Gamma (\mu)_t \rangle = \langle 1, \mu_0 \rangle - \int_0^t
   \langle 1, \mu_s \rangle \langle x^{- \alpha}, \mu_s \rangle \dd s
   \le \langle 1, \mu_0 \rangle . \]
Furthermore,
\[ \langle x^{- \alpha}, \Gamma (\mu)_t \rangle = \langle x^{- \alpha}, \mu_0
   \rangle - \int_0^t \int_{\RR^{\ast}_+} \int_{\RR^{\ast}_+}
   K_{\alpha} (x, y) \left( \frac{1}{x^{\alpha}} + \frac{1}{y^{\alpha}} -
   \frac{1}{(x + y)^{\alpha}} \right) \mu_s (\dd x) \mu_s (\dd y) . \]
Remark that for all $x, y > 0$,
\begin{equation}
  \frac{1}{x^{\alpha}} + \frac{1}{y^{\alpha}} \ge \frac{2^{\alpha +
  1}}{(x + y)^{\alpha}} \label{eq:bound-xyalpha},
\end{equation}
hence
\[ \langle x^{- \alpha}, \Gamma (\mu)_t \rangle \le \langle x^{-
   \alpha}, \mu_0 \rangle \]
and $\Gamma (\mu)_t \in \cM_1$ thanks to Proposition
\ref{prop:equivalence-M1}.

Take $s \le t$,
\[ | \langle 1, \Gamma (\mu)_t \rangle - \langle 1, \Gamma (\mu)_s \rangle |
   \le \int_s^t \langle 1, \mu_s \rangle \langle x^{- \alpha}, \mu_s
   \rangle \dd s \le | t - s | K^2 . \]
Furthermore, for any $f \in  \text{ Lip } ((\RR_+^{\ast}, d_{\alpha}) ;
\RR)$ with $f (1) = 0$ and $\llbracket f \rrbracket_{ \text{ Lip } }
\le 1$, one gets
\[ | f (x + y) - f (x) - f (y) | \le \left| \frac{1}{(x + y)^{\alpha}} -
   1 \right| + \left| \frac{1}{x^{\alpha}} - 1 \right| + \left|
   \frac{1}{y^{\alpha}} - 1 \right| \le 3 \left( \frac{1}{x^{\alpha}} +
   \frac{1}{y^{\alpha}} + 1 \right) . \]
This gives that

\begin{align*}
  | \langle f, \Gamma (\mu)_t \rangle - \langle f, \Gamma (\mu)_s \rangle | =
  & \left| \int_s^t \langle f, L_{\alpha} (\mu_r) \rangle \dd r \right|\\
  \lesssim & \int_s^t \int_{\RR^{\ast}_+} \int_{\RR^{\ast}_+}
  \left( \frac{1}{x^{\alpha}} + \frac{1}{y^{\alpha}} + 1 \right) \left(
  \frac{1}{x^{\alpha}} + \frac{1}{y^{\alpha}} \right) \mu_r (\dd x) \mu_r
  (\dd y) \dd r\\
  \lesssim & \int_s^t \int_{\RR^{\ast}_+} \int_{\RR^{\ast}_+}
  \left( \frac{1}{x^{2 \alpha}} + \frac{1}{y^{2 \alpha}} + 1 \right) \mu_r
  (\dd x) \mu_r (\dd y) \dd r\\
  \lesssim_{\delta} & (t - s) K^2 .
\end{align*}

We have just proved that, taking the supremum on $f$,
\[ W_1 (\Gamma (\mu)_t, \Gamma (\mu)_s) \le (t - s) K^2, \]
which guaranties the continuity. Finally remark that
\[ \langle 1, \Gamma (\mu)_t \rangle = \langle 1, \mu_0 \rangle - \int_s^t
   \langle 1, \mu_s \rangle \langle x^{- \alpha}, \mu_s \rangle \dd s
   \le \langle 1, \mu_0 \rangle . \]
One also gets that $\langle x, L_{\alpha} (\mu_s) \rangle = 0$ and $\langle x,
\Gamma (\mu)_t \rangle = \langle x, \mu_0 \rangle$. Finally}


\bibliographystyle{plain}
\bibliography{CV_smoluchowski}

\end{document}
